\chapter{Introduction to $\sigma$-algebra}

From the classical probability, we encountered \textit{Bertnard's Paradox}, which highlighted the significance of rotational and translational invariance in probability measurements. We discovered that probabilities, much like lengths, areas, or volumes, should remain unchanged when subjected to such transformations. For example, if two points are separated by a distance $d$, shifting them by an equal amount in the same direction preserves that distance. This invariance hints that probability is not merely a tool for quantifying uncertainty but rather a form of measure - just like length, area or volume. \\

This realization serves as our starting point for a deeper, more formal approach to understanding probability, known as \textbf{Measure Theoretic Probability}. By treating probability as a measure, we establish a rigorous mathematical foundation that allows us to precisely define, manipulate, and compute probabilities, even in complex scenarios involving infinite spaces or continuous distributions. \\

In this chapter, we will build from the fundamentals of measure theory, gradually developing the key concepts required to form a robust understanding of probability in this framework \cite{probabilityfoundations}.

\section{Introduction to $\sigma-algebra$}

In the early chapters of \textit{Real Analyis}, we introduced the concept of a \textit{field}. A field is an ordered triple, for example, $(\mathbb{Q}, +, \times)$, consisting of the set of rational numbers $\mathbb{Q}$ and two binary operations, $+$ and $\times$, defined on them. These operations follow specific properties, such as having an identity element, the existence of an inverse element for each non-zero element, and distributivity, among others. This structure forms what is commonly referred to as \textit{arithmetic or numeric algebra}.\\

But what if we change the set and the operations? Suppose instead of $\mathbb{Q}$, we take the set $\{0, 1\}^{\infty}$ (the set of all binary sequences) and define appropriate binary operations, such as $+$ and $\cdot$. The resulting structure is called a \textit{boolean algebra}. Similarly, if we take the set of matrices $M$ and define addition and multiplication operations on them, we obtain what is known as \textit{matrix algebra}. These examples illustrate that the notion of algebra is not restricted to numbers; it can be generalized to other sets with appropriate operations.\\

Now, consider a large set with many subsets as its elements, and define two operations: union $\cup$ and intersection $\cap$. This leads to what is known as a \textit{set algebra}, which is central to our discussion.\\

\begin{definition}
    Let $\Omega$ be a sample space and let $\mathcal{F}_0$ be a collection of subsets of $\Omega$. Then, $\mathcal{F}_0$ is said to be an algebra (or a field) if the following conditions hold:
\begin{enumerate}
    \item $\emptyset \in \mathcal{F}_0$.
    \item If $A \in \mathcal{F}_0$, then $A^c \in \mathcal{F}_0$.
    \item If $A \in \mathcal{F}_0$ and $B \in \mathcal{F}_0$, then $A \cup B \in \mathcal{F}_0$.
\end{enumerate}
\end{definition}

While the terms \textit{field} and \textit{algebra} are sometimes used interchangeably in the context of sets, there is a subtle difference when we generalize to other structures. A \textit{field} refers specifically to a set with two binary operations (like $+$ and $\times$) that satisfy a complete set of properties such as associativity, commutativity, distributivity, and the existence of identity and inverse elements. \\

On the other hand, an \textit{algebra} is a broader concept. It is a structure consisting of a set and operations that may or may not satisfy all the properties required of a field. For instance, in set theory, a set algebra satisfies closure under union, intersection, and complement, but it does not necessarily satisfy all the numeric properties of a field, such as the existence of multiplicative inverses. Thus, while all fields can be considered a type of algebra, not all algebras are fields. The key distinction lies in the specific operations and properties defined on the set.\\

\begin{theorem}
    An algebra is closed under finite union and finite intersection.
\end{theorem}


\begin{proof}

\textbf{Closed Under Finite Union:} \\

To prove that $\mathcal{F}_0$ is closed under finite union, we proceed by induction. \\

\textbf{Base Case:} Let $A_1, A_2 \in \mathcal{F}_0$. By definition of an algebra, $A_1 \cup A_2 \in \mathcal{F}_0$. This shows that the union of two sets in $\mathcal{F}_0$ is also in $\mathcal{F}_0$. \\

\textbf{Induction Step:} Suppose for some $n \in \mathbb{N}$, the union of $n$ sets in $\mathcal{F}_0$, say $A_1, A_2, \ldots, A_n$, is also in $\mathcal{F}_0$. That is,
\[
A_1 \cup A_2 \cup \ldots \cup A_n \in \mathcal{F}_0.
\]

Now, consider $A_1 \cup A_2 \cup \ldots \cup A_n \cup A_{n+1}$. We can rewrite this as:
\[
(A_1 \cup A_2 \cup \ldots \cup A_n) \cup A_{n+1}.
\]
By the induction hypothesis, $(A_1 \cup A_2 \cup \ldots \cup A_n) \in \mathcal{F}_0$. Since $A_{n+1} \in \mathcal{F}_0$ and $\mathcal{F}_0$ is closed under the union of two sets, it follows that:
\[
(A_1 \cup A_2 \cup \ldots \cup A_n) \cup A_{n+1} \in \mathcal{F}_0.
\]
By the principle of mathematical induction, $\mathcal{F}_0$ is closed under finite union. \\

\textbf{Closed Under Finite Intersection:} \\

To show closure under finite intersection, note that for any sets $A, B \in \mathcal{F}_0$, we have $A^c, B^c \in \mathcal{F}_0$ because complements of sets in an algebra are also in the algebra. \\

Using De Morgan's laws, we know that:
\[
A \cap B = (A^c \cup B^c)^c.
\]

Since $A^c, B^c \in \mathcal{F}_0$ and $\mathcal{F}_0$ is closed under finite union, it follows that $A^c \cup B^c \in \mathcal{F}_0$. Therefore, $(A^c \cup B^c)^c \in \mathcal{F}_0$, meaning $A \cap B \in \mathcal{F}_0$. \\

By similar reasoning and using induction, it can be shown that $\mathcal{F}_0$ is closed under the intersection of any finite number of sets. Thus, $\mathcal{F}_0$ is closed under finite intersection.

\end{proof}

We have not defined the concept of \textit{event} yet. Informally, for now consider that an event is an subset of sample space that is of our interest. A natural question that arises at this point is \textit{Is the structure of an algebra enough to study events of typical interest in probability theory?} 

\subsubsection{An Event Not Included in an Algebra}

Consider the following example. Toss a coin repeatedly until the first heads appears. The sample space is:
\[
\Omega = \{H, TH, TTH, \ldots\}
\]
where $H$ represents heads appearing on the first toss, $TH$ represents tails followed by heads, $TTH$ represents two tails followed by heads, and so on. \\

Now, suppose we are interested in determining whether the number of tosses before seeing a head is even. Let $E$ denote this event. Then,
\[
E = \{TH, TTTH, TTTTTH, \ldots\}
\]
which includes all outcomes where heads appears after an even number of tosses. \\

Notice that $E$ is a countably infinite union of individual outcomes:
\[
E = \{TH\} \cup \{TTTH\} \cup \{TTTTTH\} \cup \ldots
\]

However, an \textit{algebra} is defined to contain only finite unions of subsets. Since $E$ involves a countably infinite union, it cannot be part of the algebra of subsets of $\Omega$. This shows that our \textit{event} of interest is not included in the algebra. \\

This limitation motivates the need for a more comprehensive structure called a \textit{$\sigma$-algebra}. A $\sigma$-algebra extends the notion of an algebra by allowing countably infinite unions of subsets, ensuring that events like $E$ are included within the framework of probability theory.


\begin{definition}
    A collection $\mathcal{F}$ of subsets of $\Omega$ is called a \textit{$\sigma$-algebra} (or \textit{$\sigma$-field}) if:

\begin{enumerate}
    \item $\emptyset \in \mathcal{F}$.
    \item If $A \in \mathcal{F}$, then $A^c \in \mathcal{F}$ (i.e., the complement of $A$ is also in $\mathcal{F}$).
    \item If $A_1, A_2, A_3, \ldots$ is a countable collection of subsets in $\mathcal{F}$, then $\bigcup\limits_{i=1}^{\infty} A_i \in \mathcal{F}$.
\end{enumerate}
\end{definition}

\noindent Note that, unlike an algebra, a $\sigma$-algebra is closed under countable union and countable intersection.

\subsubsection{Examples of $\sigma-algebras$:}

Here are some intuitive examples of $\sigma$-algebras:

\begin{enumerate}
    \item \textbf{Trivial $\sigma$-algebra:} The smallest $\sigma$-algebra on a sample space $\Omega$ is $\mathcal{F} = \{\emptyset, \Omega\}$. This is known as the trivial $\sigma$-algebra and contains only the empty set and the entire sample space.

    \item \textbf{Power Set $\sigma$-algebra:} The largest $\sigma$-algebra on a sample space $\Omega$ is the power set of $\Omega$, denoted as $2^{\Omega}$. It includes all possible subsets of $\Omega$. This is the most comprehensive $\sigma$-algebra possible on $\Omega$.

    \item \textbf{Finite and Countable $\sigma$-algebras:} Consider a finite or countable sample space, such as $\Omega = \{1, 2, 3, \ldots\}$. The collection of all subsets of $\Omega$ forms a $\sigma$-algebra, as it is closed under countable unions, intersections, and complements.\\
\end{enumerate}


\begin{theorem}
    \textit{Every $\sigma$-algebra is an algebra, but the converse is not true.} 
\end{theorem} 

\begin{proof}
    
\textbf{Part 1: Every $\sigma$-algebra is an algebra}\\

Let $\mathcal{F}$ be a $\sigma$-algebra.

\begin{enumerate}
    \item \textbf{Contains the empty set:} By the definition of a $\sigma$-algebra, we have $\emptyset \in \mathcal{F}$.
    
    \item \textbf{Closed under complementation:} If $A \in \mathcal{F}$, then by definition, $A^c \in \mathcal{F}$.
    
    \item \textbf{Closed under finite unions:} Let $A, B \in \mathcal{F}$. We can consider the finite union:
    \[
    A \cup B = A \cup B = \bigcup_{i=1}^{2} A_i.
    \]
    Here, we can denote $A_1 = A$ and $A_2 = B$. Since $\mathcal{F}$ is closed under countable unions, we have:
    \[
    A \cup B \in \mathcal{F}.
    \]
\end{enumerate}

Since $\mathcal{F}$ satisfies all three properties of an algebra, we conclude that every $\sigma$-algebra is indeed an algebra.\\

\textbf{Part 2: The converse is not true}\\

To show that not every algebra is a $\sigma$-algebra, we can provide a counterexample.\\

Consider the set $\Omega = \{1, 2, 3\}$ and the algebra $\mathcal{A} = \{ \emptyset, \{1\}, \{2\}, \{3\}, \{1, 2\}, \{1, 3\}, \{2, 3\},$
$ \{1, 2, 3\} \}$.

\begin{enumerate}
    \item \textbf{Contains the empty set:} $\emptyset \in \mathcal{A}$.
    
    \item \textbf{Closed under complementation:} The complement of each set in $\mathcal{A}$ is also in $\mathcal{A}$.
    
    \item \textbf{Closed under finite unions:} The union of any finite number of sets in $\mathcal{A}$ is also in $\mathcal{A}$.
\end{enumerate}

However, the collection $\mathcal{A}$ is not a $\sigma$-algebra because it is not closed under countable unions. For instance, if we consider the countable collection of subsets:
\[
A_1 = \{1\}, \quad A_2 = \{2\}, \quad A_3 = \{3\}, \quad \ldots
\]
the union $\bigcup_{i=1}^{\infty} A_i = \{1, 2, 3\} = \Omega$, which is included, but if we consider an infinite union of disjoint sets from $\mathcal{A}$ that leads to more than three elements, it will not be contained within $\mathcal{A}$.\\

Thus, we conclude that not every algebra is a $\sigma$-algebra.
\end{proof}

\subsubsection{Examples of algebras which are not $\sigma$-algebras:}

Below are some simple examples of an algebra that is not a $\sigma$-algebra: \\

\textbf{Example 1: The Finite Subsets of $\mathbb{N}$}\\

Consider the set $\Omega = \mathbb{N}$, the set of all natural numbers. Let $\mathcal{A}$ be the collection of all finite subsets of $\mathbb{N}$ along with $\mathbb{N}$ itself. This collection forms an \textbf{algebra} because:
\begin{itemize}
    \item The union or intersection of any two finite sets is finite (or possibly $\mathbb{N}$).
    \item The complement of any finite subset is also an infinite subset, and in this case, it is $\mathbb{N}$ (which belongs to $\mathcal{A}$).
\end{itemize}

However, $\mathcal{A}$ is \textbf{not a $\sigma$-algebra} because it is not closed under countable union. For instance, if we take a sequence of singletons $\{1\}, \{2\}, \{3\}, \ldots$, the union of these singletons is $\mathbb{N}$, which is an infinite set. While $\mathbb{N}$ is in $\mathcal{A}$, the complement of this countable union would not necessarily belong to $\mathcal{A}$, as it may not be finite.\\

\textbf{Example 2: Intervals on the Real Line}\\

Consider $\Omega = [0, 1]$ and let $\mathcal{A}$ be the collection of all finite unions of intervals of the form $[a, b]$, where $0 \leq a \leq b \leq 1$. This collection $\mathcal{A}$ forms an \textbf{algebra} because:
\begin{itemize}
    \item The union and intersection of a finite number of intervals of this form are again finite unions of intervals of this form.
    \item The complement of a finite union of such intervals is also a finite union of intervals.
\end{itemize}

However, $\mathcal{A}$ is \textbf{not a $\sigma$-algebra} because it is not necessarily closed under countable unions. For example, if we take a sequence of intervals $\left[0, \frac{1}{2}\right], \left[\frac{1}{2}, \frac{3}{4}\right], \left[\frac{3}{4}, \frac{7}{8}\right], \ldots$ such that they cover $[0, 1]$ as a whole, their countable union would be $[0, 1]$. Although $[0, 1]$ is in $\mathcal{A}$, the structure of $\mathcal{A}$ doesn’t guarantee closure under all such countable unions. \\

\textbf{Example 3: The Power Set of a Finite Set} \\

Let $\Omega = \{a, b, c\}$ be a finite set. The collection $\mathcal{A}$ of all subsets of $\Omega$ (also known as the power set of $\Omega$) forms an \textbf{algebra} because:
\begin{itemize}
    \item Any union, intersection, or complement of subsets of a finite set remains a subset of that finite set.
\end{itemize}

However, even though this is a trivial example, it demonstrates that an algebra is not necessarily a $\sigma$-algebra because $\sigma$-algebras are designed to handle infinite cases. In this finite scenario, $\mathcal{A}$ satisfies the properties of both an algebra and a $\sigma$-algebra, but it shows that if the set $\Omega$ were infinite, $\mathcal{A}$ would not generally be closed under countable operations.


\begin{example}
    Consider the random experiment of throwing a die. If a statistician is interested in the occurrence of either an odd or an even outcome, construct a sample space and a $\sigma$-algebra of subsets of this sample space.\\

    \textbf{Sample Space (\(\Omega\))}:  The sample space consists of all possible outcomes when throwing a six-sided die. Therefore, we can define the sample space as:
\[
\Omega = \{1, 2, 3, 4, 5, 6\}
\]

\textbf{Events of Interest}:  The statistician is interested in the occurrence of either an odd or an even outcome. We can categorize the outcomes as follows:
\begin{itemize}
    \item \textbf{Odd Outcomes}: \( \{1, 3, 5\} \)
    \item \textbf{Even Outcomes}: \( \{2, 4, 6\} \)
\end{itemize}

\textbf{Constructing the \(\sigma\)-Algebra (\(\mathcal{F}\))}: A \(\sigma\)-algebra is a collection of subsets of \(\Omega\) that satisfies the following properties:
\begin{itemize}
    \item It contains the empty set and the sample space itself.
    \item It is closed under complementation.
    \item It is closed under countable unions.
\end{itemize}

Given the events of interest, we can construct the \(\sigma\)-algebra as follows:
\[
\mathcal{F} = \{ \emptyset, \{1\}, \{2\}, \{3\}, \{4\}, \{5\}, \{6\}, \{1, 3, 5\}, \{2, 4, 6\}, \{1, 2, 3, 4, 5, 6\} \}
\]

\textbf{Checking the Properties of the \(\sigma\)-Algebra}:\\

 \textbf{Contains the Empty Set and Sample Space}: \(\emptyset \in \mathcal{F}\) and \(\Omega = \{1, 2, 3, 4, 5, 6\} \in \mathcal{F}\).\\
    
\textbf{Closed under Complementation}: 
    \begin{itemize}
        \item The complement of \(\emptyset\) is \(\{1, 2, 3, 4, 5, 6\}\), which is in \(\mathcal{F}\).
        \item The complement of \(\{1, 3, 5\}\) is \(\{2, 4, 6\}\), which is in \(\mathcal{F}\).
        \item The complement of \(\{2, 4, 6\}\) is \(\{1, 3, 5\}\), which is in \(\mathcal{F}\).
    \end{itemize}
    
\textbf{Closed under Countable Unions}: For any events in \(\mathcal{F}\), the union will also be in \(\mathcal{F}\). For instance, 
    \(\{1\} \cup \{2\} = \{1, 2\} \in \mathcal{F}\), and 
    \(\{1, 3, 5\} \cup \{2, 4, 6\} = \{1, 2, 3, 4, 5, 6\} \in \mathcal{F}\).\\

\end{example}

\begin{example}
    Let \( A_1, A_2, \ldots, A_n \) be arbitrary subsets of \( \Omega \). Describe (explicitly) the smallest \(\sigma\)-algebra \( \mathcal{F} \) containing \( A_1, A_2, \ldots, A_n \). How many sets are there in \( \mathcal{F} \)? (Give an attainable upper bound under certain conditions). List all the sets in \( \mathcal{F} \) for \( n = 2 \).\\

    \textbf{Smallest \(\sigma\)-algebra containing \( A_1, A_2, \ldots, A_n \):}  \\

The smallest \(\sigma\)-algebra \( \mathcal{F} \) containing the subsets \( A_1, A_2, \ldots, A_n \) is generated by these sets. This means \( \mathcal{F} \) includes all possible unions, intersections, and complements of these sets.\\

To explicitly describe \( \mathcal{F} \):\\
1. Include \( A_1, A_2, \ldots, A_n \).\\
2. Include the complements of each set: \( A_1^c, A_2^c, \ldots, A_n^c \).\\
3. Include all possible unions and intersections of these sets and their complements.\\

\textbf{Counting the Sets in \( \mathcal{F} \):}  \\

In the worst-case scenario, if \( A_1, A_2, \ldots, A_n \) are arbitrary subsets with no restrictions, the number of distinct sets that can be formed is determined by the combinations of unions and intersections. An attainable upper bound for the number of sets in \( \mathcal{F} \) can be given by:
\[
|\mathcal{F}| \leq 2^{2^n}
\]
This upper bound arises from considering all subsets of \( \Omega \) formed by the possible intersections of the \( 2n \) sets (including both original sets and their complements).\\

\textbf{Example for \( n = 2 \):} \\

Let \( A_1 \) and \( A_2 \) be two arbitrary subsets of \( \Omega \). The smallest \(\sigma\)-algebra \( \mathcal{F} \) generated by \( A_1 \) and \( A_2 \) contains the following sets: \( A_1 \), \( A_2 \), \( A_1^c \), \( A_2^c \), \( A_1 \cap A_2 \), \( A_1 \cap A_2^c \), \( A_1^c \cap A_2 \) and \( A_1^c \cap A_2^c \).\\

Thus, the sets in \( \mathcal{F} \) when \( n = 2 \) are:
\[
\mathcal{F} = \{ A_1, A_2, A_1^c, A_2^c, A_1 \cap A_2, A_1 \cap A_2^c, A_1^c \cap A_2, A_1^c \cap A_2^c \}\\
\]
\end{example}

\begin{example}
    Let \( F \) and \( G \) be two \(\sigma\)-algebras of subsets of \(\Omega\). \\
    \textbf{(a)} Is \( F \cup G \), the collection of subsets of \(\Omega\) lying in either \( F \) or \( G \), a \(\sigma\)-algebra?\\
    \textbf{(b)} Show that \( F \cap G \), the collection of subsets of \(\Omega\) lying in both \( F \) and \( G \), is a \(\sigma\)-algebra.\\
    \textbf{(c)} Generalize (b) to arbitrary intersections as follows. Let \( I \) be an arbitrary index set (possibly uncountable), and let \( \{F_i\}_{i \in I} \) be a collection of \(\sigma\)-algebras on \(\Omega\). Show that \( \bigcap_{i \in I} F_i \) is also a \(\sigma\)-algebra. \\

To determine whether \( F \cup G \) is a \(\sigma\)-algebra, we need to check the three properties:\\

\textbf{Contains the empty set and sample space:} Since both \( F \) and \( G \) are \(\sigma\)-algebras, they each contain \( \emptyset \) and \( \Omega \). Thus, \( F \cup G \) contains both \( \emptyset \) and \( \Omega \).\\
    
\textbf{Closed under complementation:} Let \( A \in F \cup G \). If \( A \in F \), then \( A^c \in F \) (since \( F \) is a \(\sigma\)-algebra), and similarly for \( G \). However, \( A^c \) might not be in \( F \cup G \) if \( A \) is in one algebra but not in the other. Thus, \( F \cup G \) is not closed under complementation.\\

\textbf{Closed under countable unions:} Let \( A_1, A_2, \ldots \in F \cup G \). If all \( A_i \) are in \( F \), then \( \bigcup_{i=1}^\infty A_i \in F \). If all \( A_i \) are in \( G \), then \( \bigcup_{i=1}^\infty A_i \in G \). However, if some \( A_i \) are in \( F \) and some in \( G \), \( \bigcup_{i=1}^\infty A_i \) may not be in \( F \cup G \). Therefore, \( F \cup G \) is not closed under countable unions.\\

Hence, \( F \cup G \) is \textbf{not a \(\sigma\)-algebra}.\\

To show that \( F \cap G \) is a \(\sigma\)-algebra, we verify the three properties:\\

\textbf{Contains the empty set and sample space:} Since both \( F \) and \( G \) contain \( \emptyset \) and \( \Omega \), we have \( \emptyset \in F \cap G \) and \( \Omega \in F \cap G \).\\
    
\textbf{Closed under complementation:} Let \( A \in F \cap G \). Then \( A \in F \) and \( A \in G \). Thus, \( A^c \in F \) and \( A^c \in G \), which implies \( A^c \in F \cap G \).\\
    
\textbf{Closed under countable unions:} Let \( A_1, A_2, \ldots \in F \cap G \). Then \( A_i \in F \) for all \( i \) and \( A_i \in G \) for all \( i \). Thus, \( \bigcup_{i=1}^\infty A_i \in F \) and \( \bigcup_{i=1}^\infty A_i \in G \), which implies \( \bigcup_{i=1}^\infty A_i \in F \cap G \).\\

Therefore, \( F \cap G \) \textbf{is a \(\sigma\)-algebra}.\\

To prove that \( \bigcap_{i \in I} F_i \) is a \(\sigma\)-algebra, we check the three properties:\\

\textbf{Contains the empty set and sample space:} Since each \( F_i \) contains \( \emptyset \) and \( \Omega \), we have \( \emptyset \in \bigcap_{i \in I} F_i \) and \( \Omega \in \bigcap_{i \in I} F_i \).\\
    
\textbf{Closed under complementation:} Let \( A \in \bigcap_{i \in I} F_i \). Then \( A \in F_i \) for all \( i \). Thus, \( A^c \in F_i \) for all \( i \), which implies \( A^c \in \bigcap_{i \in I} F_i \).\\

\textbf{Closed under countable unions:} Let \( A_1, A_2, \ldots \in \bigcap_{i \in I} F_i \). Then \( A_j \in F_i \) for all \( j \) and for all \( i \). Thus, \( \bigcup_{j=1}^\infty A_j \in F_i \) for all \( i \), which implies \( \bigcup_{j=1}^\infty A_j \in \bigcap_{i \in I} F_i \).\\

Therefore, \( \bigcap_{i \in I} F_i \) \textbf{is a \(\sigma\)-algebra}.\\

\end{example}

\begin{example}
    Let \(\Omega\) be an arbitrary set. Answer the following questions:\\
    \textbf{(a)} Is the collection \(F_1\) consisting of all finite subsets of \(\Omega\) an algebra?\\
    \textbf{(b)} Let \(F_2\) consist of all finite subsets of \(\Omega\) and all subsets of \(\Omega\) having a finite complement.\\
    Is \(F_2\) an algebra?\\
    \textbf{(c)} Is \(F_2\) a \(\sigma\)-algebra?\\
    \textbf{(d)} Let \(F_3\) consist of all countable subsets of \(\Omega\) and all subsets of \(\Omega\) having a countable complement. Is \(F_3\) a \(\sigma\)-algebra?\\


To determine if \(F_1\) is an algebra, we must check the three properties:

    \begin{enumerate}
        \item \textbf{Contains the empty set:} \(\emptyset \in F_1\) since the empty set is a finite subset.
        
        \item \textbf{Closed under complementation:} If \(A \in F_1\) (i.e., \(A\) is a finite subset of \(\Omega\)), then its complement \(A^c\) may not be finite. Therefore, \(F_1\) is not closed under complementation.
        
        \item \textbf{Closed under finite unions:} If \(A, B \in F_1\), then \(A \cup B\) is also finite, so \(F_1\) is closed under finite unions.
    \end{enumerate}

    Since \(F_1\) fails to be closed under complementation, we conclude that \(F_1\) is \textbf{not an algebra}.\\

To check if \(F_2\) is an algebra, we verify the properties:

\begin{enumerate}
    \item \textbf{Contains the empty set:} \(\emptyset \in F_2\) since it is a finite subset.
    
    \item \textbf{Closed under complementation:}
    \begin{itemize}
        \item If \(A \in F_2\) is finite, then \(A^c\) has a finite complement, which is infinite. Thus, it is in \(F_2\).
        \item If \(B \in F_2\) has a finite complement, then \(B^c\) is finite. Therefore, \(B^c \in F_2\).
    \end{itemize}
    Hence, \(F_2\) is closed under complementation.

    \item \textbf{Closed under finite unions:}
    \begin{itemize}
        \item If \(A, B \in F_2\) are both finite, then \(A \cup B\) is finite.
        \item If \(A\) is finite and \(B\) has a finite complement, then \(A \cup B\) has a finite complement.
        \item If both \(A\) and \(B\) have finite complements, then \((A \cup B)^c = A^c \cap B^c\), which is finite.
    \end{itemize}
\end{enumerate}

Thus, \(F_2\) is closed under finite unions.\\

To determine if \(F_2\) is a \(\sigma\)-algebra, we need to check the closure under countable unions.\\

Consider the countable union of finite sets:
\[
A_1 = \{1\}, A_2 = \{2\}, A_3 = \{3\}, \ldots
\]
Then,
\[
\bigcup_{i=1}^{\infty} A_i = \{1, 2, 3, \ldots\}
\]
which is not finite. Therefore, \(F_2\) is not closed under countable unions.\\

Thus, \(F_2\) is \textbf{not a \(\sigma\)-algebra}.\\

To check if \(F_3\) is a \(\sigma\)-algebra, we verify:

\begin{enumerate}
    \item \textbf{Contains the empty set:} \(\emptyset \in F_3\) since it is countable.

    \item \textbf{Closed under complementation:}
    \begin{itemize}
        \item If \(A \in F_3\) is countable, then \(A^c\) has a countable complement.
        \item If \(B \in F_3\) has a countable complement, then \(B^c\) is countable.
    \end{itemize}
    Hence, \(F_3\) is closed under complementation.

    \item \textbf{Closed under countable unions:}
    \begin{itemize}
        \item If \(A_1, A_2, A_3, \ldots\) are countable sets, then
        \[
        \bigcup_{i=1}^{\infty} A_i
        \]
        is also countable.
        \item If \(B\) has a countable complement, then
        \[
        B^c \in F_3 \implies B^c = \bigcup_{i=1}^{\infty} C_i \quad \text{for } C_i \text{ countable}.
        \]
        Therefore, \(B\) itself is in \(F_3\).
    \end{itemize}
\end{enumerate}

Since \(F_3\) satisfies all properties, we conclude that \(F_3\) is \textbf{a \(\sigma\)-algebra}.\\

\end{example}

\begin{example}
    Let \( X \) and \( Y \) be two sets and let \( f: X \rightarrow Y \) be a function. If \( F \) is a \(\sigma\)-algebra over the subsets of \( Y \), and \( G = \{ A \mid \exists B \in F \text{ such that } f^{-1}(B) = A \} \), does \( G \) form a \(\sigma\)-algebra of subsets of \( X \)?  Note that \( f^{-1}(N) \) is the notation used for the pre-image of set \( N \) under the function \( f \) for some \( N \subseteq Y \). That is, \( f^{-1}(N) = \{x \in X \mid f(x) \in N\} \) for some \( N \subseteq Y \).\\

    To show that \( G \) forms a \(\sigma\)-algebra of subsets of \( X \), we need to verify that \( G \) satisfies the three properties of a \(\sigma\)-algebra:\\

\textbf{Contains the empty set}:  The \(\sigma\)-algebra \( F \) over \( Y \) contains the empty set, \( \emptyset \). Let \( B = \emptyset \in F \). Then the pre-image under \( f \), \( f^{-1}(\emptyset) = \emptyset \), is also in \( G \). Therefore, \( \emptyset \in G \).\\

\textbf{Closed under complementation}:  Let \( A \in G \). By definition of \( G \), there exists a set \( B \in F \) such that \( f^{-1}(B) = A \). We need to show that \( A^c \in G \). Consider the complement of \( B \) in \( F \), denoted as \( B^c \). Since \( F \) is a \(\sigma\)-algebra, \( B^c \in F \).  \\

    Now, observe that:
    \[
    f^{-1}(B^c) = \{x \in X \mid f(x) \notin B\} = A^c
    \]
    Hence, \( A^c \in G \), showing that \( G \) is closed under complementation.\\

    \textbf{Closed under countable unions}: Let \( \{A_i\}_{i=1}^{\infty} \) be a countable collection of sets in \( G \). For each \( A_i \in G \), there exists \( B_i \in F \) such that \( f^{-1}(B_i) = A_i \). Since \( F \) is a \(\sigma\)-algebra, it is closed under countable unions, so \( \bigcup_{i=1}^{\infty} B_i \in F \). \\

    Now, consider the pre-image:
    \[
    f^{-1}\left(\bigcup_{i=1}^{\infty} B_i\right) = \bigcup_{i=1}^{\infty} f^{-1}(B_i) = \bigcup_{i=1}^{\infty} A_i
    \]
    Therefore, \( \bigcup_{i=1}^{\infty} A_i \in G \), showing that \( G \) is closed under countable unions.

\end{example}


\section{Measurable Space}


Consider a large canvas $\Omega$, which represents an entire art studio wall. The goal is to measure the total amount of paint used on specific regions of the wall. However, you don't want to measure paint usage for every possible shape or region on the wall (which could be infinitely complex). Instead, you decide to focus only on certain, manageable regions such as: rectangles, circles, simple polygons, and unions and intersections of these shapes.\\

These regions form a collection, say $\mathcal{F}$, which represents all the shapes and combinations that you are interested in measuring. The pair $(\Omega, \mathcal{F})$ then becomes a \textit{measurable space}, where:
\begin{itemize}
    \item $\Omega$ is the entire canvas, representing all possible points on the wall.
    \item $\mathcal{F}$ is a collection of specific shapes (rectangles, circles, etc.) and their combinations, which are the regions you can measure the amount of paint for.
\end{itemize}

The measurable sets in $\mathcal{F}$ are those specific shapes and combinations that you have chosen to focus on, similar to how measurable sets in probability theory are those events that belong to a specific $\sigma$-algebra. In context of probability theory, imagine you have a sample space $\Omega$, which represents all the possible outcomes of an experiment. For instance, if you flip a coin, the sample space is $\Omega = \{\text{Heads}, \text{Tails}\}$.\\

A measurable space is a pair $(\Omega, \mathcal{F})$, where:
\begin{itemize}
    \item $\Omega$ is the sample space, representing all possible outcomes.
    \item $\mathcal{F}$ is a $\sigma$-algebra of subsets of $\Omega$. It is a collection of subsets that includes the empty set, is closed under complements, and closed under countable unions.
\end{itemize}

The subsets in $\mathcal{F}$ are the ones we can \textit{measure}, hence the term \textit{measurable space}.

\begin{definition}
    The 2-tuple $(\Omega, \mathcal{F})$ is called a \textbf{measurable space}. Here:
    \begin{itemize}
        \item $\Omega$ is the sample space, a non-empty set.
        \item $\mathcal{F}$ is a $\sigma$-algebra on $\Omega$, meaning it satisfies:
        \begin{enumerate}
            \item $\emptyset \in \mathcal{F}$.
            \item If $A \in \mathcal{F}$, then $A^c \in \mathcal{F}$ (closure under complements).
            \item If $A_1, A_2, A_3, \ldots \in \mathcal{F}$, then $\bigcup\limits_{i=1}^{\infty} A_i \in \mathcal{F}$ (closure under countable unions).
        \end{enumerate}
    \end{itemize}
\end{definition}

\begin{definition}
    Every member of the $\sigma$-algebra $\mathcal{F}$ is called an \textbf{$\mathcal{F}$-measurable set} in the context of measure theory. These are the subsets of $\Omega$ that we can measure using the $\sigma$-algebra $\mathcal{F}$.
\end{definition}

\begin{definition}
    $\mathcal{F}$-measurable sets are called \textbf{events}. This means that an event is not just any subset of $\Omega$, but one that belongs to the $\sigma$-algebra $\mathcal{F}$ under consideration.
\end{definition}

\subsubsection{Examples Where Subsets of $\Omega$ Are Not $\mathcal{F}$-Measurable Sets:}

In measure theory and probability, not all subsets of a sample space $\Omega$ are necessarily $\mathcal{F}$-measurable sets. This depends on the construction of the $\sigma$-algebra $\mathcal{F}$ associated with $\Omega$. Below are some examples where subsets of $\Omega$ are not $\mathcal{F}$-measurable: \\

\textbf{Example 1: A Countable Union Not in an Algebra} \\

Let $\Omega = \mathbb{N}$, the set of natural numbers, and let $\mathcal{F}$ be an \textbf{algebra} consisting of all finite subsets of $\mathbb{N}$ and their complements (which are cofinite sets). In this setup, $\mathcal{F}$ contains only finite unions and intersections.\\

Now, consider the subset $A = \{2, 4, 6, \ldots\}$, the set of all even numbers. This is an \textbf{infinite} set but not cofinite (its complement, the set of odd numbers, is also infinite). Since $\mathcal{F}$ only contains finite or cofinite sets, $A$ is not in $\mathcal{F}$. Thus, $A$ is an example of a subset of $\Omega$ that is not $\mathcal{F}$-measurable.\\

\textbf{Example 2: Subsets in the Cantor Set}\\

Let $\Omega$ be the \textbf{Cantor set}, which is a subset of the interval $[0, 1]$. Construct $\mathcal{F}$ to be the $\sigma$-algebra generated by all \textbf{intervals} in $[0, 1]$. While $\mathcal{F}$ will contain many subsets, it will not include certain highly irregular subsets of the Cantor set that are not expressible as a countable union, intersection, or complement of intervals.\\

For instance, a subset of the Cantor set that is formed using a complex pattern based on the binary expansion of its elements may not be measurable in this $\sigma$-algebra. Hence, such a subset would not be $\mathcal{F}$-measurable.\\

\textbf{Example 3: Students in a Class}\\

Imagine a class of 30 students, represented by the set:

\[
\Omega = \{s_1, s_2, s_3, \ldots, s_{30}\}
\]

Define a \( \sigma \)-algebra \( \mathcal{F} \) that consists only of subsets containing an even number of students. This \( \sigma \)-algebra could include sets like:

\[
\mathcal{F} = \{\emptyset, \{s_1, s_2\}, \{s_3, s_4\}, \{s_5, s_6\}, \ldots, \{s_1, s_2, s_3, s_4\}, \ldots, \{s_1, s_2, \ldots, s_{30}\}\}
\]

Now, consider the subset of interest \( A = \{s_1, s_3, s_5, s_7, \ldots, s_{29}\} \), which contains all the odd-numbered students in the class.\\

In the \( \sigma \)-algebra \( \mathcal{F} \), every set is constructed to contain only an even number of students. The set \( A \), which contains an odd number of students, cannot be expressed as a union or intersection of sets from \( \mathcal{F} \).\\

Thus, \( A \) is not \( \mathcal{F} \)-measurable because it does not fit within the constraints of our \( \sigma \)-algebra.\\

\textbf{Example 4: Days of Week}\\

Let \( \Omega = \{\text{Monday, Tuesday, Wednesday, Thursday, Friday, Saturday, Sunday}\} \). \\

Suppose we define a \( \sigma \)-algebra \( \mathcal{F} \) that only includes subsets that contain weekdays:
\[
\mathcal{F} = \{\emptyset, \{\text{Monday, Tuesday, Wednesday, Thursday, Friday}\}, \{\text{Saturday, Sunday}\}, \Omega\}
\]

Now, consider the subset of interest \( A = \{\text{Saturday}\} \).\\

The \( \sigma \)-algebra \( \mathcal{F} \) only contains the sets of weekdays and their complements but does not include individual weekend days like Saturday. Thus, \( A \) cannot be constructed as a union or intersection of sets in \( \mathcal{F} \).\\

Since \( A \) cannot be represented within the existing \( \sigma \)-algebra \( \mathcal{F} \), it is not \( \mathcal{F} \)-measurable.

\subsection{Measure}

\begin{definition}
    Let \((\Omega, \mathcal{F})\) be a measurable space. A \textit{measure} on \((\Omega, \mathcal{F})\) is a function \(\mu: \mathcal{F} \to [0, \infty]\) such that:

\begin{enumerate}
    \item \(\mu(\emptyset) = 0.\)
    \item If \(\{A_i, i \geq 1\}\) is a sequence of disjoint sets in \(\mathcal{F}\), then the measure of the union of these countably infinite disjoint sets is equal to the sum of the measures of the individual sets:
    \[
    \mu\left(\bigcup_{i=1}^{\infty} A_i\right) = \sum_{i=1}^{\infty} \mu(A_i).
    \]
\end{enumerate}
\end{definition}

The second property stated above is known as the \textit{countable additivity property} of measures. From the definition, it is clear that a measure can only be assigned to elements of \(\mathcal{F}\).

\begin{definition}
    The triplet \((\Omega, \mathcal{F}, \mu)\) is called a \textit{measure space}. 
\end{definition}

The measure \(\mu\) is said to be a \textit{finite measure} if \(\mu(\Omega) < \infty\); otherwise, \(\mu\) is said to be an \textit{infinite measure}. In particular, if \(\mu(\Omega) = 1\), then \(\mu\) is referred to as a \textit{probability measure}.


\subsection{Probability Measure}

\begin{definition}
    A \textit{probability measure} is a function \( P: \mathcal{F} \to [0, 1] \) such that:

\begin{enumerate}
    \item \( P(\emptyset) = 0 \).
    \item \( P(\Omega) = 1 \).
    \item \textbf{Countable Additivity:} If \( \{A_i, i \geq 1\} \) is a sequence of disjoint sets in \( \mathcal{F} \), then
    \[
    P\left(\bigcup_{i=1}^{\infty} A_i\right) = \sum_{i=1}^{\infty} P(A_i).
    \]
\end{enumerate}
\end{definition}

\begin{definition}
    The triplet \( (\Omega, \mathcal{F}, P) \) is called a \textit{probability space}, and the three properties stated above are referred to as the \textit{axioms of probability}.
\end{definition}

It is clear from the definition that probabilities are defined only for elements of \( \mathcal{F} \), and not necessarily for all subsets of \( \Omega \). In other words, probability measures are assigned only to \textit{events}. Even when we speak of the probability of an elementary outcome \( \omega \), it should be interpreted as the probability assigned to the singleton set \( \{\omega\} \) (assuming, of course, that the singleton is an event).

\subsection{Properties of Probability Measure}

We will derive some fundamental properties of probability measures, which follow directly from the axioms of probability. In what follows, \((\Omega, \mathcal{F}, P)\) is a probability space. \\

\textbf{Property 1:} Suppose \( A \) be a subset of \(\Omega\) such that \( A \in \mathcal{F} \). Then,
\[
P(A^c) = 1 - P(A). 
\]
\textit{Proof:} Given any subset \( A \subseteq \Omega \), \( A \) and \( A^c \) partition the sample space. Hence, \( A^c \cup A = \Omega \) and \( A^c \cap A = \emptyset \). By the "Countable Additivity" axiom of probability, \( P(A^c \cup A) = P(A) + P(A^c) \). Therefore, \( P(\Omega) = P(A) + P(A^c) \implies P(A^c) = 1 - P(A) \).\\

\textbf{Property 2:} Consider events \( A \) and \( B \) such that \( A \subseteq B \) and \( A, B \in \mathcal{F} \). Then \( P(A) \leq P(B) \).\\

\textit{Proof:} The set \( B \) can be written as the union of two disjoint sets \( A \) and \( A^c \cap B \). Therefore, we have \( P(A) + P(A^c \cap B) = P(B) \implies P(A) \leq P(B) \) since \( P(A^c \cap B) \geq 0 \).\\

\textbf{Property 3:} (Finite Additivity) If \( A_1, A_2, \ldots, A_n \) are a finite number of disjoint events, then
\[
P\left(\bigcup_{i=1}^{n} A_i\right) = \sum_{i=1}^{n} P(A_i). 
\]
\textit{Proof:} This property follows directly from the axiom of countable additivity of probability measures. It is obtained by setting the events \( A_{n+1}, A_{n+2}, \ldots \) as empty sets. The left-hand side (LHS) will simplify as:
\[
P\left(\bigcup_{i=1}^{\infty} A_i\right) = P\left(\bigcup_{i=1}^{n} A_i\right).
\]
The right-hand side (RHS) can be manipulated as follows:
\[
\sum_{i=1}^{\infty} P(A_i) \overset{(a)}{=} \lim_{k \to \infty} \sum_{i=1}^{k} P(A_i) = \sum_{i=1}^{n} P(A_i) + \lim_{k \to \infty} \sum_{i=n+1}^{k} P(A_i) \overset{(b)}{=} \sum_{i=1}^{n} P(A_i) + \lim_{k \to \infty} 0 = \sum_{i=1}^{n} P(A_i).
\]
where (a) follows from the definition of an infinite series and (b) is a consequence of setting the events from \( A_{n+1} \) onwards to null sets.\\

\textbf{Property 4:} For any \( A, B \in F \),
\[
P(A \cup B) = P(A) + P(B) - P(A \cap B). 
\]
In general, for a family of events \( \{A_i\}_{i=1}^{n} \subset F \),
\[
P\left(\bigcup_{i=1}^{n} A_i\right) = \sum_{i=1}^{n} P(A_i) - \sum_{i < j} P(A_i \cap A_j) + \sum_{i < j < k} P(A_i \cap A_j \cap A_k) + \ldots + (-1)^{n+1} P\left(\bigcap_{i=1}^{n} A_i\right). 
\]
This property is proved using induction on \( n \). The property can be proved in a much simpler way using the concept of Indicator Random Variables, which will be discussed in the subsequent lectures.\\

\textit{Proof} The set \( A \cup B \) can be written as \( A \cup B = A \cup (A^c \cap B) \). Since \( A \) and \( A^c \cap B \) are disjoint events, \( P(A \cup B) = P(A) + P(A^c \cap B) \). Now, set \( B \) can be partitioned as \( B = (A \cap B) \cup (A^c \cap B) \). Hence, \( P(B) = P(A \cap B) + P(A^c \cap B) \). On substituting this result in the expression of \( P(A \cup B) \), we will obtain the final result that \( P(A \cup B) = P(A) + P(B) - P(A \cap B) \).\\

\textbf{Property 5:} If \( \{A_i, i \geq 1\} \) are events, then
\[
P\left(\bigcup_{i=1}^{\infty} A_i\right) = \lim_{m \to \infty} P\left(\bigcup_{i=1}^{m} A_i\right). 
\]
This result is known as the continuity of probability measures.\\

\textit{How to visualise this property?} Imagine $P_m$ is the probability of union of $A_1, A_2, ..., A_m$. Then the sequence of $P_m$'s is a monotonically increasing sequence. Also, the sequence is bounded by the interval $[0, 1]$. We know, that every monotonically increasing sequence that is bounded must converge. So the RHS of the property 5 is a finite quantity. So is the LHS because the countable union of sets is a well-definied set for which the probability measure is defined. The property says both are equal. \\

\textit{Proof:} Define a new family of sets \( B_1 = A_1, B_2 = A_2 \setminus A_1, \ldots, B_n = A_n \setminus \bigcup_{i=1}^{n-1} A_i, \ldots \).  \\

Then, the following claims are placed: \\
\textbf{Claim 1:} \( B_i \cap B_j = \emptyset, \forall i \neq j \).  \\
\textbf{Claim 2:} \( \bigcup_{i=1}^{\infty} A_i = \bigcup_{i=1}^{\infty} B_i \).  \\

Since \( \{B_i, i \geq 1\} \) is a disjoint sequence of events, and using the above claims, we get
\[
P\left(\bigcup_{i=1}^{\infty} A_i\right) = P\left(\bigcup_{i=1}^{\infty} B_i\right) = \sum_{i=1}^{\infty} P(B_i).
\]

Therefore,
\[
P\left(\bigcup_{i=1}^{\infty} A_i\right) = \sum_{i=1}^{\infty} P(B_i) \quad (a) = \lim_{m \to \infty} \sum_{i=1}^{m} P(B_i) \quad (b) = \lim_{m \to \infty} P\left(\bigcup_{i=1}^{m} B_i\right) \quad 
\]
\[
    (c) = \lim_{m \to \infty} P\left(\bigcup_{i=1}^{m} A_i\right).
\]

Here, (a) follows from the definition of an infinite series, (b) follows from Claim 1 in conjunction with the Countable Additivity axiom of probability measure, and (c) follows from the intermediate result required to prove Claim 2.  Hence proved.\\

\textbf{Property 6:} If \( \{A_i, i \geq 1\} \) is a sequence of increasing nested events i.e., \( A_i \subseteq A_{i+1}, \forall i \geq 1 \), then
\[
P\left(\bigcup_{i=1}^{\infty} A_i\right) = \lim_{m \to \infty} P(A_m).
\]

\textbf{Property 7:} If \( \{A_i, i \geq 1\} \) is a sequence of decreasing nested events i.e., \( A_{i+1} \subseteq A_i, \forall i \geq 1 \), then
\[
P\left(\bigcap_{i=1}^{\infty} A_i\right) = \lim_{m \to \infty} P(A_m). 
\]

Properties 6 and 7 are said to be corollaries to Property 5.\\

\textbf{Property 8:} Suppose \( \{A_i, i \geq 1\} \) are events, then
\[
P\left(\bigcup_{i=1}^{\infty} A_i\right) \leq \sum_{i=1}^{\infty} P(A_i).
\]

This result is known as the Union Bound. This bound is trivial if \( \sum_{i=1}^{\infty} P(A_i) \geq 1 \) since the LHS of Property 8 is a probability of some event. This is a very widely used bound, and has several applications. For instance, the union bound is used in the probability of error analysis in Digital Communications for complicated modulation schemes. \\

\textit{Proof:} Define a new family of sets \( B_1 = A_1, B_2 = A_2 \setminus A_1, \ldots, B_n = A_n \setminus \bigcup_{i=1}^{n-1} A_i, \ldots \).  \\

\textbf{Claim 1:} \( B_i \cap B_j = \emptyset, \forall i \neq j \).  \\
\textbf{Claim 2:} \( \bigcup_{i=1}^{\infty} A_i = \bigcup_{i=1}^{\infty} B_i \). \\ 

Since \( \{B_i, i \geq 1\} \) is a disjoint sequence of events, and using the above claims, we get
\[
P\left(\bigcup_{i=1}^{\infty} A_i\right) = P\left(\bigcup_{i=1}^{\infty} B_i\right) = \sum_{i=1}^{\infty} P(B_i).
\]
Also, since \( B_i \subseteq A_i \, \forall i \geq 1 \), \( P(B_i) \leq P(A_i) \, \forall i \geq 1 \) (using Property 2). Therefore, the finite sum of probabilities follows
\[
\sum_{i=1}^{n} P(B_i) \leq \sum_{i=1}^{n} P(A_i).
\]
Eventually, in the limit, the following holds:
\[
\sum_{i=1}^{\infty} P(B_i) \leq \sum_{i=1}^{\infty} P(A_i).
\]
Finally, we arrive at the result,
\[
P\left(\bigcup_{i=1}^{\infty} A_i\right) \leq \sum_{i=1}^{\infty} P(A_i).
\]

\begin{exercise}
    A standard card deck (52 cards) is distributed to two persons: 26 cards to each person. All partitions are equally likely. Find the probability that the first person receives all four aces.
\end{exercise}

\begin{solution}
    To find the probability that the first person receives all four aces when a standard deck of 52 cards is distributed equally between two persons (each receiving 26 cards), we use a measure-theoretic approach.\\

    Let \((\Omega, \mathcal{F}, P)\) be the probability space where:\\
    - \(\Omega\) is the set of all ways to partition the deck into two hands of 26 cards each.\\
    - \(\mathcal{F}\) is the \(\sigma\)-algebra of subsets of \(\Omega\).\\
    - \(P\) is the uniform probability measure on \((\Omega, \mathcal{F})\).\\
    
    \textbf{Step 1: Total Number of Outcomes}\\
    
    The total number of ways to choose 26 cards out of 52 for the first person is given by:
    \[
    |\Omega| = \binom{52}{26}
    \]
    where \(\binom{52}{26}\) denotes the binomial coefficient representing the number of ways to choose 26 cards from 52.\\
    
    \textbf{Step 2: Number of Favorable Outcomes}\\
    
    Next, we find the number of ways the first person can receive all four aces. If the first person is to receive all four aces, we must choose the remaining 22 cards from the remaining 48 non-ace cards. The number of ways to do this is:
    \[
    |\Omega_{\text{favorable}}| = \binom{48}{22}
    \]
    where \(\binom{48}{22}\) denotes the binomial coefficient representing the number of ways to choose 22 cards from the 48 non-ace cards.\\
    
    \textbf{Step 3: Calculating the Probability}\\
    
    The probability that the first person receives all four aces is the ratio of the number of favorable outcomes to the total number of outcomes:
    \[
    P(\text{First person receives all four aces}) = \frac{|\Omega_{\text{favorable}}|}{|\Omega|}
    \]
    \[
    P(\text{First person receives all four aces}) = \frac{\binom{48}{22}}{\binom{52}{26}}
    \]
\end{solution}

\begin{exercise}
    Let $\{A_r\}_{r=1}^n$ be a finite collection of events in a probability space $(\Omega, \mathcal{F}, P)$. We aim to prove that:

\[
P\left(\bigcup_{1 \leq r \leq n} A_r\right) \leq \min_{1 \leq k \leq n} \left\{ \sum_{1 \leq r \leq n} P(A_r) - \sum_{\substack{r: r \neq k}} P(A_r \cap A_k) \right\}
\]
\end{exercise}

\begin{solution}
    Define $S = \bigcup_{r=1}^{n} A_r$. By the inclusion-exclusion principle for a finite union of events, we have:

    \[
    P(S) = \sum_{1 \leq r \leq n} P(A_r) - \sum_{1 \leq r < s \leq n} P(A_r \cap A_s) + \ldots + (-1)^{n+1} P\left(\bigcap_{1 \leq r \leq n} A_r\right).
    \]
    
    This expression accounts for all possible intersections of the events $A_r$. However, to prove the inequality, we'll make use of the following upper bound:\\
    
    Consider any fixed $k \in \{1, 2, \ldots, n\}$. We can express $P(S)$ as:
    
    \[
    P(S) \leq P(A_k) + P\left(\bigcup_{\substack{r: r \neq k}} A_r \setminus A_k\right).
    \]
    
    This follows since the probability of $S$ is at most the probability of $A_k$ plus the probability of the events outside of $A_k$ but not overlapping with it.\\
    
    Now, observe that:
    
    \[
    P\left(\bigcup_{\substack{r: r \neq k}} A_r \setminus A_k\right) \leq \sum_{\substack{r: r \neq k}} P(A_r \setminus A_k).
    \]
    
    Using the identity $P(A_r \setminus A_k) = P(A_r) - P(A_r \cap A_k)$, we can rewrite the above as:
    
    \[
    P\left(\bigcup_{\substack{r: r \neq k}} A_r \setminus A_k\right) \leq \sum_{\substack{r: r \neq k}} \left(P(A_r) - P(A_r \cap A_k)\right).
    \]
    
    Therefore:
    
    \[
    P(S) \leq P(A_k) + \sum_{\substack{r: r \neq k}} \left(P(A_r) - P(A_r \cap A_k)\right).
    \]
    
    Simplifying further:
    
    \[
    P(S) \leq \sum_{1 \leq r \leq n} P(A_r) - \sum_{\substack{r: r \neq k}} P(A_r \cap A_k).
    \]
    
    Since this inequality holds for any $k \in \{1, 2, \ldots, n\}$, we take the minimum over all $k$:
    
    \[
    P\left(\bigcup_{1 \leq r \leq n} A_r\right) \leq \min_{1 \leq k \leq n} \left\{ \sum_{1 \leq r \leq n} P(A_r) - \sum_{\substack{r: r \neq k}} P(A_r \cap A_k) \right\}.
    \]
\end{solution}

\begin{exercise}
    You are given that at least one of the events $A_n$, $1 \leq n \leq N$, is certain to occur. However, certainly no more than two occur. If $P(A_n) = p$ and $P(A_n \cap A_m) = q$, $m \neq n$, then show that $p \geq \frac{1}{N}$ and $q \leq \frac{2}{N}$. 
\end{exercise}

\begin{solution}

    Given the events $A_1, A_2, \ldots, A_N$ such that at least one event occurs and at most two occur, we have:
\[
P\left(\bigcup_{n=1}^{N} A_n\right) = 1
\]
and
\[
P\left(A_n \cap A_m\right) = q \quad \text{for} \quad m \neq n.
\]

By the principle of inclusion-exclusion, the probability of the union of these events is:
\[
P\left(\bigcup_{n=1}^{N} A_n\right) = \sum_{n=1}^{N} P(A_n) - \sum_{1 \leq n < m \leq N} P(A_n \cap A_m).
\]
Substituting the given values, we get:
\[
1 = \sum_{n=1}^{N} p - \sum_{1 \leq n < m \leq N} q.
\]
The number of terms in the first sum is $N$, so:
\[
1 = Np - \binom{N}{2}q,
\]
where $\binom{N}{2} = \frac{N(N-1)}{2}$ is the number of ways to choose 2 events from $N$.\\

Thus:
\[
1 = Np - \frac{N(N-1)}{2}q.
\]

Rearranging to solve for $p$:
\[
Np = 1 + \frac{N(N-1)}{2}q,
\]
\[
p = \frac{1}{N} + \frac{(N-1)}{2}q.
\]

Since at most two events can occur, $q$ must be small enough so that no three events can occur simultaneously. Hence, by substituting $p \geq \frac{1}{N}$:
\[
\frac{1}{N} + \frac{(N-1)}{2}q \geq \frac{1}{N},
\]
\[
\frac{(N-1)}{2}q \geq 0.
\]

To find the upper bound of $q$, note that since at most two events can occur, the total probability contributed by the intersections should not exceed $1$. Therefore:
\[
\frac{(N-1)}{2}q \leq \frac{1}{N}.
\]
\[
q \leq \frac{2}{N}.
\]
\end{solution}

\begin{exercise}
    Consider a measurable space $(\Omega, \mathcal{F})$ with $\Omega = [0, 1]$. A measure $P$ is defined on the non-empty subsets of $\Omega$ (in $\mathcal{F}$), which are all of the form $(a, b)$, $(a, b]$, $[a, b)$ and $[a, b]$, as the length of the interval, i.e.,
\[
P((a, b)) = P((a, b]) = P([a, b)) = P([a, b]) = b - a.
\]

\textbf{(a)} Show that $P$ is not just a measure, but it's a probability measure.\\
\textbf{(b)} Let $A_n = \left[ \frac{1}{n+1}, 1 \right]$ and $B_n = \left[ 0, \frac{1}{n+1} \right]$ for $n \geq 1$. Compute $P\left(\cup_{i \in \mathbb{N}} A_i\right)$, $P\left(\cap_{i \in \mathbb{N}} A_i\right)$, $P\left(\cup_{i \in \mathbb{N}} B_i\right)$, and $P\left(\cap_{i \in \mathbb{N}} B_i\right)$.\\
\textbf{(c)} Compute $P\left(\cap_{i \in \mathbb{N}} (B_i^c \cup A_i^c)\right)$.\\
\textbf{(d)} Let $C_m = \left[0, \frac{1}{m}\right]$ such that $P(C_m) = P(A_n)$. Express $m$ in terms of $n$.\\
\textbf{(e)} Evaluate $P\left(\cap_{i \in \mathbb{N}} (C_i \cap A_i)\right)$ and $P\left(\cup_{i \in \mathbb{N}} (C_i \cap A_i)\right)$.
\end{exercise}

\begin{solution}
    To show that $P$ is a probability measure, we need to verify two properties:\\

    1. Non-negativity: $P(A) \geq 0$ for all $A \in \mathcal{F}$. By definition, $P(A) = b - a \geq 0$ since $b \geq a$ for all intervals in $[0, 1]$.\\
    
    2. $P(\Omega) = 1$: The entire space $\Omega = [0, 1]$. Hence, $P([0, 1]) = 1 - 0 = 1$.\\
    
    Therefore, $P$ is a probability measure.\\
    
    $P\left(\cup_{i \in \mathbb{N}} A_i\right) = P([0, 1]) = 1$\\
    $P\left(\cap_{i \in \mathbb{N}} A_i\right) = P\left([0, 1]\right) = 1$\\
    $P\left(\cup_{i \in \mathbb{N}} B_i\right) = P([0, 1]) = 1$\\
    $P\left(\cap_{i \in \mathbb{N}} B_i\right) = P(\{0\}) = 0$\\
    
    Note that $B_i^c = [\frac{1}{n+1}, 1]$ and $A_i^c = [0, \frac{1}{n+1}]$. Thus:
    \[
    P\left(\cap_{i \in \mathbb{N}} (B_i^c \cup A_i^c)\right) = P(\emptyset) = 0
    \]
    
    We have $P(C_m) = \frac{1}{m}$ and $P(A_n) = 1 - \frac{1}{n+1}$. Equating these gives:
    \[
    \frac{1}{m} = 1 - \frac{1}{n+1}
    \]
    \[
    m = \frac{n+1}{n}
    \]
    
    
    $P\left(\cap_{i \in \mathbb{N}} (C_i \cap A_i)\right) = P(\emptyset) = 0$\\
    $P\left(\cup_{i \in \mathbb{N}} (C_i \cap A_i)\right) = P([0, 1]) = 1$\\
\end{solution}
\section{Discrete Probability Spaces}


Discrete probability spaces correspond to the case when the sample space $\Omega$ is countable. This is the most conceptually straightforward case, since it is possible to assign probabilities to all subsets of $\Omega$.

\begin{definition}
    A probability space $(\Omega, \mathcal{F}, P)$ is said to be a discrete probability space if the following conditions hold:
\begin{itemize}
    \item[(a)] The sample space $\Omega$ is finite or countably infinite,
    \item[(b)] The $\sigma$-algebra is the set of all subsets of $\Omega$, i.e., $\mathcal{F} = 2^\Omega$, and
    \item[(c)] The probability measure, $P$, is defined for every subset of $\Omega$. In particular, it can be defined in terms of the probabilities $P(\{\omega\})$ of the singletons corresponding to each of the elementary outcomes $\omega$, and satisfies for every $A \in \mathcal{F}$,
\[
P(A) = \sum_{\omega \in A} P(\{\omega\}),
\]
and
\[
\sum_{\omega \in \Omega} P(\{\omega\}) = 1.
\]
\end{itemize}
\end{definition}

The above definition highlights that it is possible to assign probabilities to each singleton set of $\Omega$, but it doesn't say about \textit{what probabilities to assign?} This depends on our use-case and what we want to model. 

\subsubsection{Examples of Discrete Probability Spaces}

\textbf{1.} Let us consider a coin toss experiment with the probability of getting a head as \( p \) and the probability of getting a tail as \( (1 - p) \). The sample space and the \(\sigma\)-algebra are defined as follows:
\[
\Omega = \{H, T\} \equiv \{0, 1\}, \quad \mathcal{F} = 2^{\Omega} = \{\emptyset, \{H\}, \{T\}, \Omega\}.
\]
The probability measure is given by:
\[
P(\{H\}) \equiv P(\{0\}) = p, \quad P(\{T\}) \equiv P(\{1\}) = 1 - p.
\]
In this case, we say that \( P(.) \) is a Bernoulli measure on \( \{\{0, 1\}, \, 2^{\{0, 1\}}\}\).\\

\textbf{2.} Let \( \Omega = \mathbb{N} \) and \( \mathcal{F} = 2^{\mathbb{N}} \). We can define the probability of a singleton as:
\[
P(\{k\}) = a_k \geq 0, \quad k \in \mathbb{N},
\]
under the constraint that:
\[
\sum_{k \in \mathbb{N}} P(\{k\}) = 1.
\]
For example, if we let \( a_k = \frac{1}{2^k}, \, k \in \mathbb{N} \), this is a valid measure, since:
\[
\sum_{k \in \mathbb{N}} \frac{1}{2^k} = 1.
\]
As another example, consider \( a_k = (1 - p)^{k-1} p \) for \( 0 < p < 1 \) and \( k \in \mathbb{N} \). This is known as a geometric measure with parameter \( p \). It is a valid probability measure since:
\[
\sum_{k \in \mathbb{N}} (1 - p)^{k-1} p = 1.
\]

\textbf{3.} Let \( \Omega = \mathbb{N} \cup \{0\} \) and \( \mathcal{F} = 2^{\Omega} \). We define the probability measure as:
\[
P(\{k\}) = \frac{e^{-\lambda} \lambda^k}{k!}, \quad \lambda > 0.
\]
This probability measure is called a Poisson measure with parameter \( \lambda \) on \( \{\Omega, \, 2^{\Omega}\}  \). This is a valid probability measure, since:
\[
\sum_{k=0}^{\infty} P(\{k\}) = \sum_{k=0}^{\infty} \frac{e^{-\lambda} \lambda^k}{k!} = e^{-\lambda} \sum_{k=0}^{\infty} \frac{\lambda^k}{k!} = e^{-\lambda} e^{\lambda} = 1.
\]

\textbf{4.} Let \( \Omega = \{0, 1, 2, \ldots, N\} \), where \( N \in \mathbb{N} \) and \( \mathcal{F} = 2^{\Omega} \). We define the probability measure as:
\[
P(\{k\}) = \binom{N}{k} p^k (1 - p)^{N-k}, \quad 0 < p < 1.
\]
This probability measure is called a Binomial measure with parameters \( (N, p) \) on \( \{\Omega, \, 2^{\Omega}\} \). This can be verified to be a valid probability measure as follows:
\[
\sum_{k \in \Omega} \binom{N}{k} p^k (1 - p)^{N-k} = (p + (1 - p))^N = 1.
\]

Note that in all the examples above, we have not explicitly specified an expression for \( P(A) \) for every \( A \subset \Omega \). Since the sample space is countable, the probability of any subset of the sample space can be obtained as the sum of probabilities of the corresponding elementary outcomes. In other words, for discrete probability spaces, it suffices to specify the probabilities of singletons corresponding to each of the elementary outcomes.


\vspace{30pt}
\hrule