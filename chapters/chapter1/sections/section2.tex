\section{Measurable Space}


Consider a large canvas $\Omega$, which represents an entire art studio wall. The goal is to measure the total amount of paint used on specific regions of the wall. However, you don't want to measure paint usage for every possible shape or region on the wall (which could be infinitely complex). Instead, you decide to focus only on certain, manageable regions such as: rectangles, circles, simple polygons, and unions and intersections of these shapes.\\

These regions form a collection, say $\mathcal{F}$, which represents all the shapes and combinations that you are interested in measuring. The pair $(\Omega, \mathcal{F})$ then becomes a \textit{measurable space}, where:
\begin{itemize}
    \item $\Omega$ is the entire canvas, representing all possible points on the wall.
    \item $\mathcal{F}$ is a collection of specific shapes (rectangles, circles, etc.) and their combinations, which are the regions you can measure the amount of paint for.
\end{itemize}

The measurable sets in $\mathcal{F}$ are those specific shapes and combinations that you have chosen to focus on, similar to how measurable sets in probability theory are those events that belong to a specific $\sigma$-algebra. In context of probability theory, imagine you have a sample space $\Omega$, which represents all the possible outcomes of an experiment. For instance, if you flip a coin, the sample space is $\Omega = \{\text{Heads}, \text{Tails}\}$.\\

A measurable space is a pair $(\Omega, \mathcal{F})$, where:
\begin{itemize}
    \item $\Omega$ is the sample space, representing all possible outcomes.
    \item $\mathcal{F}$ is a $\sigma$-algebra of subsets of $\Omega$. It is a collection of subsets that includes the empty set, is closed under complements, and closed under countable unions.
\end{itemize}

The subsets in $\mathcal{F}$ are the ones we can \textit{measure}, hence the term \textit{measurable space}.

\begin{definition}
    The 2-tuple $(\Omega, \mathcal{F})$ is called a \textbf{measurable space}. Here:
    \begin{itemize}
        \item $\Omega$ is the sample space, a non-empty set.
        \item $\mathcal{F}$ is a $\sigma$-algebra on $\Omega$, meaning it satisfies:
        \begin{enumerate}
            \item $\emptyset \in \mathcal{F}$.
            \item If $A \in \mathcal{F}$, then $A^c \in \mathcal{F}$ (closure under complements).
            \item If $A_1, A_2, A_3, \ldots \in \mathcal{F}$, then $\bigcup\limits_{i=1}^{\infty} A_i \in \mathcal{F}$ (closure under countable unions).
        \end{enumerate}
    \end{itemize}
\end{definition}

\begin{definition}
    Every member of the $\sigma$-algebra $\mathcal{F}$ is called an \textbf{$\mathcal{F}$-measurable set} in the context of measure theory. These are the subsets of $\Omega$ that we can measure using the $\sigma$-algebra $\mathcal{F}$.
\end{definition}

\begin{definition}
    $\mathcal{F}$-measurable sets are called \textbf{events}. This means that an event is not just any subset of $\Omega$, but one that belongs to the $\sigma$-algebra $\mathcal{F}$ under consideration.
\end{definition}

\subsubsection{Examples Where Subsets of $\Omega$ Are Not $\mathcal{F}$-Measurable Sets:}

In measure theory and probability, not all subsets of a sample space $\Omega$ are necessarily $\mathcal{F}$-measurable sets. This depends on the construction of the $\sigma$-algebra $\mathcal{F}$ associated with $\Omega$. Below are some examples where subsets of $\Omega$ are not $\mathcal{F}$-measurable: \\

\textbf{Example 1: A Countable Union Not in an Algebra} \\

Let $\Omega = \mathbb{N}$, the set of natural numbers, and let $\mathcal{F}$ be an \textbf{algebra} consisting of all finite subsets of $\mathbb{N}$ and their complements (which are cofinite sets). In this setup, $\mathcal{F}$ contains only finite unions and intersections.\\

Now, consider the subset $A = \{2, 4, 6, \ldots\}$, the set of all even numbers. This is an \textbf{infinite} set but not cofinite (its complement, the set of odd numbers, is also infinite). Since $\mathcal{F}$ only contains finite or cofinite sets, $A$ is not in $\mathcal{F}$. Thus, $A$ is an example of a subset of $\Omega$ that is not $\mathcal{F}$-measurable.\\

\textbf{Example 2: Subsets in the Cantor Set}\\

Let $\Omega$ be the \textbf{Cantor set}, which is a subset of the interval $[0, 1]$. Construct $\mathcal{F}$ to be the $\sigma$-algebra generated by all \textbf{intervals} in $[0, 1]$. While $\mathcal{F}$ will contain many subsets, it will not include certain highly irregular subsets of the Cantor set that are not expressible as a countable union, intersection, or complement of intervals.\\

For instance, a subset of the Cantor set that is formed using a complex pattern based on the binary expansion of its elements may not be measurable in this $\sigma$-algebra. Hence, such a subset would not be $\mathcal{F}$-measurable.\\

\textbf{Example 3: Students in a Class}\\

Imagine a class of 30 students, represented by the set:

\[
\Omega = \{s_1, s_2, s_3, \ldots, s_{30}\}
\]

Define a \( \sigma \)-algebra \( \mathcal{F} \) that consists only of subsets containing an even number of students. This \( \sigma \)-algebra could include sets like:

\[
\mathcal{F} = \{\emptyset, \{s_1, s_2\}, \{s_3, s_4\}, \{s_5, s_6\}, \ldots, \{s_1, s_2, s_3, s_4\}, \ldots, \{s_1, s_2, \ldots, s_{30}\}\}
\]

Now, consider the subset of interest \( A = \{s_1, s_3, s_5, s_7, \ldots, s_{29}\} \), which contains all the odd-numbered students in the class.\\

In the \( \sigma \)-algebra \( \mathcal{F} \), every set is constructed to contain only an even number of students. The set \( A \), which contains an odd number of students, cannot be expressed as a union or intersection of sets from \( \mathcal{F} \).\\

Thus, \( A \) is not \( \mathcal{F} \)-measurable because it does not fit within the constraints of our \( \sigma \)-algebra.\\

\textbf{Example 4: Days of Week}\\

Let \( \Omega = \{\text{Monday, Tuesday, Wednesday, Thursday, Friday, Saturday, Sunday}\} \). \\

Suppose we define a \( \sigma \)-algebra \( \mathcal{F} \) that only includes subsets that contain weekdays:
\[
\mathcal{F} = \{\emptyset, \{\text{Monday, Tuesday, Wednesday, Thursday, Friday}\}, \{\text{Saturday, Sunday}\}, \Omega\}
\]

Now, consider the subset of interest \( A = \{\text{Saturday}\} \).\\

The \( \sigma \)-algebra \( \mathcal{F} \) only contains the sets of weekdays and their complements but does not include individual weekend days like Saturday. Thus, \( A \) cannot be constructed as a union or intersection of sets in \( \mathcal{F} \).\\

Since \( A \) cannot be represented within the existing \( \sigma \)-algebra \( \mathcal{F} \), it is not \( \mathcal{F} \)-measurable.

\subsection{Measure}

\begin{definition}
    Let \((\Omega, \mathcal{F})\) be a measurable space. A \textit{measure} on \((\Omega, \mathcal{F})\) is a function \(\mu: \mathcal{F} \to [0, \infty]\) such that:

\begin{enumerate}
    \item \(\mu(\emptyset) = 0.\)
    \item If \(\{A_i, i \geq 1\}\) is a sequence of disjoint sets in \(\mathcal{F}\), then the measure of the union of these countably infinite disjoint sets is equal to the sum of the measures of the individual sets:
    \[
    \mu\left(\bigcup_{i=1}^{\infty} A_i\right) = \sum_{i=1}^{\infty} \mu(A_i).
    \]
\end{enumerate}
\end{definition}

The second property stated above is known as the \textit{countable additivity property} of measures. From the definition, it is clear that a measure can only be assigned to elements of \(\mathcal{F}\).

\begin{definition}
    The triplet \((\Omega, \mathcal{F}, \mu)\) is called a \textit{measure space}. 
\end{definition}

The measure \(\mu\) is said to be a \textit{finite measure} if \(\mu(\Omega) < \infty\); otherwise, \(\mu\) is said to be an \textit{infinite measure}. In particular, if \(\mu(\Omega) = 1\), then \(\mu\) is referred to as a \textit{probability measure}.


\subsection{Probability Measure}

\begin{definition}
    A \textit{probability measure} is a function \( P: \mathcal{F} \to [0, 1] \) such that:

\begin{enumerate}
    \item \( P(\emptyset) = 0 \).
    \item \( P(\Omega) = 1 \).
    \item \textbf{Countable Additivity:} If \( \{A_i, i \geq 1\} \) is a sequence of disjoint sets in \( \mathcal{F} \), then
    \[
    P\left(\bigcup_{i=1}^{\infty} A_i\right) = \sum_{i=1}^{\infty} P(A_i).
    \]
\end{enumerate}
\end{definition}

\begin{definition}
    The triplet \( (\Omega, \mathcal{F}, P) \) is called a \textit{probability space}, and the three properties stated above are referred to as the \textit{axioms of probability}.
\end{definition}

It is clear from the definition that probabilities are defined only for elements of \( \mathcal{F} \), and not necessarily for all subsets of \( \Omega \). In other words, probability measures are assigned only to \textit{events}. Even when we speak of the probability of an elementary outcome \( \omega \), it should be interpreted as the probability assigned to the singleton set \( \{\omega\} \) (assuming, of course, that the singleton is an event).

\subsection{Properties of Probability Measure}

We will derive some fundamental properties of probability measures, which follow directly from the axioms of probability. In what follows, \((\Omega, \mathcal{F}, P)\) is a probability space. \\

\textbf{Property 1:} Suppose \( A \) be a subset of \(\Omega\) such that \( A \in \mathcal{F} \). Then,
\[
P(A^c) = 1 - P(A). 
\]
\textit{Proof:} Given any subset \( A \subseteq \Omega \), \( A \) and \( A^c \) partition the sample space. Hence, \( A^c \cup A = \Omega \) and \( A^c \cap A = \emptyset \). By the "Countable Additivity" axiom of probability, \( P(A^c \cup A) = P(A) + P(A^c) \). Therefore, \( P(\Omega) = P(A) + P(A^c) \implies P(A^c) = 1 - P(A) \).\\

\textbf{Property 2:} Consider events \( A \) and \( B \) such that \( A \subseteq B \) and \( A, B \in \mathcal{F} \). Then \( P(A) \leq P(B) \).\\

\textit{Proof:} The set \( B \) can be written as the union of two disjoint sets \( A \) and \( A^c \cap B \). Therefore, we have \( P(A) + P(A^c \cap B) = P(B) \implies P(A) \leq P(B) \) since \( P(A^c \cap B) \geq 0 \).\\

\textbf{Property 3:} (Finite Additivity) If \( A_1, A_2, \ldots, A_n \) are a finite number of disjoint events, then
\[
P\left(\bigcup_{i=1}^{n} A_i\right) = \sum_{i=1}^{n} P(A_i). 
\]
\textit{Proof:} This property follows directly from the axiom of countable additivity of probability measures. It is obtained by setting the events \( A_{n+1}, A_{n+2}, \ldots \) as empty sets. The left-hand side (LHS) will simplify as:
\[
P\left(\bigcup_{i=1}^{\infty} A_i\right) = P\left(\bigcup_{i=1}^{n} A_i\right).
\]
The right-hand side (RHS) can be manipulated as follows:
\[
\sum_{i=1}^{\infty} P(A_i) \overset{(a)}{=} \lim_{k \to \infty} \sum_{i=1}^{k} P(A_i) = \sum_{i=1}^{n} P(A_i) + \lim_{k \to \infty} \sum_{i=n+1}^{k} P(A_i) \overset{(b)}{=} \sum_{i=1}^{n} P(A_i) + \lim_{k \to \infty} 0 = \sum_{i=1}^{n} P(A_i).
\]
where (a) follows from the definition of an infinite series and (b) is a consequence of setting the events from \( A_{n+1} \) onwards to null sets.\\

\textbf{Property 4:} For any \( A, B \in F \),
\[
P(A \cup B) = P(A) + P(B) - P(A \cap B). 
\]
In general, for a family of events \( \{A_i\}_{i=1}^{n} \subset F \),
\[
P\left(\bigcup_{i=1}^{n} A_i\right) = \sum_{i=1}^{n} P(A_i) - \sum_{i < j} P(A_i \cap A_j) + \sum_{i < j < k} P(A_i \cap A_j \cap A_k) + \ldots + (-1)^{n+1} P\left(\bigcap_{i=1}^{n} A_i\right). 
\]
This property is proved using induction on \( n \). The property can be proved in a much simpler way using the concept of Indicator Random Variables, which will be discussed in the subsequent lectures.\\

\textit{Proof} The set \( A \cup B \) can be written as \( A \cup B = A \cup (A^c \cap B) \). Since \( A \) and \( A^c \cap B \) are disjoint events, \( P(A \cup B) = P(A) + P(A^c \cap B) \). Now, set \( B \) can be partitioned as \( B = (A \cap B) \cup (A^c \cap B) \). Hence, \( P(B) = P(A \cap B) + P(A^c \cap B) \). On substituting this result in the expression of \( P(A \cup B) \), we will obtain the final result that \( P(A \cup B) = P(A) + P(B) - P(A \cap B) \).\\

\textbf{Property 5:} If \( \{A_i, i \geq 1\} \) are events, then
\[
P\left(\bigcup_{i=1}^{\infty} A_i\right) = \lim_{m \to \infty} P\left(\bigcup_{i=1}^{m} A_i\right). 
\]
This result is known as the continuity of probability measures.\\

\textit{How to visualise this property?} Imagine $P_m$ is the probability of union of $A_1, A_2, ..., A_m$. Then the sequence of $P_m$'s is a monotonically increasing sequence. Also, the sequence is bounded by the interval $[0, 1]$. We know, that every monotonically increasing sequence that is bounded must converge. So the RHS of the property 5 is a finite quantity. So is the LHS because the countable union of sets is a well-definied set for which the probability measure is defined. The property says both are equal. \\

\textit{Proof:} Define a new family of sets \( B_1 = A_1, B_2 = A_2 \setminus A_1, \ldots, B_n = A_n \setminus \bigcup_{i=1}^{n-1} A_i, \ldots \).  \\

Then, the following claims are placed: \\
\textbf{Claim 1:} \( B_i \cap B_j = \emptyset, \forall i \neq j \).  \\
\textbf{Claim 2:} \( \bigcup_{i=1}^{\infty} A_i = \bigcup_{i=1}^{\infty} B_i \).  \\

Since \( \{B_i, i \geq 1\} \) is a disjoint sequence of events, and using the above claims, we get
\[
P\left(\bigcup_{i=1}^{\infty} A_i\right) = P\left(\bigcup_{i=1}^{\infty} B_i\right) = \sum_{i=1}^{\infty} P(B_i).
\]

Therefore,
\[
P\left(\bigcup_{i=1}^{\infty} A_i\right) = \sum_{i=1}^{\infty} P(B_i) \quad (a) = \lim_{m \to \infty} \sum_{i=1}^{m} P(B_i) \quad (b) = \lim_{m \to \infty} P\left(\bigcup_{i=1}^{m} B_i\right) \quad 
\]
\[
    (c) = \lim_{m \to \infty} P\left(\bigcup_{i=1}^{m} A_i\right).
\]

Here, (a) follows from the definition of an infinite series, (b) follows from Claim 1 in conjunction with the Countable Additivity axiom of probability measure, and (c) follows from the intermediate result required to prove Claim 2.  Hence proved.\\

\textbf{Property 6:} If \( \{A_i, i \geq 1\} \) is a sequence of increasing nested events i.e., \( A_i \subseteq A_{i+1}, \forall i \geq 1 \), then
\[
P\left(\bigcup_{i=1}^{\infty} A_i\right) = \lim_{m \to \infty} P(A_m).
\]

\textbf{Property 7:} If \( \{A_i, i \geq 1\} \) is a sequence of decreasing nested events i.e., \( A_{i+1} \subseteq A_i, \forall i \geq 1 \), then
\[
P\left(\bigcap_{i=1}^{\infty} A_i\right) = \lim_{m \to \infty} P(A_m). 
\]

Properties 6 and 7 are said to be corollaries to Property 5.\\

\textbf{Property 8:} Suppose \( \{A_i, i \geq 1\} \) are events, then
\[
P\left(\bigcup_{i=1}^{\infty} A_i\right) \leq \sum_{i=1}^{\infty} P(A_i).
\]

This result is known as the Union Bound. This bound is trivial if \( \sum_{i=1}^{\infty} P(A_i) \geq 1 \) since the LHS of Property 8 is a probability of some event. This is a very widely used bound, and has several applications. For instance, the union bound is used in the probability of error analysis in Digital Communications for complicated modulation schemes. \\

\textit{Proof:} Define a new family of sets \( B_1 = A_1, B_2 = A_2 \setminus A_1, \ldots, B_n = A_n \setminus \bigcup_{i=1}^{n-1} A_i, \ldots \).  \\

\textbf{Claim 1:} \( B_i \cap B_j = \emptyset, \forall i \neq j \).  \\
\textbf{Claim 2:} \( \bigcup_{i=1}^{\infty} A_i = \bigcup_{i=1}^{\infty} B_i \). \\ 

Since \( \{B_i, i \geq 1\} \) is a disjoint sequence of events, and using the above claims, we get
\[
P\left(\bigcup_{i=1}^{\infty} A_i\right) = P\left(\bigcup_{i=1}^{\infty} B_i\right) = \sum_{i=1}^{\infty} P(B_i).
\]
Also, since \( B_i \subseteq A_i \, \forall i \geq 1 \), \( P(B_i) \leq P(A_i) \, \forall i \geq 1 \) (using Property 2). Therefore, the finite sum of probabilities follows
\[
\sum_{i=1}^{n} P(B_i) \leq \sum_{i=1}^{n} P(A_i).
\]
Eventually, in the limit, the following holds:
\[
\sum_{i=1}^{\infty} P(B_i) \leq \sum_{i=1}^{\infty} P(A_i).
\]
Finally, we arrive at the result,
\[
P\left(\bigcup_{i=1}^{\infty} A_i\right) \leq \sum_{i=1}^{\infty} P(A_i).
\]

\begin{exercise}
    A standard card deck (52 cards) is distributed to two persons: 26 cards to each person. All partitions are equally likely. Find the probability that the first person receives all four aces.
\end{exercise}

\begin{solution}
    To find the probability that the first person receives all four aces when a standard deck of 52 cards is distributed equally between two persons (each receiving 26 cards), we use a measure-theoretic approach.\\

    Let \((\Omega, \mathcal{F}, P)\) be the probability space where:\\
    - \(\Omega\) is the set of all ways to partition the deck into two hands of 26 cards each.\\
    - \(\mathcal{F}\) is the \(\sigma\)-algebra of subsets of \(\Omega\).\\
    - \(P\) is the uniform probability measure on \((\Omega, \mathcal{F})\).\\
    
    \textbf{Step 1: Total Number of Outcomes}\\
    
    The total number of ways to choose 26 cards out of 52 for the first person is given by:
    \[
    |\Omega| = \binom{52}{26}
    \]
    where \(\binom{52}{26}\) denotes the binomial coefficient representing the number of ways to choose 26 cards from 52.\\
    
    \textbf{Step 2: Number of Favorable Outcomes}\\
    
    Next, we find the number of ways the first person can receive all four aces. If the first person is to receive all four aces, we must choose the remaining 22 cards from the remaining 48 non-ace cards. The number of ways to do this is:
    \[
    |\Omega_{\text{favorable}}| = \binom{48}{22}
    \]
    where \(\binom{48}{22}\) denotes the binomial coefficient representing the number of ways to choose 22 cards from the 48 non-ace cards.\\
    
    \textbf{Step 3: Calculating the Probability}\\
    
    The probability that the first person receives all four aces is the ratio of the number of favorable outcomes to the total number of outcomes:
    \[
    P(\text{First person receives all four aces}) = \frac{|\Omega_{\text{favorable}}|}{|\Omega|}
    \]
    \[
    P(\text{First person receives all four aces}) = \frac{\binom{48}{22}}{\binom{52}{26}}
    \]
\end{solution}

\begin{exercise}
    Let $\{A_r\}_{r=1}^n$ be a finite collection of events in a probability space $(\Omega, \mathcal{F}, P)$. We aim to prove that:

\[
P\left(\bigcup_{1 \leq r \leq n} A_r\right) \leq \min_{1 \leq k \leq n} \left\{ \sum_{1 \leq r \leq n} P(A_r) - \sum_{\substack{r: r \neq k}} P(A_r \cap A_k) \right\}
\]
\end{exercise}

\begin{solution}
    Define $S = \bigcup_{r=1}^{n} A_r$. By the inclusion-exclusion principle for a finite union of events, we have:

    \[
    P(S) = \sum_{1 \leq r \leq n} P(A_r) - \sum_{1 \leq r < s \leq n} P(A_r \cap A_s) + \ldots + (-1)^{n+1} P\left(\bigcap_{1 \leq r \leq n} A_r\right).
    \]
    
    This expression accounts for all possible intersections of the events $A_r$. However, to prove the inequality, we'll make use of the following upper bound:\\
    
    Consider any fixed $k \in \{1, 2, \ldots, n\}$. We can express $P(S)$ as:
    
    \[
    P(S) \leq P(A_k) + P\left(\bigcup_{\substack{r: r \neq k}} A_r \setminus A_k\right).
    \]
    
    This follows since the probability of $S$ is at most the probability of $A_k$ plus the probability of the events outside of $A_k$ but not overlapping with it.\\
    
    Now, observe that:
    
    \[
    P\left(\bigcup_{\substack{r: r \neq k}} A_r \setminus A_k\right) \leq \sum_{\substack{r: r \neq k}} P(A_r \setminus A_k).
    \]
    
    Using the identity $P(A_r \setminus A_k) = P(A_r) - P(A_r \cap A_k)$, we can rewrite the above as:
    
    \[
    P\left(\bigcup_{\substack{r: r \neq k}} A_r \setminus A_k\right) \leq \sum_{\substack{r: r \neq k}} \left(P(A_r) - P(A_r \cap A_k)\right).
    \]
    
    Therefore:
    
    \[
    P(S) \leq P(A_k) + \sum_{\substack{r: r \neq k}} \left(P(A_r) - P(A_r \cap A_k)\right).
    \]
    
    Simplifying further:
    
    \[
    P(S) \leq \sum_{1 \leq r \leq n} P(A_r) - \sum_{\substack{r: r \neq k}} P(A_r \cap A_k).
    \]
    
    Since this inequality holds for any $k \in \{1, 2, \ldots, n\}$, we take the minimum over all $k$:
    
    \[
    P\left(\bigcup_{1 \leq r \leq n} A_r\right) \leq \min_{1 \leq k \leq n} \left\{ \sum_{1 \leq r \leq n} P(A_r) - \sum_{\substack{r: r \neq k}} P(A_r \cap A_k) \right\}.
    \]
\end{solution}

\begin{exercise}
    You are given that at least one of the events $A_n$, $1 \leq n \leq N$, is certain to occur. However, certainly no more than two occur. If $P(A_n) = p$ and $P(A_n \cap A_m) = q$, $m \neq n$, then show that $p \geq \frac{1}{N}$ and $q \leq \frac{2}{N}$. 
\end{exercise}

\begin{solution}

    Given the events $A_1, A_2, \ldots, A_N$ such that at least one event occurs and at most two occur, we have:
\[
P\left(\bigcup_{n=1}^{N} A_n\right) = 1
\]
and
\[
P\left(A_n \cap A_m\right) = q \quad \text{for} \quad m \neq n.
\]

By the principle of inclusion-exclusion, the probability of the union of these events is:
\[
P\left(\bigcup_{n=1}^{N} A_n\right) = \sum_{n=1}^{N} P(A_n) - \sum_{1 \leq n < m \leq N} P(A_n \cap A_m).
\]
Substituting the given values, we get:
\[
1 = \sum_{n=1}^{N} p - \sum_{1 \leq n < m \leq N} q.
\]
The number of terms in the first sum is $N$, so:
\[
1 = Np - \binom{N}{2}q,
\]
where $\binom{N}{2} = \frac{N(N-1)}{2}$ is the number of ways to choose 2 events from $N$.\\

Thus:
\[
1 = Np - \frac{N(N-1)}{2}q.
\]

Rearranging to solve for $p$:
\[
Np = 1 + \frac{N(N-1)}{2}q,
\]
\[
p = \frac{1}{N} + \frac{(N-1)}{2}q.
\]

Since at most two events can occur, $q$ must be small enough so that no three events can occur simultaneously. Hence, by substituting $p \geq \frac{1}{N}$:
\[
\frac{1}{N} + \frac{(N-1)}{2}q \geq \frac{1}{N},
\]
\[
\frac{(N-1)}{2}q \geq 0.
\]

To find the upper bound of $q$, note that since at most two events can occur, the total probability contributed by the intersections should not exceed $1$. Therefore:
\[
\frac{(N-1)}{2}q \leq \frac{1}{N}.
\]
\[
q \leq \frac{2}{N}.
\]
\end{solution}

\begin{exercise}
    Consider a measurable space $(\Omega, \mathcal{F})$ with $\Omega = [0, 1]$. A measure $P$ is defined on the non-empty subsets of $\Omega$ (in $\mathcal{F}$), which are all of the form $(a, b)$, $(a, b]$, $[a, b)$ and $[a, b]$, as the length of the interval, i.e.,
\[
P((a, b)) = P((a, b]) = P([a, b)) = P([a, b]) = b - a.
\]

\textbf{(a)} Show that $P$ is not just a measure, but it's a probability measure.\\
\textbf{(b)} Let $A_n = \left[ \frac{1}{n+1}, 1 \right]$ and $B_n = \left[ 0, \frac{1}{n+1} \right]$ for $n \geq 1$. Compute $P\left(\cup_{i \in \mathbb{N}} A_i\right)$, $P\left(\cap_{i \in \mathbb{N}} A_i\right)$, $P\left(\cup_{i \in \mathbb{N}} B_i\right)$, and $P\left(\cap_{i \in \mathbb{N}} B_i\right)$.\\
\textbf{(c)} Compute $P\left(\cap_{i \in \mathbb{N}} (B_i^c \cup A_i^c)\right)$.\\
\textbf{(d)} Let $C_m = \left[0, \frac{1}{m}\right]$ such that $P(C_m) = P(A_n)$. Express $m$ in terms of $n$.\\
\textbf{(e)} Evaluate $P\left(\cap_{i \in \mathbb{N}} (C_i \cap A_i)\right)$ and $P\left(\cup_{i \in \mathbb{N}} (C_i \cap A_i)\right)$.
\end{exercise}

\begin{solution}
    To show that $P$ is a probability measure, we need to verify two properties:\\

    1. Non-negativity: $P(A) \geq 0$ for all $A \in \mathcal{F}$. By definition, $P(A) = b - a \geq 0$ since $b \geq a$ for all intervals in $[0, 1]$.\\
    
    2. $P(\Omega) = 1$: The entire space $\Omega = [0, 1]$. Hence, $P([0, 1]) = 1 - 0 = 1$.\\
    
    Therefore, $P$ is a probability measure.\\
    
    $P\left(\cup_{i \in \mathbb{N}} A_i\right) = P([0, 1]) = 1$\\
    $P\left(\cap_{i \in \mathbb{N}} A_i\right) = P\left([0, 1]\right) = 1$\\
    $P\left(\cup_{i \in \mathbb{N}} B_i\right) = P([0, 1]) = 1$\\
    $P\left(\cap_{i \in \mathbb{N}} B_i\right) = P(\{0\}) = 0$\\
    
    Note that $B_i^c = [\frac{1}{n+1}, 1]$ and $A_i^c = [0, \frac{1}{n+1}]$. Thus:
    \[
    P\left(\cap_{i \in \mathbb{N}} (B_i^c \cup A_i^c)\right) = P(\emptyset) = 0
    \]
    
    We have $P(C_m) = \frac{1}{m}$ and $P(A_n) = 1 - \frac{1}{n+1}$. Equating these gives:
    \[
    \frac{1}{m} = 1 - \frac{1}{n+1}
    \]
    \[
    m = \frac{n+1}{n}
    \]
    
    
    $P\left(\cap_{i \in \mathbb{N}} (C_i \cap A_i)\right) = P(\emptyset) = 0$\\
    $P\left(\cup_{i \in \mathbb{N}} (C_i \cap A_i)\right) = P([0, 1]) = 1$\\
\end{solution}