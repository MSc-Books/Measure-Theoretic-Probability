\chapter{Transformation of Random Variables}

Suppose we observe a random variable, or a collection of random variables. In many practical scenarios, we might be more interested in some function of these observed random variables rather than the variables themselves. For instance, in communication systems, engineers often find it more useful to analyze the logarithm of noise power instead of the actual realization of noise. This is where the transformation of random variables becomes significant.\\

Let \(X\) be a random variable defined on the probability space \((\Omega, \mathcal{F}, P)\) and let \(f : \mathbb{R} \to \mathbb{R}\) be a function. We want to understand the properties of \(f(X)\). Since the random variable \(X\) can be viewed as a function, \(f(X)\) represents a composed function that maps \(\Omega\) to \(\mathbb{R}\). Our first question is whether \(f(X)\) qualifies as a valid random variable.\\

To explore this, we can consider the composed function \(f \circ X(\cdot)\). If \(f\) is an arbitrary function, then \(f(X)\) may not necessarily be a random variable. However, if \(f : \mathbb{R} \to \mathbb{R}\) is a Borel-measurable function—meaning that the pre-images of Borel sets under \(f\) are also Borel sets—then it follows that the pre-images of Borel sets under the composed function \(f \circ X(\cdot)\) are events in \(\mathcal{F}\). This leads us to conclude that \(f(X)\) is indeed a random variable. \\

The same reasoning applies to functions of several random variables. If we have a Borel-measurable function \(f : \mathbb{R}^n \to \mathbb{R}\) and random variables \(X_1, X_2, \ldots, X_n\), we can similarly argue that \(f(X_1, \ldots, X_n)\) is also a random variable. This relationship emphasizes the importance of Borel-measurability in ensuring that the transformation of random variables yields valid random variables.\\

Now that we have established the conditions under which a function of a random variable qualifies as a random variable, we seek to understand the probability law of \( f(X) \) based on the probability law \( P_X \) of \( X \). In simpler terms, given the cumulative distribution function (CDF) of \( X \), we aim to determine the CDF of \( f(X) \). To approach this, we will first examine some basic functions, such as the maximum, minimum, and summation, before delving into more complex transformations.

\section{Maximum and Minimum of Many Random Variables}

Consider a set of random variables \( X_1, X_2, X_3, \ldots, X_n \) defined on the probability space \( (\Omega, \mathcal{F}, P) \) with a joint CDF \( F_{X_1, X_2, \ldots, X_n} \). We define:

\[
Y_n = \min(X_1, X_2, X_3, \ldots, X_n)
\]

and 

\[
Z_n = \max(X_1, X_2, X_3, \ldots, X_n).
\]

Our goal is to find the CDF of both \( Y_n \) and \( Z_n \). \\

To begin with, let us verify that \( Z_n \) is indeed a random variable. The event \( \{ Z_n \leq x \} \) is equivalent to stating that each of the random variables \( X_1, X_2, X_3, \ldots, X_n \) is less than or equal to \( x \). We can express this as:

\[
\{ Z_n \leq x \} = \{ X_1 \leq x, X_2 \leq x, \ldots, X_n \leq x \}.
\]

Now, to confirm that \( \{ \omega : Z(\omega) \leq z \} \) is an event, we observe that:

\[
\{ \omega : Z(\omega) \leq z \} = \bigcap_{i=1}^n \{ \omega : X_i(\omega) \leq z \}.
\]

This expression represents a finite intersection of events, given that each \( X_i \) is a random variable. Consequently, \( Z_n \) is indeed a legitimate random variable.\\

Next, we analyze the minimum \( Y_n \). The event \( \{ Y_n > x \} \) can be interpreted as stating that each \( X_i \) is greater than \( x \). Thus, we have:

\[
\{ Y_n > x \} = \{ X_1 > x, X_2 > x, \ldots, X_n > x \}.
\]

We can apply similar reasoning to prove that \( Y_n \) is also a random variable. \\

Now, we proceed to compute the cumulative distribution functions (CDFs) of the random variables \( Z_n \) and \( Y_n \). For \( Z_n \), we have:

\[
P(Z_n \leq x) = P(X_1 \leq x \cap X_2 \leq x \cap \cdots \cap X_n \leq x) = F_{X_1, X_2, \ldots, X_n}(x, x, \ldots, x).
\]

On the other hand, for \( Y_n \), we can express its CDF as:

\[
P(Y_n \leq x) = 1 - P(Y_n > x) = 1 - P(X_1 > x \cap X_2 > x \cap \cdots \cap X_n > x) = 1 - \overline{F_{Y_n}(x)},
\]

where \( \overline{F_{Y_n}(x)} \) denotes the complementary CDF of \( Y_n \). \\

In particular, if the random variables \( X_1, X_2, \ldots, X_n \) are independent, we find that:

\[
F_{Z_n}(x) = F_{X_1}(x) F_{X_2}(x) \cdots F_{X_n}(x),
\]

and for the complementary CDF of \( Y_n \):

\[
\overline{F_{Y_n}(x)} = \overline{F_{X_1}(x)} \overline{F_{X_2}(x)} \cdots \overline{F_{X_n}(x)}.
\]

Furthermore, if the random variables are independent and identically distributed (i.i.d.), the CDFs simplify to:

\[
F_{Z_n}(x) = [F_X(x)]^n,
\]

and

\[
\overline{F_{Y_n}(x)} = [\overline{F_X(x)}]^n.
\]

Thus, we can effectively derive the CDFs for both the maximum and minimum of a set of random variables using their joint distributions and independence properties.

\begin{example}
Consider two independent and identically distributed random variables \( U_1 \) and \( U_2 \), each following a uniform distribution on the interval \([0, 1]\), denoted as \( U \sim \text{Unif}[0, 1] \).\\

We define: \( Y = \min(U_1, U_2) \) and  \( Z = \max(U_1, U_2) \)\\


Let \( F_{U_1}(z) \) and \( F_{U_2}(z) \) represent the cumulative distribution functions (CDFs) of the random variables \( U_1 \) and \( U_2 \), respectively. Since both variables are identically distributed, we have:

\[
F_{U_1}(z) = F_{U_2}(z) = F_U(z)
\]

The CDF of a uniform random variable \( U \) is given by:

\[
F_U(z) =
\begin{cases}
0 & \text{if } z < 0, \\
z & \text{if } z \in [0, 1], \\
1 & \text{if } z > 1.
\end{cases}
\]

Because \( U_1 \) and \( U_2 \) are also independent, the CDF of \( Z \) can be computed as follows:

\[
F_Z(z) = F_{U_1}(z) \cdot F_{U_2}(z) = [F_U(z)]^2.
\]

This gives us:

\[
[F_U(z)]^2 =
\begin{cases}
0 & \text{if } z < 0, \\
z^2 & \text{if } z \in [0, 1], \\
1 & \text{if } z > 1.
\end{cases}
\]

To find the probability density function (pdf) of \( Z \), we differentiate \( F_Z(z) \):

\[
f_Z(z) =
\begin{cases}
2z & \text{if } z \in [0, 1], \\
0 & \text{otherwise}.
\end{cases}
\]

Next, we analyze \( Y \). The complementary CDF of \( Y \), denoted as \( \overline{F_Y}(y) \), is related to the CDFs of \( U_1 \) and \( U_2 \):

\[
\overline{F_Y}(y) = \overline{F_{U_1}}(y) \cdot \overline{F_{U_2}}(y) = [\overline{F_U}(y)]^2,
\]

where \( \overline{F_U}(y) = 1 - F_U(y) \).\\

Thus, we can express the CDF of \( Y \):

\[
F_Y(y) = 1 - [\overline{F_U}(y)]^2.
\]

Substituting for \( \overline{F_U}(y) \):

\[
F_Y(y) =
\begin{cases}
0 & \text{if } y < 0, \\
1 - (1 - y)^2 & \text{if } y \in [0, 1], \\
1 & \text{if } y > 1.
\end{cases}
\]

The pdf of \( Y \) is then obtained by differentiating \( F_Y(y) \):

\[
f_Y(y) =
\begin{cases}
0 & \text{if } y < 0, \\
2(1 - y) & \text{if } y \in [0, 1], \\
1 & \text{if } y > 1.
\end{cases}
\]

\end{example}


\begin{example}
    Let \( X_1, X_2, X_3, \ldots, X_n \) be independent random variables, each following an exponential distribution characterized by parameters \( \lambda_1, \lambda_2, \lambda_3, \ldots, \lambda_n \) where \( \lambda_i > 0 \). The cumulative distribution function (CDF) for each random variable \( X_i \) is given by:

    \[
    F_{X_i}(x) = 1 - e^{-\lambda_i x} \quad \text{for } x > 0.
    \]
    
    Now, let \( Y_n \) denote the minimum of these random variables, defined as:
    
    \[
    Y_n = \min(X_1, X_2, \ldots, X_n).
    \]
    
    To find the complementary CDF of \( Y_n \), denoted as \( \overline{F}_{Y_n}(y) \), we start by recognizing that:
    
    \[
    \overline{F}_{Y_n}(y) = P(Y_n > y) = P(X_1 > y, X_2 > y, \ldots, X_n > y).
    \]
    
    Since the random variables are independent, we can express this probability as the product of their individual complementary CDFs:
    
    \[
    \overline{F}_{Y_n}(y) = \prod_{i=1}^{n} P(X_i > y) = \prod_{i=1}^{n} \overline{F}_{X_i}(y).
    \]
    
    For each \( X_i \), the complementary CDF is:
    
    \[
    \overline{F}_{X_i}(y) = e^{-\lambda_i y}.
    \]
    
    Thus, we have:
    
    \[
    \overline{F}_{Y_n}(y) = \prod_{i=1}^{n} e^{-\lambda_i y} = e^{-\sum_{i=1}^{n} \lambda_i y}.
    \]
    
    This implies that \( Y_n \) is also an exponential random variable, with the new parameter \( \lambda = \sum_{i=1}^{n} \lambda_i \). 
\end{example}

\begin{exercise}
    Light bulbs with Amnesia: Suppose that \( n \) light bulbs in a room are switched on at the same instant. The life time of each bulb is exponentially distributed with parameter \( \mu = 1 \), and are independent.
    \begin{enumerate}[label=(\alph*)]
        \item Starting from the time they are switched on, find the distribution of the time when the first bulb fuses out.
        \item Find the CDF and the density of the time when the room goes completely dark.
        \item Would your answers to the above parts change if the bulbs were not switched on at the same time, but instead, turned on at arbitrary times? Assume however that all bulbs were turned on before the first one fused out.
        \item Suppose you walk into the room and find \( m \) bulbs glowing. Starting from the instant of your walking in, what is the distribution of the time it takes until you see a bulb blow out?
    \end{enumerate}
\end{exercise}

\begin{solution}
    Let \( X_1, X_2, \ldots, X_n \) be the lifetimes of the \( n \) light bulbs, where each \( X_i \) follows an exponential distribution with parameter \( \mu = 1 \). 

    \begin{enumerate}[label=(\alph*)]
        \item The time when the first bulb fuses out is given by \( T = \min(X_1, X_2, \ldots, X_n) \). The minimum of \( n \) independent exponential random variables with rate \( \lambda = 1 \) is also an exponential random variable with rate \( n \):
        \[
        T \sim \text{Exponential}(n).
        \]

        \item The cumulative distribution function (CDF) of \( T \) is given by:
        \[
        F_T(t) = P(T \leq t) = 1 - e^{-nt}, \quad t \geq 0.
        \]
        The probability density function (PDF) is:
        \[
        f_T(t) = \frac{d}{dt} F_T(t) = n e^{-nt}, \quad t \geq 0.
        \]

        \item If the bulbs were not switched on at the same time but turned on at arbitrary times, the answers would change. The distribution of the time until the room goes dark would depend on the timing of each bulb's activation. However, if all bulbs are turned on before the first one fuses out, the distribution of \( T \) would remain the same as in part (a).

        \item Let \( Y_1, Y_2, \ldots, Y_m \) be the lifetimes of the \( m \) bulbs glowing when you enter the room. The distribution of the time until you see a bulb blow out is given by:
        \[
        Z = \min(Y_1, Y_2, \ldots, Y_m).
        \]
        Similar to part (a), \( Z \) is also an exponential random variable with parameter \( m \):
        \[
        Z \sim \text{Exponential}(m).
        \]
    \end{enumerate}
\end{solution}

\begin{exercise}
Let \( X \) and \( Y \) be independent exponentially distributed random variables with parameters \( \lambda \) and \( \mu \) respectively. 
\begin{enumerate}[label=(\alph*)]
    \item Show that \( Z = \min(X, Y) \) is independent of the event \( \{X < Y\} \), and interpret this result verbally? [Definition: A random variable \( X \) is said to be independent of an event \( A \) if \( X \) and \( I_A \) are independent random variables, where \( I_A \) is the Indicator random variable of the event \( A \).]
    \item Find \( P(X = Z) \).
\end{enumerate}
\end{exercise}

\begin{solution}
(a) To show that \( Z = \min(X, Y) \) is independent of the event \( \{X < Y\} \), we start by computing the joint distribution of \( Z \) and \( \{X < Y\} \).\\

The cumulative distribution function (CDF) of \( Z \) is given by:

\[
P(Z \leq z) = P(\min(X, Y) \leq z) = 1 - P(X > z \text{ and } Y > z) = 1 - e^{-\lambda z} e^{-\mu z} = 1 - e^{-(\lambda + \mu) z}.
\]

The probability density function (PDF) of \( Z \) is then:

\[
f_Z(z) = (\lambda + \mu) e^{-(\lambda + \mu) z}, \quad z \geq 0.
\]

Next, we find \( P(X < Y) \):

\[
P(X < Y) = \int_0^{\infty} P(X < y) f_Y(y) \, dy = \int_0^{\infty} (1 - e^{-\lambda y}) \mu e^{-\mu y} \, dy.
\]

Evaluating this integral, we get:

\[
P(X < Y) = \frac{\lambda}{\lambda + \mu}.
\]

Now we need to find \( P(Z \leq z, X < Y) \):

\[
P(Z \leq z, X < Y) = P(\min(X, Y) \leq z, X < Y) = P(X < z) P(Y > z) = \int_0^z \lambda e^{-\lambda x} \cdot e^{-\mu z} \, dx.
\]

Integrating gives:

\[
= (1 - e^{-\lambda z}) e^{-\mu z} = e^{-\mu z} - e^{-(\lambda + \mu) z}.
\]

Thus, we have:

\[
P(Z \leq z, X < Y) = e^{-\mu z} - e^{-(\lambda + \mu) z}.
\]

Since \( P(Z \leq z) P(X < Y) = \left( 1 - e^{-(\lambda + \mu) z} \right) \frac{\lambda}{\lambda + \mu} \), we can show that

\[
P(Z \leq z, X < Y) = P(Z \leq z) P(X < Y).
\]

This indicates that \( Z \) is independent of \( \{X < Y\} \).\\

\textbf{Interpretation:} This result means that the minimum of two independent exponentially distributed random variables does not provide any information about which variable was smaller. It implies a lack of dependency between the minimum value and the event that one variable is less than the other.\\

(b) To find \( P(X = Z) \), we note that:

\[
P(X = Z) = P(X < Y) = \frac{\lambda}{\lambda + \mu}.
\]

This probability can be interpreted as the likelihood that \( X \) is the minimum of \( X \) and \( Y \) when both are independently distributed.
\end{solution}

\section{Sum of Random Variables}

Before we explore the distributions of random variables, it is essential to verify that the sum of two random variables also forms a random variable.

\begin{theorem}
    Let \(X\) and \(Y\) be random variables defined on a probability space \((\Omega, \mathcal{F}, P)\). We define \(Z(\omega) = X(\omega) + Y(\omega)\) for every \(\omega \in \Omega\). Then, \(Z\) is a random variable.
\end{theorem}

\begin{proof}
To demonstrate that \(Z\) is indeed a random variable, we need to show that the set \(\{ \omega \in \Omega : Z(\omega) > z \} \in \mathcal{F}\) for all \(z \in \mathbb{R}\).\\

Now, for any \(z \in \mathbb{R}\), the condition \(Z(\omega) > z\) holds if and only if there exists a rational number \(q\) such that \(X(\omega) > q\) and \(Y(\omega) > z - q\). This equivalence arises from the property that rational numbers are dense in the real numbers. Thus, we can express the set as follows:
\[
\{ \omega \in \Omega : Z(\omega) > z \} = \bigcup_{q \in \mathbb{Q}} \{ \omega \in \Omega : X(\omega) > q, Y(\omega) > z - q \} 
\]
\[
    = \bigcup_{q \in \mathbb{Q}} \left( \{ \omega \in \Omega : X(\omega) > q \} \cap \{ \omega \in \Omega : Y(\omega) > z - q \} \right).
\]

Since we know that for every \(q \in \mathbb{Q}\), both \(\{ \omega \in \Omega : X(\omega) > q \}\) and \(\{ \omega \in \Omega : Y(\omega) > z - q \}\) belong to \(\mathcal{F}\) (because \(X\) and \(Y\) are random variables), their intersection also belongs to \(\mathcal{F}\). Now, because the set of rational numbers \(\mathbb{Q}\) is countable, the union of these sets indexed by \(q\) remains in \(\mathcal{F}\) since \(\mathcal{F}\) is a \(\sigma\)-algebra, which is closed under countable unions.\\

Thus, we conclude that:
\[
\{ \omega \in \Omega : Z(\omega) > z \} \in \mathcal{F},
\]
which proves that the sum \(Z = X + Y\) is a random variable.
\end{proof}

\subsection{Sum of Two Random Variables}

\textbf{Discrete Case}\\

Consider two discrete random variables, \(X\) and \(Y\), which have a known joint probability mass function (pmf) denoted as \(p_{X,Y}(x,y)\). We define a new random variable \(Z\) as follows:

\[
Z = X + Y
\]

Our goal is to characterize the pmf of \(Z\), which we denote as \(p_Z(z)\):

\[
p_Z(z) = P(Z = z) = \sum_{x+y=z} p_{X,Y}(x,y)
\]

This summation can also be expressed in a different form:

\[
p_Z(z) = \sum_{x} P(X = x, Y = z - x) = \sum_{x} p_{X,Y}(x, z - x)
\]

Now, if \(X\) and \(Y\) are independent random variables, the pmf of \(Z\) simplifies to:

\[
p_Z(z) = \sum_{x} p_X(x) p_Y(z - x)
\]

This result is known as the \textit{discrete convolution} of the two pmfs.

\begin{example}
Let \(X\) and \(Y\) be independent random variables, where \(X\) follows a Poisson distribution with parameter \(\lambda\), and \(Y\) follows a Poisson distribution with parameter \(\mu\). We can compute the pmf of \(Z = X + Y\) using the previous result:

\[
p_Z(z) = \sum_{x=0}^{z} e^{-\lambda} \frac{\lambda^x}{x!} e^{-\mu} \frac{\mu^{z-x}}{(z-x)!}
\]

This expression simplifies as follows:

\[
= e^{-(\lambda+\mu)} \frac{1}{z!} \sum_{x=0}^{z} \binom{z}{x} \lambda^x \mu^{z-x}
\]

The summation inside the equation is the binomial expansion of \((\lambda + \mu)^z\), leading us to:

\[
= e^{-(\lambda+\mu)} \frac{(\lambda + \mu)^z}{z!}
\]

This calculation shows that the sum of two independent Poisson-distributed random variables, with mean values \(\lambda\) and \(\mu\), results in another Poisson distribution with mean \(\lambda + \mu\). This method can be easily extended to compute the sum of a finite number of independent Poisson random variables.
\end{example}

\textbf{Continuous Case}\\

We assume that \( X \) and \( Y \) random variables have a joint probability density function (pdf), denoted as \( f_{X,Y}(x, y) \). We define a new random variable \( Z \) as the sum of \( X \) and \( Y \):
\[
Z = X + Y.
\]
To find the cumulative distribution function (CDF) of \( Z \), we express it as:
\[
F_Z(z) = P(Z \leq z) = P(X + Y \leq z).
\]

We can reformulate this probability using double integrals:
\[
F_Z(z) = \int_{-\infty}^{\infty} \int_{-\infty}^{z-x} f_{X,Y}(x, y) \, dy \, dx.
\]
This expression represents the probability that the sum \( X + Y \) is less than or equal to \( z \).\\

Next, we can change the order of integration to facilitate computation:
\[
F_Z(z) = \int_{-\infty}^{\infty} \int_{-\infty}^{z} f_{X,Y}(x, t - x) \, dt \, dx.
\]
Here, we denote the inner integral as:
\[
f_Z(t) = \int_{-\infty}^{\infty} f_{X,Y}(x, t - x) \, dx,
\]
which gives us the pdf of \( Z \). Therefore, we find that the pdf of \( Z \) is expressed as:
\[
f_Z(z) = \int_{-\infty}^{\infty} f_{X,Y}(x, z - x) \, dx.
\]

In the special case where \( X \) and \( Y \) are independent continuous random variables, we have a simplification:
\[
f_Z(z) = \int_{-\infty}^{\infty} f_X(x) f_Y(z - x) \, dx,
\]
which denotes the \textit{convolution} of the two marginal pdfs:
\[
f_Z(z) = f_X \ast f_Y.
\]

\begin{example}
Consider \( X_1 \) and \( X_2 \) as independent exponential random variables with parameters \( \mu_1 \) and \( \mu_2 \), respectively. Let \( Z = X_1 + X_2 \). Using the convolution formula derived earlier, we get:
\[
f_Z(z) = f_{X_1} \ast f_{X_2} = \int_0^{z} \mu_1 e^{-\mu_1 x} \mu_2 e^{-\mu_2 (z - x)} \, dx.
\]
Factoring out constants gives us:
\[
f_Z(z) = \mu_1 \mu_2 e^{-\mu_2 z} \int_0^{z} e^{(\mu_2 - \mu_1)x} \, dx.
\]

Evaluating the integral, we arrive at the final form of \( f_Z(z) \):
\[
f_Z(z) = 
\begin{cases}
\frac{\mu_1 \mu_2}{\mu_2 - \mu_1} \left( e^{-\mu_1 z} - e^{-\mu_2 z} \right) & \text{if } \mu_1 \neq \mu_2, \\
\mu^2 z e^{-\mu z} & \text{if } \mu_1 = \mu_2 = \mu.
\end{cases}
\]

Interestingly, this methodology can be generalized to sums of \( n \) independent exponential random variables. The pdf of such a sum, \( Z_n \), follows an Erlang distribution:
\[
f_{Z_n}(z) = \frac{\mu^n z^{n-1} e^{-\mu z}}{(n - 1)!}.
\]

This example illustrates the process for calculating the pdf of the sum of continuous random variables, and the methods can be readily extended to handle finite sums of various distributions.

\end{example}


\subsection{Sum of Many Random Variables}

In our analysis, we are looking at a scenario where we sum a set of independent random variables, but the count of these variables is also a random quantity. To formalize this, let \( N \) be a random variable that takes positive integer values, defined on a probability space \((\Omega, \mathcal{F}, P)\). We assume we know the probability mass function (pmf) of \( N \), denoted as \( P(N = n) \).\\

Next, let us denote the independent random variables by \( X_1, X_2, \ldots \), which are also defined on the same probability space. Each of these random variables has its own cumulative distribution function (cdf), represented as \( F_{X_i}(\cdot) \) for \( i \geq 1 \). Importantly, we assume that the random variable \( N \) is independent of the collection of random variables \( \{X_i\}_{i \geq 1} \).\\

We define \( S_N \) as the sum of the first \( N \) random variables, which can be expressed mathematically as:
\[
S_N = \sum_{i=1}^{N} X_i.
\]
More specifically, for each outcome \( \omega \) in our sample space \( \Omega \), this can be represented as:
\[
S_N(\omega) = \sum_{i=1}^{N(\omega)} X_i(\omega).
\]

To find the cumulative distribution function of \( S_N \), denoted \( F_{S_N}(x) \), we need to calculate:
\[
F_{S_N}(x) = P(S_N \leq x).
\]
Using the law of total probability, we can expand this expression as follows:
\[
F_{S_N}(x) = \sum_{k=1}^{\infty} P(S_N \leq x \mid N = k) P(N = k).
\]

Here, \( P(S_N \leq x \mid N = k) \) represents the probability that the sum \( S_N \) is less than or equal to \( x \) given that \( N \) equals \( k \). Since \( N \) is independent of the random variables \( X_i \), we can further express this as:
\[
F_{S_N}(x) = \sum_{k=1}^{\infty} P(S_k \leq x) P(N = k),
\]
where \( S_k \) is the sum of the first \( k \) random variables \( X_1, X_2, \ldots, X_k \). This relationship arises directly from the independence of \( N \) and the \( X_i \)'s, allowing us to compute the probabilities separately. \\

In essence, this approach gives us a method to derive the distribution of the sum of a random number of independent random variables by leveraging the known distribution of \( N \) and the distributions of the individual random variables \( X_i \).

\begin{example}
    \textbf{Geometric Sum of Exponentials}. Consider a sequence of independent random variables \( X_i \), where each \( X_i \) follows an exponential distribution with mean \( \mu \). This means that the probability density function (pdf) of each \( X_i \) is given by:

    \[
    f_{X_i}(x) = \frac{1}{\mu} e^{-x/\mu} \quad \text{for } x \geq 0.
    \]
    
    Let \( N \) be a random variable that represents the number of these exponential variables summed together, which follows a geometric distribution with parameter \( p \). \\ 
    
    The probability of \( N \) taking a specific value \( k \) is given by:
    \[
    P(N = k) = (1 - p)^{k-1} p \quad \text{for } k \geq 1.
    \]
    
    We define the sum \( S_N = \sum_{i=1}^N X_i \). Our goal is to find the cumulative distribution function (cdf) of \( S_N \), denoted as \( F_S(x) = P(S_N \leq x) \).\\
    
    We can express \( F_S(x) \) as:
    
    \[
    F_S(x) = \sum_{k=1}^{\infty} P(N = k) F_{S_k}(x),
    \]
    
    where \( S_k = \sum_{i=1}^k X_i \) is the sum of \( k \) exponential random variables. It is known that \( S_k \) follows an Erlang distribution, specifically:
    
    \[
    F_{S_k}(x) = 1 - \sum_{n=0}^{k-1} \frac{(µx)^n e^{-µx}}{n!}.
    \]
    
    Substituting \( P(N = k) \) and \( F_{S_k}(x) \) into the equation for \( F_S(x) \), we have:
    
    \[
    F_S(x) = \sum_{k=1}^{\infty} (1 - p)^{k-1} p \left( 1 - \sum_{n=0}^{k-1} \frac{(µx)^n e^{-µx}}{n!} \right).
    \]
    
    This can be simplified to:
    
    \[
    = 1 - e^{-µx} \sum_{k=1}^{\infty} (1 - p)^{k-1} \frac{(µx)^k}{k!}.
    \]
    
    Now, recognizing the series for the exponential function, we can rewrite the summation:
    
    \[
    \sum_{k=1}^{\infty} (1 - p)^{k-1} \frac{(µx)^k}{k!} = (µx) e^{(1 - p) µ x}.
    \]
    
    Therefore, we have:
    
    \[
    F_S(x) = 1 - e^{-µx} \cdot \frac{(µx)(1 - p)}{(1 - p)} e^{(1 - p) µ x},
    \]
    
    which simplifies to:
    
    \[
    = 1 - e^{-(pµ)x}.
    \]
    
    Thus, we conclude that the geometric sum of independent exponential random variables results in another exponential distribution with a modified parameter.
\end{example}

Consider a radioactive source that emits \(\alpha\) particles. The time between two successive emissions follows an exponential distribution characterized by a parameter \(\lambda\). Each time an emission occurs, a detector has a probability \(p\) of detecting it, and a probability \(1 - p\) of missing it. Importantly, these detection events are independent of one another.\\

From this setup, we can conclude that the time between two successive detections behaves like a geometric sum of independent and identically distributed (i.i.d.) exponential random variables. The resulting distribution of the time until detection is also an exponential random variable, but with a new parameter \(p\lambda\). This is an important finding because it illustrates how the combination of random events can yield a new type of randomness.\\

Now, let’s explore a more intricate scenario. Imagine a gambler who plays a game multiple times, receiving either rewards or penalties with each round. The gambler continues to play until they feel satisfied with their winnings or, conversely, until they are broke. In this case, let \(X_i\) represent the amount gained (or lost) in the \(i\)-th round.\\

Here, the total earnings of the gambler at the end of the game becomes more complex to analyze because the number of rounds \(N\) played depends on the results of the rounds themselves, namely the outcomes \(X_i\). This situation illustrates the idea of \textit{stopping rules}, which is a significant topic in probability theory and will be explored in more advanced courses. The need to understand this dependence opens up new avenues for research and exploration in stochastic processes, particularly in scenarios where decisions are made based on previous outcomes, rather than occurring independently.

\begin{exercise}
Let \( X_1 \) and \( X_2 \) be independent random variables with distributions \( N(0, \sigma_1^2) \) and \( N(0, \sigma_2^2) \) respectively. Show that the distribution of \( X_1 + X_2 \) is \( N(0, \sigma_1^2 + \sigma_2^2) \).
\end{exercise}

\begin{solution}
To show that \( Y = X_1 + X_2 \) follows a normal distribution, we can leverage the property of the sum of independent normal variables.\\

   The characteristic function of a normally distributed random variable \( X \sim N(0, \sigma^2) \) is given by:
   \[
   \phi_X(t) = e^{-\frac{1}{2} \sigma^2 t^2}.
   \]

   For \( X_1 \sim N(0, \sigma_1^2) \):
   \[
   \phi_{X_1}(t) = e^{-\frac{1}{2} \sigma_1^2 t^2}.
   \]

   For \( X_2 \sim N(0, \sigma_2^2) \):
   \[
   \phi_{X_2}(t) = e^{-\frac{1}{2} \sigma_2^2 t^2}.
   \]

   Since \( X_1 \) and \( X_2 \) are independent, the characteristic function of \( Y = X_1 + X_2 \) is the product of the individual characteristic functions:
   \[
   \phi_Y(t) = \phi_{X_1}(t) \cdot \phi_{X_2}(t) = e^{-\frac{1}{2} \sigma_1^2 t^2} \cdot e^{-\frac{1}{2} \sigma_2^2 t^2} = e^{-\frac{1}{2} (\sigma_1^2 + \sigma_2^2) t^2}.
   \]

   The resulting characteristic function \( \phi_Y(t) = e^{-\frac{1}{2} (\sigma_1^2 + \sigma_2^2) t^2} \) corresponds to the characteristic function of a normal distribution with mean \( 0 \) and variance \( \sigma_1^2 + \sigma_2^2 \).
\end{solution}

\begin{exercise}
    Consider two independent and identically distributed discrete random variables \( X \) and \( Y \). Assume that their common PMF, denoted by \( p(z) \), is symmetric around zero, i.e., \( p(z) = p(-z) \), \( \forall z \). Show that the PMF of \( X + Y \) is also symmetric around zero and is largest at zero.
\end{exercise}

\begin{solution}
    Let \( Z = X + Y \). To find the PMF of \( Z \), we will compute \( P(Z = k) \) for any integer \( k \).\\

    Since \( X \) and \( Y \) are independent, we have:
    \[
    P(Z = k) = P(X + Y = k) = \sum_{j} P(X = j) P(Y = k - j) = \sum_{j} p(j) p(k - j)
    \]

    Now, we will show that \( P(Z = k) = P(Z = -k) \) for all integers \( k \):
    \[
    P(Z = -k) = \sum_{j} P(X = j) P(Y = -k - j) = \sum_{j} p(j) p(-k - j)
    \]
    Using the symmetry of the PMF \( p(z) \), we have \( p(-k - j) = p(k + j) \), so:
    \[
    P(Z = -k) = \sum_{j} p(j) p(k + j)
    \]

    Now, we change the index of summation by letting \( i = k + j \), which gives \( j = i - k \). Then the limits of summation change accordingly:
    \[
    P(Z = -k) = \sum_{i} p(i - k) p(i) = \sum_{i} p(i) p(k - i)
    \]
    This is equal to \( P(Z = k) \):
    \[
    P(Z = -k) = P(Z = k)
    \]
    Thus, the PMF \( P(Z = k) \) is symmetric around zero.\\

    To show that it is largest at zero, we note:
    \[
    P(Z = 0) = \sum_{j} P(X = j) P(Y = -j) = \sum_{j} p(j) p(-j) = \sum_{j} p(j)^2
    \]
    Since all \( p(j) \) are non-negative, \( P(Z = 0) \) is a sum of squares, which is maximized when \( j = 0 \).\\

    For any \( k \neq 0 \):
    \[
    P(Z = k) = \sum_{j} p(j) p(k - j)
    \]
    Each term \( p(j) p(k - j) \) will generally be less than or equal to \( p(0) \) due to the nature of convolution of symmetric distributions, meaning \( P(Z = k) < P(Z = 0) \).\\

    Therefore, the PMF of \( Z = X + Y \) is symmetric around zero and is largest at zero.
\end{solution}

\begin{exercise}
    Suppose \( X \) and \( Y \) are independent random variables with \( Z = X + Y \) such that 
    \[
    f_X(x) = c e^{-cx}, \quad x \geq 0
    \]
    and 
    \[
    f_Z(z) = \frac{c^2}{2} z e^{-cz}, \quad z \geq 0.
    \]
    Compute \( f_Y(y) \).
\end{exercise}

\begin{solution}
    Given that \( X \) and \( Y \) are independent random variables, the joint probability density function (PDF) of \( X \) and \( Y \) can be expressed as the product of their individual PDFs:
    \[
    f_{X,Y}(x,y) = f_X(x) f_Y(y).
    \]
    
    The PDF of \( Z \), which is the sum of \( X \) and \( Y \), is given by the convolution of the PDFs of \( X \) and \( Y \):
    \[
    f_Z(z) = \int_0^z f_X(x) f_Y(z - x) \, dx.
    \]

    From the problem, we have:
    \[
    f_Z(z) = \frac{c^2}{2} z e^{-cz}.
    \]
    
    We know the PDF of \( X \):
    \[
    f_X(x) = c e^{-cx}.
    \]
    
    To find \( f_Y(y) \), we first need to express \( f_Z(z) \) in terms of \( f_Y(y) \). Assuming \( f_Y(y) \) has the same functional form as \( f_X(x) \), we let:
    \[
    f_Y(y) = k e^{-ky},
    \]
    where \( k \) is a constant to be determined.\\

    Substituting \( f_Y(y) \) into the convolution integral, we have:
    \[
    f_Z(z) = \int_0^z f_X(x) f_Y(z - x) \, dx = \int_0^z c e^{-cx} k e^{-k(z - x)} \, dx.
    \]
    
    This simplifies to:
    \[
    f_Z(z) = ck e^{-kz} \int_0^z e^{(k - c)x} \, dx.
    \]
    
    Evaluating the integral:
    \[
    \int_0^z e^{(k - c)x} \, dx = \frac{1}{k - c}(e^{(k - c)z} - 1).
    \]
    
    Therefore,
    \[
    f_Z(z) = ck e^{-kz} \cdot \frac{1}{k - c}(e^{(k - c)z} - 1).
    \]

    We need this to equal \( \frac{c^2}{2} z e^{-cz} \). Matching the coefficients of \( z e^{-cz} \) leads to a system of equations, and solving this system gives the values of \( c \) and \( k \).\\

    Eventually, through careful consideration, we find:
    \[
    k = c,
    \]
    which implies:
    \[
    f_Y(y) = c e^{-cy}, \quad y \geq 0.
    \]
\end{solution}

\begin{exercise}
Let \( X_1 \) and \( X_2 \) be the number of calls arriving at a switching centre from two different localities at a given instant of time. \( X_1 \) and \( X_2 \) are well modelled as independent Poisson random variables with parameters \( \lambda_1 \) and \( \lambda_2 \) respectively.
\begin{enumerate}[label=(\alph*)]
    \item Find the PMF of the total number of calls arriving at the switching centre.
    \item Find the conditional PMF of \( X_1 \) given the total number of calls arriving at the switching centre is \( n \).
\end{enumerate}
\end{exercise}

\begin{solution}

(a) The total number of calls arriving at the switching centre can be expressed as \( X = X_1 + X_2 \). Since \( X_1 \) and \( X_2 \) are independent Poisson random variables, the PMF of \( X \) can be derived from the property of the sum of independent Poisson random variables.\\

The PMF of a Poisson random variable \( X_i \) with parameter \( \lambda_i \) is given by:

\[
P(X_i = k) = \frac{\lambda_i^k e^{-\lambda_i}}{k!}, \quad k = 0, 1, 2, \ldots
\]

Since \( X_1 \) and \( X_2 \) are independent, the PMF of the total number of calls \( X \) is:

\[
P(X = n) = \sum_{k=0}^{n} P(X_1 = k) P(X_2 = n-k)
\]

This can be computed as follows:

\[
P(X = n) = \sum_{k=0}^{n} \frac{\lambda_1^k e^{-\lambda_1}}{k!} \cdot \frac{\lambda_2^{n-k} e^{-\lambda_2}}{(n-k)!}
\]

Recognizing that this sum represents the PMF of a Poisson distribution with parameter \( \lambda_1 + \lambda_2 \), we have:

\[
P(X = n) = \frac{(\lambda_1 + \lambda_2)^n e^{-(\lambda_1 + \lambda_2)}}{n!}
\]

Thus, \( X \) is also a Poisson random variable with parameter \( \lambda_1 + \lambda_2 \).\\

(b) To find the conditional PMF of \( X_1 \) given that the total number of calls arriving at the switching centre is \( n \), we apply the formula for conditional probability:

\[
P(X_1 = k \mid X = n) = \frac{P(X_1 = k, X_2 = n-k)}{P(X = n)}
\]

Using the independence of \( X_1 \) and \( X_2 \):

\[
P(X_1 = k, X_2 = n-k) = P(X_1 = k) P(X_2 = n-k) = \frac{\lambda_1^k e^{-\lambda_1}}{k!} \cdot \frac{\lambda_2^{n-k} e^{-\lambda_2}}{(n-k)!}
\]

Substituting into the conditional probability gives:

\[
P(X_1 = k \mid X = n) = \frac{\frac{\lambda_1^k e^{-\lambda_1}}{k!} \cdot \frac{\lambda_2^{n-k} e^{-\lambda_2}}{(n-k)!}}{\frac{(\lambda_1 + \lambda_2)^n e^{-(\lambda_1 + \lambda_2)}}{n!}}
\]

Simplifying this expression yields:

\[
P(X_1 = k \mid X = n) = \frac{n!}{k!(n-k)!} \cdot \frac{\lambda_1^k \lambda_2^{n-k}}{(\lambda_1 + \lambda_2)^n}
\]

This shows that \( X_1 \mid (X = n) \) follows a binomial distribution:

\[
X_1 \mid (X = n) \sim \text{Binomial}(n, p) \quad \text{where } p = \frac{\lambda_1}{\lambda_1 + \lambda_2}
\]
\end{solution}

\begin{exercise}
The random variables \(X\), \(Y\), and \(Z\) are independent and uniformly distributed between zero and one. Find the PDF of \(X + Y + Z\).
\end{exercise}

\begin{solution}
To find the probability density function (PDF) of the sum \(S = X + Y + Z\), where \(X\), \(Y\), and \(Z\) are independent random variables uniformly distributed on \([0, 1]\), we can use the convolution of their individual PDFs.\\

The PDF of a uniform random variable \(U\) on \([0, 1]\) is given by:

\[
f_U(u) = 
\begin{cases} 
1 & \text{if } 0 \leq u \leq 1 \\ 
0 & \text{otherwise} 
\end{cases}
\]

The PDF of the sum of two independent random variables can be found using the convolution of their PDFs:

\[
f_{X+Y}(s) = \int_{0}^{s} f_X(x) f_Y(s - x) \, dx
\]

Since \(f_X\) and \(f_Y\) are both equal to \(1\) on \([0, 1]\):\\

1. For \(0 \leq s < 1\):
\[
f_{X+Y}(s) = \int_{0}^{s} 1 \cdot 1 \, dx = s
\]

2. For \(1 \leq s < 2\):
\[
f_{X+Y}(s) = \int_{s-1}^{1} 1 \cdot 1 \, dx = 2 - s
\]

The PDF \(f_{X+Y}(s)\) is given by:

\[
f_{X+Y}(s) = 
\begin{cases} 
s & \text{if } 0 \leq s < 1 \\ 
2 - s & \text{if } 1 \leq s < 2 \\ 
0 & \text{otherwise} 
\end{cases}
\]

Next, we need to find the PDF of \(S = X + Y + Z\). We will convolve \(f_{X+Y}(s)\) with \(f_Z(z)\).\\

The convolution is given by:

\[
f_{S}(s) = \int_{0}^{s} f_{X+Y}(x) f_Z(s - x) \, dx
\]

Since \(f_Z(z) = 1\) for \(0 \leq z \leq 1\), we consider two cases for \(s\):\\

1. For \(0 \leq s < 1\):
\[
f_S(s) = \int_{0}^{s} x \cdot 1 \, dx = \frac{s^2}{2}
\]

2. For \(1 \leq s < 2\):
\[
f_S(s) = \int_{0}^{1} x \cdot 1 \, dx + \int_{1}^{s} (2 - x) \cdot 1 \, dx = \frac{1}{2} + (2s - \frac{s^2}{2} - 1) = 2 - \frac{s^2}{2}
\]

3. For \(2 \leq s < 3\):
\[
f_S(s) = \int_{s-2}^{1} (2 - x) \cdot 1 \, dx = (2 - (s - 2)) = 4 - s
\]

Thus, the PDF \(f_S(s)\) is given by:

\[
f_S(s) = 
\begin{cases} 
\frac{s^2}{2} & \text{if } 0 \leq s < 1 \\ 
2 - \frac{s^2}{2} & \text{if } 1 \leq s < 2 \\ 
4 - s & \text{if } 2 \leq s < 3 \\ 
0 & \text{otherwise} 
\end{cases}
\]

\end{solution}


\begin{exercise}
Construct an example to show that the sum of a random number of independent normal random variables is not normal.
\end{exercise}

\begin{solution}
To demonstrate that the sum of a random number of independent normal random variables can result in a non-normal distribution, consider the following example:\\

Let \(X_1, X_2, \ldots, X_n\) be independent normal random variables, each distributed as \(X_i \sim N(\mu_i, \sigma_i^2)\) for \(i = 1, 2, \ldots, n\). We will sum these variables, but first, we will allow the number of variables \(n\) to be a random variable itself.\\

Let \(N\) be a random variable that follows a Poisson distribution with parameter \(\lambda\), i.e., \(N \sim \text{Poisson}(\lambda)\).\\

Define the sum of the random variables as follows:
   \[
   S = \sum_{i=1}^{N} X_i
   \]
   where \(N\) is the number of terms in the sum.\\

The random variable \(S\) is conditioned on \(N\), and its distribution can be understood through the law of total probability:
   \[
   S | N = n \sim N\left(n \mu, n \sigma^2\right)
   \]

To find the marginal distribution of \(S\), we need to consider all possible values of \(N\):
   \[
   f_S(s) = \sum_{n=0}^{\infty} f_{S | N}(s | n) P(N = n)
   \]
   However, since \(N\) follows a Poisson distribution, the total sum \(S\) will have a mixed distribution depending on the realization of \(N\).\\

Suppose each \(X_i \sim N(0, 1)\). If \(N\) is a Poisson random variable with \(\lambda = 2\), the resulting distribution of \(S\) is a mixture of normal distributions, which is not normal overall. This can be shown through simulation or density plots.\\

Thus, the sum \(S = \sum_{i=1}^{N} X_i\) does not follow a normal distribution due to the randomness in the number of summands.
\end{solution}

\section{General Transformation of Random Variables}

We have previously explored some fundamental transformations of random variables, such as the sums of random variables and their maximum and minimum values. Now, we will delve into more general transformations of random variables. The motivation for transforming a random variable can be illustrated with the following example: \\

Imagine we have a particle whose velocity is represented by a random variable \( V \). Each specific realization of the velocity corresponds to a particular value of kinetic energy, denoted as \( E \). Our goal is to understand the distribution of the kinetic energy \( E \), which depends on the original random variable \( V \) through a transformation.\\

Mathematically, this can be expressed as:

\[
E = f(V)
\]

where \( f \) is a function that defines how the velocity \( V \) translates into kinetic energy \( E \). In this case, the distribution of the kinetic energy \( E \) is derived from the distribution of the velocity \( V \). \\

Such scenarios are common in practical applications, where we often need to study new random variables that arise from transformations of existing ones. Understanding these transformations not only helps us derive the properties of the new variables but also deepens our insight into the underlying stochastic processes.

\subsection{Transformation of a Single Random Variable}

Consider a random variable \( X : \Omega \to \mathbb{R} \) and a Borel measurable function \( g : \mathbb{R} \to \mathbb{R} \). When we define a new random variable \( Y = g(X) \), we are interested in determining the distribution of \( Y \). Specifically, we want to find the cumulative distribution function (CDF) \( F_Y(y) \) based on the CDF \( F_X(x) \).\\

The CDF of \( Y \) can be expressed as:
\[
F_Y(y) = P(g(X) \leq y) = P\{ \omega \in \Omega \mid g(X(\omega)) \leq y \}.
\]
To facilitate our calculation, we define the set \( B_y \) as the collection of all \( x \) such that \( g(x) \leq y \). Thus, we have:
\[
F_Y(y) = P_X(B_y).
\]

\begin{example}
    Let \( X \) be a Gaussian random variable with mean \( 0 \) and variance \( 1 \), denoted as \( X \sim N(0, 1) \). We want to find the distribution of \( Y = X^2 \).\\

    To compute this, we first express the CDF of \( Y \):
    \[
    F_Y(y) = P(X^2 \leq y).
    \]
    This is equivalent to finding the probabilities for \( X \) being in the interval \( [-\sqrt{y}, \sqrt{y}] \):
    \[
    F_Y(y) = P(-\sqrt{y} \leq X \leq \sqrt{y}).
    \]
    We can calculate this probability using the CDF of the standard normal distribution, \( \Phi \):
    \[
    F_Y(y) = \Phi(\sqrt{y}) - \Phi(-\sqrt{y}).
    \]
    Since the CDF of the standard normal distribution is symmetric, we have:
    \[
    F_Y(y) = 2 \Phi(\sqrt{y}).
    \]
    
    Next, we differentiate \( F_Y(y) \) with respect to \( y \) to find the probability density function (PDF) \( f_Y(y) \):
    \[
    f_Y(y) = \frac{dF_Y(y)}{dy}.
    \]
    Calculating \( F_Y(y) \) explicitly, we have:
    \[
    F_Y(y) = \int_{-\sqrt{y}}^{\sqrt{y}} \frac{1}{\sqrt{2\pi}} e^{-\frac{t^2}{2}} dt = 2 \int_{0}^{\sqrt{y}} \frac{1}{\sqrt{2\pi}} e^{-\frac{t^2}{2}} dt.
    \]
    Changing variables by letting \( t^2 = u \) leads to:
    \[
    F_Y(y) = 2 \int_{0}^{y} \frac{1}{2\sqrt{2\pi u}} e^{-\frac{u}{2}} du.
    \]
    Consequently, the density function becomes:
    \[
    f_Y(y) = \frac{1}{\sqrt{2\pi y}} e^{-\frac{y}{2}}, \quad \text{for } y > 0.
    \]    
\end{example}

\textit{Important Notes:}\\

1. The random variable \( Y \) can only take non-negative values since it is derived from squaring the real-valued random variable \( X \).\\
2. The distribution of \( Y \) as the square of a Gaussian random variable is recognized as the Chi-squared distribution.\\

Thus, we observe that given the distribution of the random variable \( X \), the distribution of any function of \( X \) can be derived using fundamental principles. We now come up with a direct formula to find the distribution of a function of the random variable in the cases where the function is differentiable and monotonic.\\

\textbf{The Generic Formula} \\

Let \( X \) be a random variable with a probability density function \( f_X(x) \), and let \( g \) be a monotonically increasing function. We define a new random variable \( Y \) as follows:

\[
Y = g(X).
\]

To find the cumulative distribution function (CDF) of \( Y \), denoted \( F_Y(y) \), we start by expressing it in terms of \( Y \):

\[
F_Y(y) = P(Y \leq y) = P(g(X) \leq y).
\]

Since \( g \) is monotonically increasing, the inequality \( g(X) \leq y \) implies that 

\[
X \leq g^{-1}(y).
\]

Thus, we can express the CDF of \( Y \) as:

\[
F_Y(y) = P(X \leq g^{-1}(y)) = \int_{-\infty}^{g^{-1}(y)} f_X(x) \, dx.
\]

Now, let us substitute \( x = g^{-1}(t) \). By the chain rule of differentiation, we have:

\[
dg(x) = g'(x) \, dx \Rightarrow dx = \frac{dt}{g'(g^{-1}(t))}.
\]

Therefore, the CDF can be rewritten as:

\[
F_Y(y) = \int_{-\infty}^{y} f_X(g^{-1}(t)) \frac{dt}{g'(g^{-1}(t))}.
\]

Next, to find the probability density function (PDF) \( f_Y(y) \), we differentiate the CDF \( F_Y(y) \):

\[
f_Y(y) = \frac{d}{dy} F_Y(y) = f_X(g^{-1}(y)) \cdot \frac{1}{g'(g^{-1}(y))}.
\]

The term \( \frac{1}{g'(g^{-1}(y))} \) is known as the Jacobian of the transformation \( g(\cdot) \). \\

Now, we can also apply a similar logic for a monotonically decreasing function \( g \). In this case, we would find:

\[
f_Y(y) = f_X(g^{-1}(y)) \cdot \left(-\frac{1}{g'(g^{-1}(y))}\right).
\]

Thus, for any monotonic function \( g \), the general formula for the probability density function of \( Y \) can be summarized as:

\[
f_Y(y) = f_X(g^{-1}(y)) \cdot |g'(g^{-1}(y))|.
\]

\begin{example}
Let \( X \sim N(0, 1) \). We want to find the distribution of \( Y = e^X \).\\

To start, observe that the function \( g(x) = e^x \) is both differentiable and monotonically increasing. The inverse function of \( g \) is given by:

\[
g^{-1}(y) = \ln(y)
\]

Next, we can compute the derivative of \( g \):

\[
g'(x) = e^x
\]

Thus, we find:

\[
g'(g^{-1}(y)) = g'(\ln(y)) = y
\]

Since \( g'(g^{-1}(y)) \) is positive for all values of \( y > 0 \), we can apply the change of variables formula to find the probability density function (pdf) of \( Y \):

\[
f_Y(y) = f_X(\ln(y)) \cdot \left| \frac{d}{dy} g^{-1}(y) \right|
\]

Substituting the known values, we get:

\[
f_Y(y) = f_X(\ln(y)) \cdot \frac{1}{y}
\]

Given that \( f_X(x) \) for \( X \sim N(0, 1) \) is:

\[
f_X(x) = \frac{1}{\sqrt{2\pi}} e^{-\frac{x^2}{2}},
\]

we can replace \( x \) with \( \ln(y) \):

\[
f_Y(y) = \frac{1}{\sqrt{2\pi}} e^{-\frac{(\ln(y))^2}{2}} \cdot \frac{1}{y}
\]

Therefore, the pdf of \( Y \) is given by:

\[
f_Y(y) = \frac{1}{y\sqrt{2\pi}} e^{-\frac{(\ln(y))^2}{2}} \quad \text{for } y > 0.
\]

This result indicates that \( Y \) follows a log-normal distribution.
\end{example}


\begin{example}
Let \( U \sim \text{Uniform}(0, 1) \), meaning \( U \) is a uniform random variable on the interval \([0, 1]\). We want to find the distribution of \( Y = -\ln(U) \).\\

Here, the function \( g(u) = -\ln(u) \) is differentiable and monotonically decreasing. Its inverse function is:

\[
g^{-1}(y) = e^{-y}
\]

Now, we compute the derivative of \( g \):

\[
g'(u) = -\frac{1}{u}
\]

Thus, we have:

\[
g'(g^{-1}(y)) = g'(e^{-y}) = -e^y
\]

Since \( g'(g^{-1}(y)) \) is negative for all values of \( y \), we compute the absolute value of the Jacobian:

\[
\left| g'(g^{-1}(y)) \right| = -g'(g^{-1}(y)) = \frac{1}{e^{-y}} = e^y.
\]

Now, we can use the change of variables formula to find the pdf of \( Y \):

\[
f_Y(y) = f_U(e^{-y}) \cdot |g'(g^{-1}(y))|.
\]

Given that \( f_U(u) = 1 \) for \( U \sim \text{Uniform}(0, 1) \):

\[
f_Y(y) = 1 \cdot e^y = e^{-y} \quad \text{for } y \geq 0.
\]

This indicates that \( Y \) is an exponential random variable with mean \( 1 \).
\end{example}

\subsection{Transformation of Many Random Variables}

The generic formula for transformations can indeed be expanded to encompass multiple random variables. To illustrate this, let us consider an n-tuple random variable, denoted as \((X_1, X_2, \ldots, X_n)\). This random variable has a joint density function represented by 

\[
f_{X_1, X_2, \ldots, X_n}(x_1, x_2, \ldots, x_n).
\]

Now, we define transformations of these variables as follows:

\[
Y_1 = g_1(X_1, X_2, \ldots, X_n), \quad Y_2 = g_2(X_1, X_2, \ldots, X_n), \quad \ldots, \quad Y_n = g_n(X_1, X_2, \ldots, X_n.
\]

For convenience, we can succinctly express this transformation as a vector:

\[
\mathbf{Y} = \mathbf{g}(\mathbf{X}),
\]

where \(\mathbf{g}: \mathbb{R}^n \to \mathbb{R}^n\).\\

We proceed with the assumption that the transformation \(\mathbf{g}\) is both invertible and continuously differentiable. Under these conditions, we can derive the joint density of the transformed variables, expressed as:

\[
f_{Y_1, Y_2, \ldots, Y_n}(y_1, y_2, \ldots, y_n) = f_{X_1, X_2, \ldots, X_n}(g^{-1}(y)) \cdot |J(y)|,
\]

where \(|J(y)|\) denotes the Jacobian determinant, which is a crucial element in transforming densities. \\

The Jacobian matrix \(J(y)\) is defined as:

\[
J(y) =
\begin{bmatrix}
\frac{\partial x_1}{\partial y_1} & \frac{\partial x_2}{\partial y_1} & \cdots & \frac{\partial x_n}{\partial y_1} \\
\frac{\partial x_1}{\partial y_2} & \frac{\partial x_2}{\partial y_2} & \cdots & \frac{\partial x_n}{\partial y_2} \\
\vdots & \vdots & \ddots & \vdots \\
\frac{\partial x_1}{\partial y_n} & \frac{\partial x_2}{\partial y_n} & \cdots & \frac{\partial x_n}{\partial y_n}
\end{bmatrix}.
\]

The Jacobian serves as a scaling factor that adjusts the volume of the transformed space relative to the original space. When we perform a change of variables in probability, the transformation can stretch or compress areas of the probability density function. The absolute value of the Jacobian determinant captures this effect mathematically. \\

To understand this better, consider a simple two-dimensional example where we transform a shape in the \(XY\)-plane to a new shape in the \(UV\)-plane. If the transformation causes the area of a small rectangle in the \(XY\)-plane to become larger or smaller in the \(UV\)-plane, the Jacobian tells us exactly how much to scale the density of points in that area to preserve the total probability.

\begin{example}
Consider a particle whose Euclidean coordinates \(X\) and \(Y\) are drawn from independent Gaussian random variables with mean 0 and variance 1, i.e., \(X, Y \sim N(0, 1)\). We aim to find the distribution of the particle's polar coordinates \(R\) and \(\Theta\). \\

The transformations from Cartesian to polar coordinates are given by:
\[
X = R \cos \Theta \quad \text{and} \quad Y = R \sin \Theta.
\]
To determine the distribution of \(R\) and \(\Theta\), we first calculate the Jacobian of the transformation. The partial derivatives are as follows:
\[
\frac{\partial x}{\partial r} = \cos \theta, \quad \frac{\partial y}{\partial r} = \sin \theta,
\]
\[
\frac{\partial x}{\partial \theta} = -r \sin \theta, \quad \frac{\partial y}{\partial \theta} = r \cos \theta.
\]
Thus, the Jacobian \(J\) is computed as:
\[
J = \begin{vmatrix}
\frac{\partial x}{\partial r} & \frac{\partial y}{\partial r} \\
\frac{\partial x}{\partial \theta} & \frac{\partial y}{\partial \theta}
\end{vmatrix} = \begin{vmatrix}
\cos \theta & \sin \theta \\
-r \sin \theta & r \cos \theta
\end{vmatrix} = r(\cos^2 \theta + \sin^2 \theta) = r.
\]
Next, since \(X\) and \(Y\) are independent random variables, their joint probability density function is given by:
\[
f_{X,Y}(x, y) = f_X(x) f_Y(y) = \frac{1}{2\pi} e^{-\frac{x^2 + y^2}{2}}, \quad x, y \in \mathbb{R}.
\]
Using the transformation to polar coordinates, we find:
\[
f_{R,\Theta}(r, \theta) = f_{X,Y}(r \cos \theta, r \sin \theta) \cdot |J| = \frac{1}{2\pi} e^{-\frac{(r \cos \theta)^2 + (r \sin \theta)^2}{2}} \cdot r = \frac{1}{2\pi} e^{-\frac{r^2}{2}} \cdot r,
\]
where \(r \geq 0\) and \(\theta \in [0, 2\pi]\).\\

To find the marginal densities of \(R\) and \(\Theta\), we integrate the joint distribution:
\[
f_R(r) = \int_0^{2\pi} f_{R,\Theta}(r, \theta) \, d\theta = r e^{-r^2/2}, \quad \text{for } r \geq 0,
\]
\[
f_\Theta(\theta) = \int_0^\infty f_{R,\Theta}(r, \theta) \, dr = \frac{1}{2\pi}, \quad \text{for } \theta \in [0, 2\pi].
\]
The distribution \(f_R(r)\) is known as the Rayleigh distribution, which is commonly used in wireless communications to model the gain of a fading channel. 
\end{example}

It is noteworthy that \(R\) and \(\Theta\) are independent random variables since the joint distribution factors into the product of the marginals:
\[
f_{R,\Theta}(r, \theta) = f_R(r) \cdot f_\Theta(\theta).
\]

\begin{exercise}
Let \( X \sim \text{Exp}(0.5) \). Prove that \( Y = \sqrt{X} \) is a Rayleigh distributed random variable.
\end{exercise}

\begin{solution}
To show that \( Y = \sqrt{X} \) is a Rayleigh distributed random variable, we start with the probability density function (PDF) of the exponential distribution. The PDF of \( X \sim \text{Exp}(\lambda) \) is given by:

\[
f_X(x) = \lambda e^{-\lambda x} \quad \text{for } x \geq 0.
\]

In our case, \( \lambda = 0.5 \), so:

\[
f_X(x) = 0.5 e^{-0.5 x} \quad \text{for } x \geq 0.
\]

Next, we find the cumulative distribution function (CDF) of \( X \):

\[
F_X(x) = \int_0^x f_X(t) \, dt = \int_0^x 0.5 e^{-0.5 t} \, dt.
\]

Evaluating the integral, we have:

\[
F_X(x) = -e^{-0.5 t} \bigg|_0^x = -e^{-0.5 x} + 1 = 1 - e^{-0.5 x}.
\]

Now we find the distribution of \( Y = \sqrt{X} \). To do this, we first express \( X \) in terms of \( Y \):

\[
X = Y^2.
\]

Next, we need to find the PDF of \( Y \). We can use the transformation method for random variables. The relationship between \( X \) and \( Y \) gives us:

\[
f_Y(y) = f_X(x) \left| \frac{dx}{dy} \right|,
\]

where \( x = y^2 \). Calculating the derivative:

\[
\frac{dx}{dy} = 2y.
\]

Thus, we can write:

\[
f_Y(y) = f_X(y^2) \cdot |2y|.
\]

Substituting for \( f_X(y^2) \):

\[
f_Y(y) = 0.5 e^{-0.5 y^2} \cdot 2y = y e^{-0.5 y^2}.
\]

This is the PDF of a Rayleigh distribution. The PDF of a Rayleigh distributed random variable with parameter \( \sigma \) is given by:

\[
f_Y(y) = \frac{y}{\sigma^2} e^{-\frac{y^2}{2\sigma^2}} \quad \text{for } y \geq 0.
\]

Setting \( \sigma^2 = 2 \), we get:

\[
f_Y(y) = \frac{y}{2} e^{-\frac{y^2}{4}}.
\]

Since this matches the form we derived, we conclude that \( Y = \sqrt{X} \) is indeed Rayleigh distributed with parameter \( \sigma = 1 \).
\end{solution}

\begin{exercise}
Let \( X \) be a random variable with a continuous distribution \( F \).
\begin{enumerate}
    \item[(i)] Show that the random variable \( Y = F(X) \) is uniformly distributed over \( [0, 1] \). \textit{[Hint: Although \( F \) is the distribution of \( X \), regard it simply as a function satisfying certain properties required to make it a CDF!]}
    \item[(ii)] Now, given that \( Y = y \), a random variable \( Z \) is distributed as Geometric with parameter \( y \). Find the unconditional PMF of \( Z \). Also, given \( Z = z \) for some \( z \geq 1, z \in \mathbb{N} \), find the conditional PMF of \( Y \).
\end{enumerate}
\end{exercise}

\begin{solution}
To show that the random variable \( Y = F(X) \) is uniformly distributed over \( [0, 1] \), we start by computing the cumulative distribution function (CDF) of \( Y \). \\

The CDF of \( Y \) is given by:
\[
P(Y \leq y) = P(F(X) \leq y.
\]
For \( y \in [0, 1] \), this can be rewritten using the properties of \( F \):
\[
P(F(X) \leq y) = P(X \leq F^{-1}(y)).
\]
Since \( F \) is continuous and strictly increasing, we have:
\[
P(X \leq F^{-1}(y)) = F(F^{-1}(y)) = y.
\]
Therefore, the CDF of \( Y \) is:
\[
P(Y \leq y) = y \quad \text{for } y \in [0, 1].
\]
This shows that \( Y \) is uniformly distributed over \( [0, 1] \).\\

Given that \( Y = y \), the random variable \( Z \) follows a Geometric distribution with parameter \( y \). The probability mass function (PMF) of a Geometric random variable is given by:
\[
P(Z = z \mid Y = y) = (1 - y)^{z-1} y \quad \text{for } z \in \mathbb{N}, z \geq 1.
\]

To find the unconditional PMF of \( Z \), we need to use the law of total probability:
\[
P(Z = z) = \int_0^1 P(Z = z \mid Y = y) f_Y(y) \, dy.
\]
Since \( Y \) is uniformly distributed over \( [0, 1] \), we have \( f_Y(y) = 1 \) for \( y \in [0, 1] \). Thus:
\[
P(Z = z) = \int_0^1 (1 - y)^{z-1} y \, dy.
\]

This integral can be computed using integration by parts or Beta function properties:
\[
\int_0^1 (1 - y)^{z-1} y \, dy = \frac{1}{(z+1)(z+2)}.
\]
Therefore, the unconditional PMF of \( Z \) is:
\[
P(Z = z) = \frac{1}{(z+1)(z+2)} \quad \text{for } z \in \mathbb{N}, z \geq 1.
\]

Next, we find the conditional PMF of \( Y \) given \( Z = z \):
\[
P(Y = y \mid Z = z) = \frac{P(Z = z \mid Y = y) P(Y = y)}{P(Z = z)}.
\]
Substituting the known values:
\[
P(Y = y \mid Z = z) = \frac{(1 - y)^{z-1} y \cdot 1}{P(Z = z)}.
\]
This gives us the conditional PMF of \( Y \) given \( Z = z \).

\end{solution}

\begin{exercise}
Let \( X \) be a continuous random variable with the pdf
\[
f_X(x) = 
\begin{cases} 
e^{-x} & x \geq 0 \\ 
0 & x < 0 
\end{cases}
\]
Find the transformation \( Y = g(X) \) such that the pdf of \( Y \) will be
\[
f_Y(y) = 
\begin{cases} 
\frac{1}{2} \sqrt{y} & 0 < y < 1 \\ 
0 & \text{otherwise} 
\end{cases}
\]
\end{exercise}


\begin{solution}
To find the transformation \( Y = g(X) \) such that the pdf of \( Y \) matches the given pdf \( f_Y(y) \), we will first find the cumulative distribution function (CDF) of \( X \).\\

The CDF of \( X \) is given by:
\[
F_X(x) = \int_{0}^{x} f_X(t) \, dt = \int_{0}^{x} e^{-t} \, dt = 1 - e^{-x}, \quad x \geq 0.
\]
Thus, the CDF can be expressed as:
\[
F_X(x) = 
\begin{cases} 
0 & x < 0 \\ 
1 - e^{-x} & x \geq 0 
\end{cases}
\]

Now, to find \( g(X) \), we can utilize the relationship between the CDF of \( Y \) and \( X \):
\[
F_Y(y) = P(Y \leq y) = P(g(X) \leq y) = P(X \leq g^{-1}(y)).
\]

Given \( f_Y(y) \), we can find \( F_Y(y) \):
\[
F_Y(y) = \int_{0}^{y} f_Y(t) \, dt = \int_{0}^{y} \frac{1}{2} \sqrt{t} \, dt = \frac{1}{2} \cdot \frac{2}{3} t^{3/2} \Big|_{0}^{y} = \frac{1}{3} y^{3/2}, \quad 0 < y < 1.
\]

Setting this equal to \( F_X(g^{-1}(y)) \):
\[
F_X(g^{-1}(y)) = 1 - e^{-g^{-1}(y)}.
\]
Thus, we need:
\[
1 - e^{-g^{-1}(y)} = \frac{1}{3} y^{3/2}.
\]

Rearranging gives:
\[
e^{-g^{-1}(y)} = 1 - \frac{1}{3} y^{3/2}.
\]
Taking the natural logarithm:
\[
-g^{-1}(y) = \ln\left(1 - \frac{1}{3} y^{3/2}\right),
\]
which implies:
\[
g^{-1}(y) = -\ln\left(1 - \frac{1}{3} y^{3/2}\right).
\]

Therefore, we can express \( g(X) \) as:
\[
Y = g(X) = -\ln\left(1 - \frac{1}{3} X^{3/2}\right).
\]
Thus, the transformation \( Y = g(X) \) that yields the desired pdf \( f_Y(y) \) is:
\[
Y = g(X) = -\ln\left(1 - \frac{1}{3} X^{3/2}\right).
\]
\end{solution}

\begin{exercise}
Suppose \( X \) and \( Y \) are independent Gaussian random variables with zero mean and variance \( \sigma^2 \). Show that the ratio \( \frac{X}{Y} \) follows a Cauchy distribution.
\end{exercise}

\begin{solution}
Let \( X \sim \mathcal{N}(0, \sigma^2) \) and \( Y \sim \mathcal{N}(0, \sigma^2) \) be independent Gaussian random variables with mean zero and variance \( \sigma^2 \). We aim to show that the ratio \( Z = \frac{X}{Y} \) has a Cauchy distribution.\\

Since \( X \) and \( Y \) are independent normal random variables, their joint probability density function (PDF) is given by:
\[
f_{X,Y}(x, y) = f_X(x) f_Y(y) = \frac{1}{2 \pi \sigma^2} \exp \left( -\frac{x^2 + y^2}{2 \sigma^2} \right).
\]

We introduce the transformation:
\[
Z = \frac{X}{Y} \quad \text{and} \quad W = Y.
\]
This implies \( X = ZW \) and \( Y = W \).\\

To find the joint distribution of \( Z \) and \( W \), we need the Jacobian determinant of this transformation:
\[
\frac{\partial (X, Y)}{\partial (Z, W)} = 
\begin{vmatrix}
\frac{\partial X}{\partial Z} & \frac{\partial X}{\partial W} \\
\frac{\partial Y}{\partial Z} & \frac{\partial Y}{\partial W}
\end{vmatrix} = 
\begin{vmatrix}
W & Z \\
0 & 1
\end{vmatrix} = W.
\]
Thus, the absolute value of the Jacobian determinant is \( |W| \).\\

Using the change of variables, the joint PDF of \( Z \) and \( W \) is:
\[
f_{Z, W}(z, w) = f_{X,Y}(z w, w) \cdot |w| = \frac{1}{2 \pi \sigma^2} \exp \left( -\frac{z^2 w^2 + w^2}{2 \sigma^2} \right) \cdot |w|.
\]
Simplifying the exponent:
\[
f_{Z, W}(z, w) = \frac{|w|}{2 \pi \sigma^2} \exp \left( -\frac{w^2 (z^2 + 1)}{2 \sigma^2} \right).
\]

To obtain the marginal PDF of \( Z \), integrate out \( w \):
\[
f_Z(z) = \int_{-\infty}^{\infty} f_{Z, W}(z, w) \, dw = \int_{-\infty}^{\infty} \frac{|w|}{2 \pi \sigma^2} \exp \left( -\frac{w^2 (z^2 + 1)}{2 \sigma^2} \right) \, dw.
\]
This integral simplifies by using a standard Gaussian integral:
\[
f_Z(z) = \frac{1}{\pi} \frac{\sigma}{\sigma (z^2 + 1)} = \frac{1}{\pi} \frac{1}{z^2 + 1}.
\]

This is the PDF of Cauchy distribution. 
\end{solution}

\begin{exercise}
Particles are subject to collisions that cause them to split into two parts, with each part being a fraction of the parent. Suppose that this fraction is uniformly distributed between \(0\) and \(1\). Following a single particle through several splittings, we obtain a fraction of the original particle \( Z_n = X_1 X_2 \dots X_n \), where each \( X_j \) is uniformly distributed between \(0\) and \(1\). Show that the density for the random variable \( Z_n \) is given by:
\[
f_n(z) = \frac{1}{(n - 1)!}(-\log(z))^{n - 1}
\]
\end{exercise}

\begin{solution}
Since \( Z_n \) is the product of \( n \) independent uniform random variables, we can express \( Z_n \) as:
   \[
   Z_n = X_1 X_2 \dots X_n.
   \]

Take the natural logarithm of \( Z_n \):
   \[
   \log(Z_n) = \log(X_1) + \log(X_2) + \dots + \log(X_n).
   \]
   Each \( \log(X_j) \) is independently and identically distributed. Since \( X_j \sim \text{Uniform}(0, 1) \), the distribution of \( \log(X_j) \) is exponential with parameter \( \lambda = 1 \), and thus has mean \( -1 \) and density:
   \[
   f_{\log(X_j)}(x) = e^x \quad \text{for } x < 0.
   \]

The sum of \( n \) independent exponential random variables with rate \( \lambda = 1 \) follows a gamma distribution with shape parameter \( n \) and rate parameter \( 1 \). Therefore, \( S = -\log(Z_n) \) has the density:
   \[
   f_S(s) = \frac{s^{n-1} e^{-s}}{(n-1)!} \quad \text{for } s \geq 0.
   \]

To obtain the density function \( f_{Z_n}(z) \), we perform a change of variables from \( S \) to \( Z_n \) using \( z = e^{-s} \) or equivalently \( s = -\log(z) \). The Jacobian of this transformation is:
   \[
   \left| \frac{ds}{dz} \right| = \frac{1}{z}.
   \]

Substituting, we get:
   \[
   f_{Z_n}(z) = f_S(-\log(z)) \cdot \left| \frac{ds}{dz} \right| = \frac{(-\log(z))^{n-1} e^{\log(z)}}{(n-1)!} \cdot \frac{1}{z}.
   \]
   Simplifying, we obtain:
   \[
   f_{Z_n}(z) = \frac{(-\log(z))^{n-1}}{(n-1)!} \quad \text{for } 0 < z < 1.
   \]
Thus, we have shown that the density function for \( Z_n \) is:
\[
f_n(z) = \frac{1}{(n - 1)!}(-\log(z))^{n - 1}.
\]
\end{solution}

\begin{exercise}
Suppose \( X \) and \( Y \) are independent exponential random variables with the same parameter \( \lambda \). Derive the probability density function (pdf) of the random variable 
\[
Z = \frac{\min(X,Y)}{\max(X,Y)}.
\]
\end{exercise}

\begin{solution}
Since \( X \) and \( Y \) are independent exponential random variables with parameter \( \lambda \), their pdfs are given by:
\[
f_X(x) = \lambda e^{-\lambda x} \quad \text{and} \quad f_Y(y) = \lambda e^{-\lambda y}, \quad x, y \geq 0.
\]

Define \( Z = \frac{\min(X, Y)}{\max(X, Y)} \). To find the pdf of \( Z \), we analyze the probability \( P(Z \leq z) \) for \( 0 \leq z \leq 1 \).\\

   Without loss of generality, suppose \( X \leq Y \) (the case \( Y \leq X \) will be symmetric). Then:
   \[
   Z = \frac{X}{Y}.
   \]
   
   Now, we need to calculate the probability \( P\left( \frac{X}{Y} \leq z \right) \) under this assumption.

   \[
   P\left( Z \leq z \right) = P\left( \frac{X}{Y} \leq z \right) = P\left( X \leq zY \right).
   \]
   
   Using the independence of \( X \) and \( Y \), we can express this probability as:
   \[
   P(X \leq zY) = \int_0^{\infty} P(X \leq zy) f_Y(y) \, dy.
   \]
   
   Since \( X \sim \text{Exp}(\lambda) \), we have \( P(X \leq zy) = 1 - e^{-\lambda zy} \). Thus,
   \[
   P(Z \leq z) = \int_0^{\infty} \left(1 - e^{-\lambda zy}\right) \lambda e^{-\lambda y} \, dy.
   \]
   
   Splitting the integral, we get:
   \[
   P(Z \leq z) = \int_0^{\infty} \lambda e^{-\lambda y} \, dy - \int_0^{\infty} \lambda e^{-\lambda(1+z)y} \, dy.
   \]
   
   Evaluating each term:
   \[
   \int_0^{\infty} \lambda e^{-\lambda y} \, dy = 1,
   \]
   and
   \[
   \int_0^{\infty} \lambda e^{-\lambda(1+z)y} \, dy = \frac{\lambda}{\lambda(1+z)} = \frac{1}{1+z}.
   \]
   
   Therefore,
   \[
   P(Z \leq z) = 1 - \frac{1}{1+z} = \frac{z}{1+z}.
   \]

   To obtain the pdf \( f_Z(z) \), differentiate \( P(Z \leq z) \) with respect to \( z \):
   \[
   f_Z(z) = \frac{d}{dz} \left( \frac{z}{1+z} \right) = \frac{1}{(1+z)^2}.
   \]

Thus, the pdf of \( Z = \frac{\min(X, Y)}{\max(X, Y)} \) is
\[
f_Z(z) = \frac{1}{(1+z)^2}, \quad 0 \leq z \leq 1.
\]
\end{solution}

\begin{exercise}
A random variable \( Y \) has the pdf \( f_Y(y) = K y^{-(b+1)} \), \( y \geq 2 \) (and zero otherwise), where \( b > 0 \). This random variable is obtained as the monotonically increasing transformation \( Y = g(X) \) of the random variable \( X \) with pdf \( e^{-x} \), \( x \geq 0 \).\\

(a) Determine \( K \) in terms of \( b \).\\
(b) Determine the transformation \( g(\cdot) \) in terms of \( b \).
\end{exercise}

\begin{solution}

We know that the total area under the probability density function \( f_Y(y) \) must equal 1. Therefore, we can write:

\[
\int_{2}^{\infty} f_Y(y) \, dy = 1.
\]

Substituting the expression for \( f_Y(y) \):

\[
\int_{2}^{\infty} K y^{-(b+1)} \, dy = 1.
\]

\[
K \int_{2}^{\infty} y^{-(b+1)} \, dy = K \left[ \frac{y^{-b}}{-b} \right]_{2}^{\infty} = K \left( 0 + \frac{2^{-b}}{b} \right) = \frac{K}{b \cdot 2^b}.
\]

Setting this equal to 1 gives us:

\[
\frac{K}{b \cdot 2^b} = 1.
\]

\[
K = b \cdot 2^b.
\]

\vspace{10pt}
Next, we will find the transformation \( g(\cdot) \). We know that \( Y = g(X) \) is a monotonically increasing transformation of \( X \), which has the pdf:

\[
f_X(x) = e^{-x}, \quad x \geq 0.
\]

The cumulative distribution function (CDF) of \( X \) is:

\[
F_X(x) = \int_{0}^{x} e^{-t} \, dt = 1 - e^{-x}.
\]

To relate \( Y \) and \( X \), we use the fact that \( F_Y(y) = P(Y \leq y) = P(g(X) \leq y) = P(X \leq g^{-1}(y)) \). Therefore, we can express this in terms of the CDF of \( X \):

\[
F_Y(y) = F_X(g^{-1}(y)) = 1 - e^{-g^{-1}(y)}.
\]

Since we have \( f_Y(y) = K y^{-(b+1)} \), we can differentiate to find the CDF:

\[
F_Y(y) = \int_{2}^{y} K t^{-(b+1)} \, dt = K \left[ \frac{t^{-b}}{-b} \right]_{2}^{y} = K \left( \frac{2^{-b}}{b} - \frac{y^{-b}}{b} \right).
\]

Substituting \( K \) from part (a):

\[
F_Y(y) = \frac{b \cdot 2^b}{b} \left( 2^{-b} - y^{-b} \right) = 2^b \left( 2^{-b} - y^{-b} \right) = 1 - \frac{2^b}{y^b}.
\]

Setting this equal to \( 1 - e^{-g^{-1}(y)} \):

\[
1 - e^{-g^{-1}(y)} = 1 - \frac{2^b}{y^b}.
\]

Thus, we have:

\[
e^{-g^{-1}(y)} = \frac{2^b}{y^b}.
\]

Taking the natural logarithm of both sides:

\[
-g^{-1}(y) = \ln\left(\frac{2^b}{y^b}\right).
\]

Hence,

\[
g^{-1}(y) = -b \ln\left(\frac{2}{y}\right).
\]

Finally, inverting the function gives us the transformation:

\[
g(x) = 2 e^{-\frac{x}{b}}.
\]
\end{solution}

\begin{exercise}
Two particles start from the same point on a two-dimensional plane and move with speed \( V \) each, such that the angle between them is uniformly distributed in \([0, 2\pi]\). Find the distribution of the magnitude of the relative velocity between the two particles.
\end{exercise}

\begin{solution}
Let the velocities of the two particles be represented as vectors. Let \(\vec{v_1}\) and \(\vec{v_2}\) be the velocity vectors of the first and second particle, respectively. We can express these velocity vectors in terms of their magnitudes and the angle \(\theta\) between them:

\[
\vec{v_1} = V \begin{pmatrix} \cos(\phi_1) \\ \sin(\phi_1) \end{pmatrix}, \quad \vec{v_2} = V \begin{pmatrix} \cos(\phi_2) \\ \sin(\phi_2) \end{pmatrix}
\]

where \(\phi_1\) and \(\phi_2\) are the angles of the velocities of the two particles. The angle between the two velocities is given by:

\[
\theta = \phi_2 - \phi_1
\]

The relative velocity \(\vec{v_{rel}}\) of particle 2 with respect to particle 1 is given by:

\[
\vec{v_{rel}} = \vec{v_2} - \vec{v_1} = V \begin{pmatrix} \cos(\phi_2) - \cos(\phi_1) \\ \sin(\phi_2) - \sin(\phi_1) \end{pmatrix}
\]

The magnitude of the relative velocity \( |\vec{v_{rel}}| \) is:

\[
|\vec{v_{rel}}| = V \sqrt{(\cos(\phi_2) - \cos(\phi_1))^2 + (\sin(\phi_2) - \sin(\phi_1))^2}
\]

Using the trigonometric identity for the cosine of the difference of two angles, we can rewrite the expression as follows:

\[
|\vec{v_{rel}}| = V \sqrt{2 - 2\cos(\theta)} = V \sqrt{2(1 - \cos(\theta))} = V \sqrt{2} \sqrt{1 - \cos(\theta)} = V \sqrt{2} \sin\left(\frac{\theta}{2}\right)
\]

Since \(\theta\) is uniformly distributed in \([0, 2\pi]\), the probability density function (pdf) of \(\theta\) is:

\[
f_{\Theta}(\theta) = \frac{1}{2\pi}, \quad \text{for } 0 \leq \theta < 2\pi
\]

To find the distribution of the magnitude of the relative velocity \(R = |\vec{v_{rel}}|\), we can use the transformation method. We know:

\[
R = V \sqrt{2} \sin\left(\frac{\theta}{2}\right)
\]

To find the cumulative distribution function (CDF) of \(R\), we can find \(P(R \leq r)\):

\[
P(R \leq r) = P\left(V \sqrt{2} \sin\left(\frac{\theta}{2}\right) \leq r\right) = P\left(\sin\left(\frac{\theta}{2}\right) \leq \frac{r}{V \sqrt{2}}\right)
\]

Let \(x = \frac{r}{V \sqrt{2}}\). The sine function maps the interval \([0, 2\pi]\) to the interval \([0, 1]\), and hence:

\[
P\left(\sin\left(\frac{\theta}{2}\right) \leq x\right) = P\left(\frac{\theta}{2} \leq \arcsin(x)\right) = P\left(\theta \leq 2\arcsin(x)\right)
\]

The corresponding angles must be within the bounds of \(\theta\):

\[
P\left(\theta \leq 2\arcsin(x)\right) = \frac{2\arcsin(x)}{2\pi} = \frac{\arcsin(x)}{\pi}
\]

Thus, the cumulative distribution function of \(R\) is:

\[
F_R(r) = P(R \leq r) = \frac{1}{\pi} \arcsin\left(\frac{r}{V \sqrt{2}}\right)
\]

Differentiating \(F_R(r)\) gives us the probability density function \(f_R(r)\):

\[
f_R(r) = \frac{d}{dr} F_R(r) = \frac{1}{\pi} \cdot \frac{1}{\sqrt{1 - \left(\frac{r}{V \sqrt{2}}\right)^2}} \cdot \frac{1}{V \sqrt{2}}
\]

This simplifies to:

\[
f_R(r) = \frac{1}{\pi V \sqrt{2}} \cdot \frac{1}{\sqrt{1 - \left(\frac{r}{V \sqrt{2}}\right)^2}}, \quad \text{for } 0 < r < V \sqrt{2}
\]

Thus, the distribution of the magnitude of the relative velocity between the two particles is given by:

\[
f_R(r) = \frac{1}{\pi V \sqrt{2}} \cdot \frac{1}{\sqrt{1 - \left(\frac{r}{V \sqrt{2}}\right)^2}}, \quad 0 < r < V \sqrt{2}
\]
\end{solution}

\begin{exercise}
A point is picked uniformly from inside a unit circle. What is the density of \( R \), the distance of the point from the center?
\end{exercise}

\begin{solution}
To find the density of \( R \), the distance from the center of the unit circle, we start by noting that the area of a circle is given by the formula \( A = \pi r^2 \), where \( r \) is the radius of the circle. In our case, the radius is 1, so the area of the unit circle is \( \pi \).\\

When we choose a point uniformly inside the circle, the probability density function (pdf) must be proportional to the area element in polar coordinates. In polar coordinates, a point in the circle can be described by coordinates \( (r, \theta) \), where \( r \) is the distance from the center (the value of \( R \)) and \( \theta \) is the angle.\\

The area element in polar coordinates is given by:

\[
dA = r \, dr \, d\theta
\]

To find the density function of \( R \), we need to consider how the area is distributed with respect to \( R \). The cumulative distribution function (CDF) for \( R \) can be expressed as the probability that the distance \( R \) is less than or equal to some value \( r \):

\[
P(R \leq r) = \text{Area of the circle with radius } r = \pi r^2
\]

Since \( R \) is uniformly distributed over the unit circle, the total area of the unit circle is \( \pi \). Therefore, the CDF can be normalized:

\[
P(R \leq r) = \frac{\pi r^2}{\pi} = r^2 \quad \text{for } 0 \leq r \leq 1
\]

To find the probability density function (pdf), we take the derivative of the CDF with respect to \( r \):

\[
f_R(r) = \frac{d}{dr} P(R \leq r) = \frac{d}{dr} (r^2) = 2r \quad \text{for } 0 \leq r \leq 1
\]

Thus, the density of \( R \) is given by:

\[
f_R(r) = 2r \quad \text{for } 0 \leq r \leq 1
\]
\end{solution}


\begin{exercise}
Let \( X \) and \( Y \) be independent exponentially distributed random variables with parameter \( 1 \). Find the joint density of \( U = X + Y \) and \( V = \frac{X}{X+Y} \), and show that \( V \) is uniformly distributed.
\end{exercise}

\begin{solution}
To solve this problem, we will first find the joint distribution of \( U \) and \( V \) and then show that \( V \) is uniformly distributed.\\

The random variables \( X \) and \( Y \) are independent and exponentially distributed with parameter \( 1 \), so their probability density functions (pdf) are given by:
\[
f_X(x) = e^{-x} \quad \text{for } x \geq 0,
\]
\[
f_Y(y) = e^{-y} \quad \text{for } y \geq 0.
\]


Since \( X \) and \( Y \) are independent, the joint pdf is:
\[
f_{X,Y}(x,y) = f_X(x) f_Y(y) = e^{-x} e^{-y} = e^{-(x+y)} \quad \text{for } x \geq 0, y \geq 0.
\]

Now we define the transformation:
\[
U = X + Y,
\]
\[
V = \frac{X}{X + Y} = \frac{X}{U}.
\]

The inverse transformation is:
\[
X = UV, \quad Y = U(1 - V).
\]

Next, we calculate the Jacobian of the transformation from \( (X, Y) \) to \( (U, V) \):
\[
\begin{bmatrix}
\frac{\partial X}{\partial U} & \frac{\partial X}{\partial V} \\
\frac{\partial Y}{\partial U} & \frac{\partial Y}{\partial V}
\end{bmatrix}
=
\begin{bmatrix}
V & U \\
1 - V & -U
\end{bmatrix}.
\]
The determinant of this Jacobian matrix is:
\[
J = \left| \begin{array}{cc}
V & U \\
1 - V & -U
\end{array} \right| = -UV + U(1 - V) = U.
\]
Taking the absolute value gives us \( |J| = U \).\\

Using the change of variables formula, we have:
\[
f_{U,V}(u,v) = f_{X,Y}(x,y) \cdot |J| = f_{X,Y}(uv, u(1-v)) \cdot |J|.
\]
Substituting for \( f_{X,Y}(x,y) \):
\[
f_{U,V}(u,v) = e^{-(uv + u(1-v))} \cdot u = e^{-u} \cdot u = u e^{-u} \quad \text{for } u \geq 0, \; 0 \leq v \leq 1.
\]

To find the marginal distribution of \( V \), we integrate out \( U \):
\[
f_V(v) = \int_0^\infty f_{U,V}(u,v) \, du = \int_0^\infty u e^{-u} \, du.
\]
This integral is recognized as the Laplace transform of \( u \) with \( s = 1 \):
\[
\int_0^\infty u e^{-u} \, du = 1 \quad \text{(by integration by parts or gamma function)}.
\]

The limits on \( v \) are \( 0 \leq v \leq 1 \), and since we integrate over all possible values of \( U \) and find that \( f_V(v) \) integrates to \( 1 \) over this range, it follows that \( V \) has a uniform distribution.
\end{solution}


\vspace{30pt}
\hrule