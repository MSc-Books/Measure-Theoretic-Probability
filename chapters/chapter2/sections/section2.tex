\section{Introduction to Lebesgue Measure}

What we understood from the example of \textit{Vitali Set} in the last section is that we cannot assign it measure like \textit{length}. We will only assign measures to the \textit{Borel sets}. This gives us the understanding that the entire collection of subsets $2^\Omega$, where $\Omega = (0, 1]$, is not measureable. So, we can say $(\Omega, \mathcal{B})$ is the \textit{measurable space}. Now, we want to assign each Borel set a measure. \\

Consider \( \Omega = (0, 1] \). Let \( \mathcal{F}_0 \) consist of the empty set and all sets that are finite unions of intervals of the form \( (a, b] \). A typical element of this set is of the form
\[
F = (a_1, b_1] \cup (a_2, b_2] \cup \dots \cup (a_n, b_n]
\]
where \( 0 \leq a_1 < b_1 \leq a_2 < b_2 \leq \dots \leq a_n < b_n \) and \( n \in \mathbb{N} \).

\begin{lemma}
    \begin{itemize}
        \item[(a)] \( \mathcal{F}_0 \) is an algebra.
        \item[(b)] \( \mathcal{F}_0 \) is not a \( \sigma \)-algebra.
        \item[(c)] \( \sigma(\mathcal{F}_0) = \mathcal{B} \), where \( \mathcal{B} \) denotes the Borel \( \sigma \)-algebra.
    \end{itemize}
\end{lemma}

\begin{proof}
    \begin{itemize}
        \item[(a)] By definition, \( \emptyset \in \mathcal{F}_0 \). Also, \( \emptyset^C = (0, 1] \in \mathcal{F}_0 \). The complement of \( (a_1, b_1] \cup (a_2, b_2] \) is \( (0, a_1] \cup (b_1, a_2] \cup (b_2, 1] \), which also belongs to \( \mathcal{F}_0 \). \\ Furthermore, the union of finitely many sets, each of which is a finite union of intervals of the form \( (a, b] \), is also a set that is a union of a finite number of intervals, and thus belongs to \( \mathcal{F}_0 \).
        
        \item[(b)] To see this, note that \( \left(0, \frac{n}{n+1}\right] \in \mathcal{F}_0 \) for every \( n \), but \( \bigcup_{n=1}^{\infty} \left(0, \frac{n}{n+1}\right] = (0, 1) \notin \mathcal{F}_0 \). This shows that \( \mathcal{F}_0 \) is not closed under countable unions, and thus it is not a \( \sigma \)-algebra.
    
        \item[(c)] First, the null set is clearly a Borel set. Next, we have already seen that every interval of the form \( (a, b] \) is a Borel set. Hence, every element of \( \mathcal{F}_0 \) (other than the null set), which is a finite union of such intervals, is also a Borel set. Therefore, \( \mathcal{F}_0 \subseteq \mathcal{B} \). This implies \( \sigma(\mathcal{F}_0) \subseteq \mathcal{B} \).
    
        Next, we show that \( \mathcal{B} \subseteq \sigma(\mathcal{F}_0) \). For any interval of the form \( (a, b) \) in \( \mathcal{C}_0 \), we can write
        \[
        (a, b) = \bigcup_{n=1}^{\infty} \left(a, b - \frac{1}{n}\right) \cap \Omega.
        \]
        Since every interval of the form \( \left(a, b - \frac{1}{n}\right) \in \mathcal{F}_0 \), a countable union of such intervals belongs to \( \sigma(\mathcal{F}_0) \). Therefore, \( (a, b) \in \sigma(\mathcal{F}_0) \) and consequently, \( \mathcal{C}_0 \subseteq \sigma(\mathcal{F}_0) \). This gives \( \sigma(\mathcal{C}_0) \subseteq \sigma(\mathcal{F}_0) \). Using the fact that \( \sigma(\mathcal{C}_0) = \mathcal{B} \) proves the required result.
    \end{itemize}
    
\end{proof}


Now, recall that we wanted to give subset $(a, b)$ a measure that is proportional to $b-a$. While this makes sense for the intervals, it doesn't make sense for singleton sets and complex sets like \textit{Cantor set}. What we want to do now is - extend the idea of this measure to other Borel sets. This is achieved by using \textbf{Caratheodory's Extension Theorem}.

\begin{theorem}
    Let \( \mathcal{F}_0 \) be an algebra of subsets of \( \Omega \), and let \( \mathcal{F} = \sigma(\mathcal{F}_0) \) be the \( \sigma \)-algebra that it generates. Suppose that \( P_0 \) is a mapping from \( \mathcal{F}_0 \) to \([0, 1]\) that satisfies:
    \begin{enumerate}
        \item \( P_0(\Omega) = 1 \)
        \item \( P_0 \) is countably additive on \( \mathcal{F}_0 \). 
    \end{enumerate}
    Then, \( P_0 \) can be extended uniquely to a probability measure on \( (\Omega, \mathcal{F}) \). That is, there exists a unique probability measure \( P \) on \( (\Omega, \mathcal{F}) \) such that \( P(A) = P_0(A) \) for all \( A \in \mathcal{F}_0 \).
    \end{theorem}
    
    \begin{proof}
    We proceed in several steps to establish the existence and uniqueness of the extension.\\
    
    \textbf{Step 1: Construction of an Outer Measure}\\
    
    Define an outer measure \( \mu \) on the power set \( \mathcal{P}(\Omega) \) as follows:
    \[
    \mu(A) = \inf\left\{\sum_{n=1}^{\infty} P_0(E_n) : A \subseteq \bigcup_{n=1}^{\infty} E_n, \; E_n \in \mathcal{F}_0\right\}.
    \]
    This definition uses the idea of covering \( A \) with a countable union of sets from \( \mathcal{F}_0 \). The sum of the measures of these covering sets provides an upper bound for \( \mu(A) \). \\
    
    The infimum ensures that we take the smallest possible value, making \( \mu \) as small as possible while still covering \( A \).\\
    
    To show that \( \mu \) is indeed an outer measure, we verify three properties:
    \begin{itemize}
        \item \( \mu(\emptyset) = 0 \) by definition.
        \item \( \mu \) is monotone: if \( A \subseteq B \), then \( \mu(A) \leq \mu(B) \).
        \item \( \mu \) satisfies countable subadditivity: for any sequence of sets \( \{A_n\}_{n=1}^{\infty} \), \( \mu\left(\bigcup_{n=1}^{\infty} A_n\right) \leq \sum_{n=1}^{\infty} \mu(A_n) \).
    \end{itemize}
    
    \textbf{Step 2: Countable Additivity and Carathéodory's Extension Theorem}\\
    
    Since \( P_0 \) is countably additive on \( \mathcal{F}_0 \), it follows that \( \mu \) is countably additive on \( \mathcal{F}_0 \). Specifically, for any sequence of disjoint sets \( \{A_n\}_{n=1}^{\infty} \subseteq \mathcal{F}_0 \), we have:
    \[
    \mu\left(\bigcup_{n=1}^{\infty} A_n\right) = \sum_{n=1}^{\infty} \mu(A_n) = \sum_{n=1}^{\infty} P_0(A_n).
    \]
    This equality holds because the definition of \( \mu \) coincides with \( P_0 \) on \( \mathcal{F}_0 \). \\
    
    By Carathéodory's Extension Theorem, if an outer measure \( \mu \) is countably additive on a collection of sets (here, \( \mathcal{F}_0 \)), then \( \mu \) can be extended to a measure \( P \) on the \( \sigma \)-algebra generated by those sets. Thus, there exists a unique measure \( P \) on \( \mathcal{F} \) such that \( P(A) = P_0(A) \) for all \( A \in \mathcal{F}_0 \).\\
    
    \textbf{Step 3: Verification of the Extension on \( \mathcal{F} \)}\\
    
    Since \( \mathcal{F} \) is the \( \sigma \)-algebra generated by \( \mathcal{F}_0 \), every set in \( \mathcal{F} \) can be expressed through countable unions, intersections, and complements of sets in \( \mathcal{F}_0 \). The measure \( P \) extends \( P_0 \) while preserving countable additivity. Therefore, for any \( A \in \mathcal{F}_0 \), we have:
    \[
    P(A) = \mu(A) = P_0(A).
    \]
    
    \textbf{Step 4: Uniqueness of the Extension}\\
    
    Suppose there exists another probability measure \( Q \) on \( (\Omega, \mathcal{F}) \) that agrees with \( P_0 \) on \( \mathcal{F}_0 \). Let \( A \in \mathcal{F} \). We can approximate \( A \) using sets from \( \mathcal{F}_0 \). Given that both \( P \) and \( Q \) agree with \( P_0 \) on \( \mathcal{F}_0 \), for any such approximation, the measures \( P \) and \( Q \) must produce the same value. Therefore:
    \[
    P(A) \leq Q(A) \quad \text{and} \quad Q(A) \leq P(A).
    \]
    This implies \( P(A) = Q(A) \). Since \( A \) was arbitrary in \( \mathcal{F} \), \( P \) and \( Q \) must be the same measure on \( \mathcal{F} \).\\
    
    Thus, \( P \) is the unique probability measure extending \( P_0 \) to \( \mathcal{F} \).
    
    \end{proof}

    For every $F \in F_0$ of the form
    \[
    F = (a_1, b_1] \cup (a_2, b_2] \cup \dots \cup (a_n, b_n],
    \]
    we define a function $P_0 : F_0 \rightarrow [0, 1]$ such that $P_0 (\emptyset) = 0$ and 
    \[
    P_0 (F) = \sum_{i=1}^{n} (b_i - a_i).
    \]
    
    Note that $P_0 (\Omega) = P_0 ((0, 1]) = 1$. Also, if $(a_1, b_1], (a_2, b_2], \dots, (a_n, b_n]$ are disjoint sets, then
    \[
    P_0 \left(\bigcup_{i=1}^{n} (a_i, b_i]\right) = \sum_{i=1}^{n} P_0 ((a_i, b_i]) = \sum_{i=1}^{n} (b_i - a_i),
    \]
    implying the finite additivity of $P_0$. It turns out that $P_0$ is countably additive on $F_0$ as well; that is, if $(a_1, b_1], (a_2, b_2], \dots$ are disjoint sets such that $\bigcup_{i=1}^{\infty} (a_i, b_i] \in F_0$, then
    \[
    P_0 \left(\bigcup_{i=1}^{\infty} (a_i, b_i]\right) = \sum_{i=1}^{\infty} P_0 ((a_i, b_i]) = \sum_{i=1}^{\infty} (b_i - a_i).
    \]
    
    From Carathéodory's extension theorem, there exists a unique probability measure $P$ on $((0, 1], \mathcal{B})$ which is the same as $P_0$ on $F_0$. This unique probability measure on $(0, 1]$ is called the \textit{Lebesgue} or \textit{uniform measure}.\\

    The Lebesgue measure formalizes the notion of length. Specifically, it extends the intuitive idea of length of intervals to a broader set of subsets of $\mathbb{R}$, including sets that are not necessarily intervals. The Lebesgue measure assigns to each set a non-negative value that represents its \textit{size} in terms of length.\\

This suggests that the Lebesgue measure of a singleton should be zero. To demonstrate this, let $b \in (0, 1]$. Using the definition of the measure, we write
\[
P(\{b\}) = P\left(\bigcap_{n=1}^{\infty} \left(b - \frac{1}{n}, b\right] \cap \Omega\right).
\]
Let $A_n = \left(b - \frac{1}{n}, b\right]$. For each $n$, the Lebesgue measure of $A_n$ is
\[
P(A_n) = \frac{1}{n}.
\]
Since $\{A_n\}$ is a decreasing sequence of nested sets,
\[
P(\{b\}) = P\left(\bigcap_{n=1}^{\infty} A_n\right) = \lim_{n \to \infty} P(A_n) = \lim_{n \to \infty} \frac{1}{n} = 0,
\]
where the second equality follows from the continuity of probability measures.\\

Since any countable set is a countable union of singletons, the probability of a countable set is zero.\\

For example, under the uniform measure on $(0, 1]$, the probability of the set of rationals is zero, since the rational numbers in $(0, 1]$ form a countable set.\\

For $\Omega = (0, 1]$, the Lebesgue measure is also a probability measure because $P((0, 1]) = 1$. However, for other intervals (for example, $\Omega = (0, 2]$), the Lebesgue measure is only a finite measure. In such cases, the measure can be normalized as appropriate to obtain a uniform probability measure. For instance, if $\Omega = (0, 2]$, the Lebesgue measure of this interval is $2$. By dividing by $2$, we can create a uniform probability measure over $(0, 2]$.

