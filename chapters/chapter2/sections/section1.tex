\section{Introduction to Borel Sets}

Let's consider the case when the sample space is uncountable. 


\subsection{Uncountable Sample Space}

Consider the experiment of picking a real number at random from $\Omega = [0, 1]$, such that every number is “equally likely” to be picked. It is quite apparent that a simple strategy of assigning probabilities to singleton subsets of the sample space gets into difficulties quite quickly. Indeed:

\begin{enumerate}
    \item[(i)] If we assign some positive probability to each elementary outcome, then the probability of an event with infinitely many elements, such as $A = \left\{1, \frac{1}{2}, \frac{1}{3}, \ldots \right\}$, would become unbounded. This is because the sum of positive probabilities over an infinite set would diverge.

    \item[(ii)] If we assign zero probability to each elementary outcome, this alone would not be sufficient to determine the probability of an uncountable subset of $\Omega$, such as $\left[\frac{1}{2}, \frac{2}{3}\right]$. This is because probability measures are not additive over uncountable disjoint unions (of singletons in this case). Assigning zero probability to singletons does not directly imply how to handle intervals or other uncountable sets.
\end{enumerate}

Thus, we need a different approach to assign probabilities when the sample space is uncountable, such as $\Omega = [0, 1]$. In particular, we need to assign probabilities directly to specific subsets of $\Omega$. Intuitively, we would like our ‘uniform measure’ $\mu$ on $[0, 1]$ to possess the following two properties:

\begin{enumerate}
    \item[(i)] $\mu((a, b)) = \mu((a, b]) = \mu([a, b)) = \mu([a, b])$ for any interval in $[0, 1]$. This ensures that the measure is consistent across different types of intervals, capturing the idea of “equal likelihood” for any interval of the same length.
    
    \item[(ii)] \textbf{Translational Invariance}: That is, if $A \subseteq [0, 1]$, then for any $x \in \Omega$, $\mu(A \oplus x) = \mu(A)$, where the set $A \oplus x$ is defined as:
    \[
    A \oplus x = \{a + x \mid a \in A, a + x \leq 1\} \cup \{a + x - 1 \mid a \in A, a + x > 1\}
    \]
    This property ensures that the measure is invariant under translation within the interval $[0, 1]$, reflecting the uniformity of the measure.
\end{enumerate}

However, the following impossibility result asserts that there is no way to consistently define a uniform measure on all subsets of $[0, 1]$. This result is rooted in the fact that certain sets in $[0, 1]$ (those that are non-measurable) defy any consistent assignment of measure while preserving the desired properties of translation invariance and interval consistency.

\begin{theorem}
    \textbf{Impossibility Result:} There does not exist a definition of a measure $\mu(A)$ for all subsets of $[0, 1]$ satisfying:

    \begin{itemize}
        \item[(i)] $\mu((a, b)) = \mu((a, b]) = \mu([a, b)) = \mu([a, b])$
        \item[(ii)] \textit{Translational Invariance:} If $A \subseteq [0, 1]$, then for any $x \in \Omega$, $\mu(A \oplus x) = \mu(A)$, where the set $A \oplus x$ is defined as:
        \[
        A \oplus x = \{a + x \mid a \in A, a + x \leq 1\} \cup \{a + x - 1 \mid a \in A, a + x > 1\}
        \]
    \end{itemize}
\end{theorem}

\begin{proof}

    To show the impossibility, we argue that a measure satisfying these two properties for all subsets of $[0, 1]$ leads to a contradiction. We will use basic properties of measures and simple logic to establish the proof.\\

\textbf{1. Interval Length Property (i):}  \\

   The first property tells us that the measure of any interval in $[0, 1]$ is simply the length of the interval. For example, if $A = (a, b)$, then $\mu((a, b)) = b - a$. This property holds for open, closed, and half-open intervals.\\

\textbf{2. Translational Invariance (ii):}  \\

   The second property states that if we shift a set $A$ by some amount $x$, its measure should remain the same. For example, if $A$ is an interval, shifting it within $[0, 1]$ should not change its length. This makes sense intuitively, as the measure should not depend on the location of the set but only its size.\\

\textbf{3. Partitioning $[0, 1]$ into Equal Parts:}  \\

   Let’s divide the interval $[0, 1]$ into $n$ equal parts. Define sets $A_i = \left[\frac{i-1}{n}, \frac{i}{n}\right)$ for $i = 1, 2, \ldots, n-1$ and $A_n = \left[\frac{n-1}{n}, 1\right]$. By property (i), each of these sets has a measure:
   \[
   \mu(A_i) = \frac{1}{n}, \quad \text{for all } i = 1, 2, \ldots, n.
   \]
   
   Since these intervals are disjoint and together cover $[0, 1]$, by the additivity of measures:
   \[
   \mu\left(\bigcup_{i=1}^{n} A_i\right) = \sum_{i=1}^{n} \mu(A_i) = \sum_{i=1}^{n} \frac{1}{n} = 1.
   \]

\textbf{4. Constructing Translations:}  \\

   Suppose we take one of these intervals, say $A_1 = \left[0, \frac{1}{n}\right)$, and shift it by $\frac{1}{2n}$. This new set, $A_1 \oplus \frac{1}{2n}$, becomes $\left[\frac{1}{2n}, \frac{3}{2n}\right)$, which is a valid subset of $[0, 1]$.\\

   By translational invariance (property (ii)), $\mu(A_1 \oplus \frac{1}{2n}) = \mu(A_1) = \frac{1}{n}$.\\

\textbf{5. Forming a Contradiction:}  \\

   Now, let’s consider translating $A_1$ by different multiples of $\frac{1}{2n}$. We form the following sets:
   \[
   A_1, \quad A_1 \oplus \frac{1}{2n}, \quad A_1 \oplus \frac{2}{2n}, \quad \ldots, \quad A_1 \oplus \frac{n-1}{2n}.
   \]
   
   These $n$ translations of $A_1$ are all disjoint, and by property (ii), each has a measure of $\frac{1}{n}$.\\

   However, if we sum up the measures of all these disjoint translations, we get:
   \[
   \mu(A_1) + \mu(A_1 \oplus \frac{1}{2n}) + \ldots + \mu(A_1 \oplus \frac{n-1}{2n}) = \frac{1}{n} \times n = 1.
   \]

\textbf{6. Contradiction with the Total Measure:}  \\

   Observe that the union of all these translated sets may not cover the entire interval $[0, 1]$. In fact, since $A_1$ is just a small fraction of $[0, 1]$, these translations form only a portion of $[0, 1]$. Hence, the measure of their union should be less than $1$.\\

   But by translational invariance and additivity, we have shown that the sum of their measures equals $1$. This creates a contradiction because it implies that a part of $[0, 1]$ has the same measure as the whole interval.\\

   Since this contradiction arises when attempting to define $\mu$ on all subsets while preserving both the interval property and translational invariance, it is impossible to define such a measure for all subsets of $[0, 1]$. 
\end{proof}

Therefore, we must compromise, and consider a smaller $\sigma$-algebra that contains certain “nice” subsets of the sample space [0, 1]. \textbf{These “nice” subsets are the intervals}, and the resulting $\sigma$-algebra is called the Borel $\sigma$-algebra. Before defining Borel sets, we introduce the concept of generating $\sigma$-algebras from a given collection of subsets.


\subsection{Borel Sets}

Now, we know that the collection of intervals is not a $\sigma$-algebra because if $[a, b]$ is in the collection, it's complement is not an interval. So, we want to build towards a $\sigma$-algebra that contains all intervals, their complements and is closed under countable unions and countable intersections. \\

Let $\mathcal{C}$ be the collection of all nice subsets of sample space $\Omega$ in which we are interested. We have to generate the smallest $\sigma$-algebra that contains $\mathcal{C}$, that is denoted by $\sigma(\mathcal{C})$. 

\begin{theorem}
   The intersection of an arbitrary number of $\sigma$-algebras is a $\sigma$-algebra.
\end{theorem}

\begin{proof}
   Let $\{ \mathcal{F}_i \}_{i \in I}$ be a collection of $\sigma$-algebras on a collection $\mathcal{C}$, where $I$ is an index set. Define $\mathcal{F} = \bigcap_{i \in I} \mathcal{F}_i$. We want to show that $\mathcal{F}$ is a $\sigma$-algebra.\\

   Since each $\mathcal{F}_i$ is a $\sigma$-algebra, it contains $\mathcal{C}$. Therefore, $\mathcal{C} \in \mathcal{F}_i$ for all $i \in I$. By definition of the intersection, $\mathcal{C} \in \mathcal{F}$.\\

   Let $A \in \mathcal{F}$. This implies that $A \in \mathcal{F}_i$ for every $i \in I$. Since each $\mathcal{F}_i$ is a $\sigma$-algebra, $A^c \in \mathcal{F}_i$ for all $i \in I$. Therefore, $A^c \in \bigcap_{i \in I} \mathcal{F}_i = \mathcal{F}$. Hence, $\mathcal{F}$ is closed under complementation.\\

   Let $\{A_n\}_{n=1}^{\infty}$ be a sequence of sets in $\mathcal{F}$. This implies that $A_n \in \mathcal{F}_i$ for every $i \in I$ and for all $n \in \mathbb{N}$. Since each $\mathcal{F}_i$ is a $\sigma$-algebra, $\bigcup_{n=1}^{\infty} A_n \in \mathcal{F}_i$ for all $i \in I$. Therefore, $\bigcup_{n=1}^{\infty} A_n \in \bigcap_{i \in I} \mathcal{F}_i = \mathcal{F}$. Hence, $\mathcal{F}$ is closed under countable unions.\\

   Since $\mathcal{F}$ contains $\mathcal{C}$, is closed under complementation, and is closed under countable unions, $\mathcal{F}$ is a $\sigma$-algebra.\\
\end{proof}


\begin{theorem}
    The smallest $\sigma$-algebra, $\sigma(\mathcal{C})$ is $\mathcal{F} = \bigcap_{i \in I} \mathcal{F}_i$, where each $\mathcal{F}_i$ is the $\sigma$-algebra of $\mathcal{C}$.  
 \end{theorem}
 
 The above theorem just implies that the smallest $\sigma$-algebra exists and is well defined. We don't know all $\mathcal{F}_i$, and we don't intend to find them as well. What we know for now is - that they exist. And if we take all of them and take a countable intersection of them - the resultant collection of sets is well-defined. 
 
 \begin{proof}
    Let $\{ \mathcal{F}_i \}_{i \in I}$ be the collection of all $\sigma$-algebras on a set $\mathcal{C}$ that contain $\mathcal{C}$. Define $\mathcal{F} = \bigcap_{i \in I} \mathcal{F}_i$. We know from a previous result that the intersection of an arbitrary number of $\sigma$-algebras is a $\sigma$-algebra. Therefore, $\mathcal{F} = \bigcap_{i \in I} \mathcal{F}_i$ is a $\sigma$-algebra.\\
 
    Since each $\mathcal{F}_i$ contains $\mathcal{C}$ by definition, their intersection, $\mathcal{F}$, also contains $\mathcal{C}$. Thus, $\mathcal{C} \subseteq \mathcal{F}$.\\
 
    Let $\sigma(\mathcal{C})$ denote the smallest $\sigma$-algebra containing $\mathcal{C}$. By definition, $\sigma(\mathcal{C})$ is a $\sigma$-algebra and contains $\mathcal{C}$. Therefore, it must be one of the $\mathcal{F}_i$ in the collection $\{\mathcal{F}_i\}_{i \in I}$. Since $\mathcal{F}$ is the intersection of all $\sigma$-algebras containing $\mathcal{C}$, it must be contained within any other $\sigma$-algebra that contains $\mathcal{C}$. In particular, $\mathcal{F} \subseteq \sigma(\mathcal{C})$.\\
 
    Since $\mathcal{F}$ is defined as the intersection of all $\sigma$-algebras that contain $\mathcal{C}$, and $\sigma(\mathcal{C})$ itself is one of these $\sigma$-algebras, it follows that $\sigma(\mathcal{C}) \subseteq \mathcal{F}$. We have $\mathcal{F} \subseteq \sigma(\mathcal{C})$ and $\sigma(\mathcal{C}) \subseteq \mathcal{F}$. Therefore, $\mathcal{F} = \sigma(\mathcal{C})$.\\
    
    This proves that the smallest $\sigma$-algebra containing $\mathcal{C}$ is $\mathcal{F} = \bigcap_{i \in I} \mathcal{F}_i$.
 \end{proof}
 
 
We can now define the \textit{Borel $\sigma$-algebra}. For this, we will have a setup - reasons of picking this setup will get clear in some time - when you will see that how open sets can be helpful in proving that even singleton sets are Borel sets.\\

\textbf{Setup:} Let $(0, 1]$ be the sample space, $\Omega$. The collection of interesting sets of $\Omega$, represented by $\mathcal{C}_0$, contains all open-intervals $(a, b)$ in $\Omega$.

\begin{definition}
   $\sigma(\mathcal{C}_0)$ is called the Borel $\sigma$-algebra, denoted by $\mathcal{B}((0, 1])$. An element of $\mathcal{B}((0, 1])$ is called Borel measurable set, or simply a Borel set.
\end{definition}

Thus, every open interval in $(0, 1]$ is a Borel set. We next prove that every singleton set in $(0, 1]$ is a Borel set too. 

\begin{theorem}
   Every singleton set in $(0, 1]$ is a Borel set.
\end{theorem}

\begin{proof}
   Let $((0, 1], \mathcal{B})$ be the Borel space where $\mathcal{B}$ is the Borel $\sigma$-algebra generated by the open sets within $(0, 1]$. We want to show that any singleton set $\{x\}$, where $x \in (0, 1]$, is a Borel set.\\
   
   Consider the singleton set $\{x\}$, where $x \in (0, 1]$. We can write it as:
   \[
   \{x\} = \bigcap_{n=1}^{\infty} \left( x - \frac{1}{n}, x + \frac{1}{n} \right) \cap (0, 1].
   \]
   The above result can be proved by the method of contradiction. Let $h$ be an element in $\bigcap_{n=1}^{\infty} \left( b - \frac{1}{n}, b + \frac{1}{n} \right)$ other than $b$. For every such $h$, there exists a large enough $n_0$ such that $h \notin \left( b - \frac{1}{n_0}, b + \frac{1}{n_0} \right)$. This implies $h \notin \bigcap_{n=1}^{\infty} \left( b - \frac{1}{n}, b + \frac{1}{n} \right)$.\\
   
   Each interval $\left( x - \frac{1}{n}, x + \frac{1}{n} \right) \cap (0, 1]$ is an open set in $(0, 1]$, and since Borel sets are generated by open sets, these intervals belong to the Borel $\sigma$-algebra $\mathcal{B}$.\\
   
   The Borel $\sigma$-algebra $\mathcal{B}$ is closed under countable intersections. Therefore, the intersection of the countable collection of open sets $\left( x - \frac{1}{n}, x + \frac{1}{n} \right) \cap (0, 1]$, which is exactly $\{x\}$, is also in $\mathcal{B}$.\\
   
   Since $\{x\}$ can be expressed as a countable intersection of open sets within $(0, 1]$, it is a Borel set.

\end{proof}

\begin{corollary}
   As an immediate consequence of this theorem, we see that every half-open interval, $(a, b]$, is a Borel set. This follows from the fact that
   \[
   (a, b] = (a, b) \cup \{b\},
   \]
   and the fact that a countable union of Borel sets is a Borel set. For the same reason, every closed interval, $[a, b]$, is a Borel set.
\end{corollary}

This also gives one the idea of how to prove a set is a Borel set or not. If the set can be represented as a complement of an open set or as countable unions and countable intersections of open sets, it is a Borel set.


\subsubsection{How big is the Borel $\sigma$-algebra?}

\begin{theorem}
   The cardinality of the Borel $\sigma$-algebra (on the unit interval $(0, 1]$) is the same as the cardinality of the reals. Thus, the Borel $\sigma$-algebra is a much ‘smaller’ collection than the power set $2^{(0, 1]}$.
\end{theorem}

\begin{proof}
   Let $\mathcal{B}$ denote the Borel $\sigma$-algebra on the unit interval $(0, 1]$. \\

\textbf{Step 1: Show that the cardinality of $\mathcal{B}$ is at most the cardinality of the reals.}\\

The Borel $\sigma$-algebra $\mathcal{B}$ is generated by the open intervals of $(0, 1]$, which form a basis for the topology. The set of all open intervals in $(0, 1]$ has the same cardinality as $\mathbb{Q} \times \mathbb{Q}$, where $\mathbb{Q}$ is the set of rational numbers. Since $\mathbb{Q}$ is countable, it follows that the set of all open intervals in $(0, 1]$ is also countable.\\

The Borel $\sigma$-algebra is generated by applying countable unions, intersections, and complements to these open intervals. Therefore, the number of sets that can be formed is bounded by $|\mathbb{R}|$, the cardinality of the reals.\\

\textbf{Step 2: Show that the cardinality of $\mathcal{B}$ is at least the cardinality of the reals.}\\

Consider the singleton sets $\{x\}$ where $x \in (0, 1]$. Each singleton set is a Borel set, and the cardinality of these singleton sets is the same as the cardinality of the reals. Therefore, $\mathcal{B}$ must contain at least as many elements as the cardinality of $\mathbb{R}$.\\

\textbf{Conclusion:} We have shown that the cardinality of $\mathcal{B}$ is both at most and at least the cardinality of the reals. Therefore, the cardinality of the Borel $\sigma$-algebra $\mathcal{B}$ is exactly $|\mathbb{R}|$.\\

Since the power set $2^{(0, 1]}$ has a cardinality of $2^{|\mathbb{R}|}$, which is strictly greater than $|\mathbb{R}|$, it follows that the Borel $\sigma$-algebra is a much smaller collection than the power set of $(0, 1]$.
\end{proof}


\subsection{What are not Borel sets?}

The majority of sets in $(0, 1]$ are Borel sets. In fact, the Borel $\sigma$-algebra on $(0, 1]$ contains a wide range of sets, from simple open intervals to much more complex constructions. Identifying a non-Borel set is not trivial because the Borel $\sigma$-algebra is quite extensive.The Borel $\sigma$-algebra on $(0, 1]$ includes many intricate sets such as the \textbf{Cantor set}. To understand the breadth of the Borel $\sigma$-algebra, we first prove that the Cantor set is a Borel set.

\begin{lemma}
   The Cantor Set is a Borel set.
\end{lemma}

\begin{proof}
   The Cantor set, $C$, is constructed by iteratively removing the middle third from each interval of $(0, 1]$. 
\[
C = \bigcap_{n=1}^{\infty} C_n,
\]
where $C_n$ is the set obtained after the $n$-th stage of removing the middle third of each interval.\\

Each $C_n$ is a finite union of closed intervals. Since finite unions of closed sets are closed, $C_n$ is closed for each $n$. The Cantor set $C$ is the countable intersection of these closed sets. The Borel $\sigma$-algebra is closed under countable intersections of closed sets, so $C \in \mathcal{B}$, making the Cantor set a Borel set.\\
\end{proof}

\textbf{Examples of Non-Borel Sets}\\

Although most familiar sets are Borel sets, there exist sets that are not Borel. These sets are usually constructed using the Axiom of Choice and involve more intricate arguments. One such example is the \textbf{Vitali set}. The Vitali set is constructed in the following way:\\

1. Consider the interval $[0, 1]$.\\
2. Define an equivalence relation $\sim$ on $[0, 1]$ by $x \sim y$ if and only if $x - y \in \mathbb{Q}$, i.e., $x$ and $y$ differ by a rational number.\\
3. By the Axiom of Choice, we select exactly one representative from each equivalence class under this relation. The collection of these representatives forms a set $V$, called the \textbf{Vitali set}.\\

\textbf{But what is the Vitali Set actually?}\\

Imagine you're organizing the quirkiest party ever on a number line between 0 and 1. This isn't just any party - it's a Vitali set party! Here's how you create your guest list: \\

First, you declare that two numbers are \textit{dance partners} if their difference is a rational number. For example, 0.3 and 0.7 are dance partners because 0.7 - 0.3 = 0.4, which is rational. Now, you start grouping all the numbers between 0 and 1 into \textit{dance troupes.} Each troupe consists of all numbers that are dance partners with each other. Here's the twist: you decide to invite exactly one person from each dance troupe to your party. It doesn't matter who you choose from each troupe, as long as you pick one and only one. \\

The resulting guest list is what mathematicians call a Vitali set! Why is this party so special? Well, it has some mind-bending properties: \\

No two party guests are exactly one rational number apart. If Alice is at 0.3 and Bob is at 0.7, one of them didn't make the cut because they're in the same dance troupe. Yet, if you shift all your guests by any rational number, you'll get a completely new set of partiers, with no overlap with the original group! Despite seeming quite sparse (remember, we only chose one member from each dance troupe), this set has some very strange measuring properties, as we'll soon see. \\

Now that we've got our Vitali set party set up, let's explore why it's such a mathematical troublemaker! \\

\textbf{Why is the Vitali Set Not a Borel Set?}\\

Now, let's play a game called \textit{Cover the Dance Floor.} \\

Here's how it works: We start with our Vitali set party on the (0,1] dance floor. We're given a magical dance move: we can shift everyone simultaneously by any rational number between -1 and 1. Our goal? Use these dance moves to cover every spot on a new, bigger dance floor from 0 to 2!\\

Here's the kicker: with the right series of these rational shifts, we can indeed cover every single point between 0 and 2. It's like our original Vitali set party has suddenly expanded to fill twice the space! But wait a minute... if the Vitali set were a Borel set (think of Borel sets as \textit{well-behaved} sets that play nicely with measure theory), we'd run into a big problem. Here's why:\\

Borel sets have a property: if you take a Borel set and shift it by a rational number, the result is still a Borel set. We just covered the interval (0,2] using countably many rational shifts of our Vitali set.
If the Vitali set were Borel, this new covered area would also be Borel. But here's the contradiction: we know the measure (think \textit{length}) of (0,2] is 2, but it's made up of countably many copies of our original set, each of which should have the same measure as the original Vitali set.\\

Let's call the measure of the Vitali set $m$. Then we have:
\[
2 = \text{countably many} \times m
\]
This equation can't possibly work! If $m = 0$, the right side is zero. If $m > 0$, the right side is infinite. There's no value of $m$ that makes this equation true. So, we're forced to conclude that our initial assumption - that the Vitali set is a Borel set - must be wrong. The Vitali set is too \textit{wild} to be captured by the well-behaved Borel sets.
